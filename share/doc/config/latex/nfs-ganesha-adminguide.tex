\documentclass[a4paper]{book}
\usepackage{a4wide}
\usepackage{makeidx}
\usepackage{fancyhdr}
\usepackage{graphicx}
\usepackage{multicol}
\usepackage{float}
\usepackage{textcomp}
\usepackage{alltt}

\makeindex
\setcounter{tocdepth}{2}
\renewcommand{\footrulewidth}{0.4pt}
\begin{document}


\begin{titlepage}
\vspace*{7cm}
\begin{center}
{\Large GANESHA NFS server}\\
\vspace*{1cm}
{\large Administration guide \\[1ex]\large v0.4}\\
\vspace*{0.5cm}
{\small Wed Oct 8, 2008}\\
\end{center}
\end{titlepage}

\pagenumbering{roman}
\tableofcontents

\pagenumbering{arabic}
\chapter{How to write the GANESHA's configuration file}


\section{Overall view of the configuration file}

The configuration file for the GANESHA daemon may seemed huge at first glance.
It is separated into several blocks. A block contain several items.\\

A block will look like this: 

\begin{verbatim}
    <BLOCK_TAG>
    {
       <Config_Item_Tag> = <Value> ;

       # Comment 
       <Config_Item_Tag> = <Value> ;
	    .
	    .
	    .
       <Config_Item_Tag> = <Value> ;
    }
\end{verbatim}

As we will see, there are many different blocks, each of them dedicated
to the configuration of a specifc stuff in GANESHA. Not all the blocks 
are mandatory and some may not be explicitely specified. In this case, default
values will be kept> 
Inside the blocks are items which are a way to set a internal constant in the daemon
with direct effect to the program's behaviour. Like the blocks, not all the items are mandatory
and some may be lacking.
The following sections will described each blocks and items and explain all of them and their use.\\

Block's names and item's name matching algorithm is case insensitive. Naming a block EXPORT or Export or 
export work as well.\\

Every character below a '\#' sign is a comment and will be ignored
(except if it is inside a quoted string, or escaped with a backslash).\\

You can also include annexe configuration files using \texttt{\%include} statement:
\begin{verbatim}
	E.g:
	%include "annexe_file.conf"
\end{verbatim}

\begin{tabular}{|p{3.5cm}|p{3cm}|p{7.5cm}|}
\hline
{\bf Block Name }   &  {\bf Status }						& {\bf Used for... }															\\
\hline
EXPORT   & Mandatory. Can be duplicated			 & Set an export entry                 \\
\hline
FSAL   	& Recommended. One instance.		 & FSAL generic configuration      	�   \\
\hline
FileSystem			& Optionnal. One Instance   	& Filesystem's behaviour         \\
\hline
HPSS   	& Recommended. One instance   & FSAL\_HPSS's specific configuration (only with hpss.ganesha.nfsd)	\\
\hline
POSIX       & Recommended. One instance   & FSAL\_POSIX's specific configuration (only with posix.ganesha.nfsd)  \\
\hline
SNMP       & Recommended. One instance   & FSAL\_SNMP's specific configuration (only with snmp.ganesha.nfsd)  \\
\hline
NFS\_Worker\_Param	    & Recommended. One instance   & Worker Thread configuration      		 \\
\hline
NFS\_Core\_Param		& Recommended. One instance   & Daemon's core configuration      		 \\
\hline
NFS\_DupReq\_Hash		& Optionnal. One Instance   	& RPC Duplicate Request Hash configuration   			\\
\hline
NFS\_IP\_Name			& Optionnal. One Instance   	& IPadress $\leftrightarrow$ hostname resolution cache configuration   \\
\hline
NFSv4\_ClientId\_Cache  & Optionnal. One Instance   	& Configuration of the NFSv4 Clientid cache   			 \\
\hline
NFSv4\_StateId\_Cache  & Optionnal. One Instance   	& Configuration of the NFSv4 State id cache   			 \\
\hline
NFS\_KRB5   & Recommended for using KRB5. One instance & KRB5 configuration. Needed only if sys=krb5, krb5i or krb5p is used  \\
\hline
NFSv4       & Recommended. One instance   & NFSv4 Specific configuration items (e.g. lease lifetime)			\\
\hline
CacheInode\_Hash		& Recommended. One instance   & Configuration of the Cache Inode Hash Table   		 \\
\hline
CacheInode\_Client	    & Recommended. One instance   & The configuration of the workers's clients to the Cache Inode		 \\
\hline
CacheInode\_GC\_Policy  & Recommended. One instance   & Garbagge Collection Policy for the Cache Inode entries   \\
\hline
FileContent\_Client	& Recommended. One instance   & The configuration of the workers's client to the File Content Cache \\
\hline
FileContent\_GC\_Policy & Recommended. One instance   & Garbagge Collection Policy for the File Content cache entries		 \\
\hline
BUDDY\_MALLOC			& Recommended. One instance   & Configuration of the internal memory manager   		\\
\hline
\end{tabular}


\section{What to put in each block}

	\subsection{The EXPORT Block}

	This tag is used to describe the way a FSAL tree is exported via NFS. It describes clients, who has root access, which versions of the
	protocols to be used and transport layers.\\

	\subsubsection{How to describe clients ?}

        Clients can be single machines, pack of machines, netgroups,
        and networks.\\
	
		Single machine is identified by its name or ip address: e.g. localost or 127.0.0.1\\
    
		Sets of machines: you may have "farms" of machines with similar
			name like "cluster0, cluster0, ..., cluster-gateway". With a
			cluster, you can have bunches of such names. Using "unix
			jokers" can help so far. GANESHA allows you to specify names
			like "cluster*" or "cluster1?", using the old style "shell
			joker syntax"\\
			You can also specify ranges of machines using square brackets.
			For example, "machine[1-3,7-9,12]-eth1" will be interpreted as "machine1-eth1, machine2-eth1, machine3-eth1, machine7-eth1, ..."\\
    
		Networks: this should be a network name (you may have a
				/etc/networks describing them) or a reduced ip
				address describing addresses belonging to this
			networks. For example 12.13.0.0 will describe machines
			whose address is 12.13.x.y with netmask 0xffff0000\\
    
		Netgroups: clients can be identified by the fact they belong
			to a given netgroup. This is shown by adding an arobas
			at the beginning of the netgroupname. To tell GANESHA
			to provide access to machine in netgroup "nfsclients" specify "@nfsclients"\\

		Lists of clients is comma separated: "localhost,12.13.0.0,@nfsclient"\\

	\subsubsection{Export parameters}

	NB: Only 'Path' , 'Export\_Id' and 'Pseudo' are mandatory keys.\\

\begin{itemize}

\item  Export\_Id : This tag is used to set the the id for this export. This
		is mostly used internally, but this is a mandatory value. The value is to be a
		non-zero integer value\\
		e.g: Export\_Id = 1 ;

\item  Access: The list if clients that can access the export entry\\
		e.g: Access = "Machine3,Machine10*,NetworkB,@netgroupY";\\
		NB: Using Access = "*" ; will allow everybody to access the
		mount point.

\item  Root\_Access: The list of clients (see above) that have root access
		rights (root remains root when crossing the mount point).\\
		e.g: Root\_Access = "localhost,12.13.0.0,@nfsclient" ;

\item  Access\_Type: Describes the kind of access you can have of the mount
		point. Acceptable values are:
        \begin{itemize}
			\item RO: Read-Only mount point
			\item RW: Read-Write mount points
			\item MDONLY: The mount point is RW for metadata, but data
				accesses are forbidden (both read and write). This
				means you can do everything on md, but nothing on
				data.
			\item MDONLY\_RO: Like RO, but reading data is forbidden.
				You can read only metadata.
        \end{itemize}                
		e.g: Access\_Type = "RW" ;

\item  Anonymous\_root\_uid : The uid to be used for root when no root access
		was specified for this clients. This is the definition of the
		'nobody' user. "Traditional" value is -2\\
		e.g: Anonymous\_root\_uid = -2 ;
	
\item  NOSUID: a flag (with value TRUE or FALSE) showing if setuid bit is
		kept. \\
        e.g : NOSUID= TRUE;
	
\item  NOSGID: a flag (with value TRUE or FALSE) showing if setgid bit is
		kept. \\
        e.g : NOSGID= TRUE;

\item  NFS\_Protocols: The version(s) of the NFS protocol you can use for
		mounting this export entry\\
		e.g: NFS\_Protocols = "2,3,4" ;

\item  Transport\_Protocols: The transport layer to use for mount this entry
		This should be UDP or TCP or a list.\\
		e.g : Transport\_Protocols = "UDP,TCP" ;

\item  Sec\_type: List of supported RPC\_SEC\_GSS authentication flavors for this export.
  		 It can be a coma-separated list of the following values: sys, krb5i, krb5p.\\
  		e.g : SecType = "sys,krb5";


\item  MaxOffsetRead: Maximum offset allowed for a read operation (no limit if not specified). This could be
		useful for prevently from "evil" use of NFS read (like access a TB long file via NFS)\\
  		e.g : MaxRead = 409600;

\item  MaxOffsetWrite: Like MaxOffsetRead, but for Write operation\\
  		e.g : MaxWrite = 409600;

\item  MaxRead, MaxWrite, PrefRead, PrefWrite, PrefReaddir: The value to be
		returned to client when NFS3\_FSINFO is called. Better solution is not to use
		this keys and so keep the default values, optimized for NFSv3.

\item  Filesystem\_id: The filesystem id to provide to the client for this
		export entry. NFS Client will use this value to address their
		internal metadata cache. In NFSv4, both major and minor values
		are used, in NFSv2 and NFSv3, only the major is used.\\
  		e.g. : Filesystem\_id = 100.1 ;

\item  PrivilegedPort: A flag (TRUE or FALSE) to specify is a client using an
		ephemere port should be rejecting or not.\\
  		e.g. PrivilegedPort = FALSE ;

\item  Cache\_Data: A flag (TRUE or FALSE) To specify if files content should
		be cached by the GANESHA server or not\\
  		e.g : Cache\_Data = TRUE ;

\item MaxCacheSize: if export entry is datacached, this value defines the
        	maximum size of the files stored in this cache.


\item  FS\_Specific: a comman separated list of information used by the FSAL
		to perform initialization. See FSAL documentation for detail.\\
		e.g. (for HPSS/FSAL): FS\_Specific = "cos=1" ;

\item  Path: The path to export via NFS. Should have a leading slash.\\
		e.g: Path = "/nfs/my\_exported\_directory" ;

\item  Tag: A way of providing a shorter path for mounting an entry. 
		For example, you could mount an entry like this:
        \begin{verbatim}mount -o vers=3 nfsserver:/nfs/my_exported_directory /mnt \end{verbatim}
		But if you specified "Tag = ganesha;", you can simply do
        \begin{verbatim} mount -o vers=3 nfsserver:ganesha /mnt \end{verbatim}

\item  Pseudo: a NFSv4 specific key that shows the path, in NFSv4 pseudo file
		system were the 'actual' mount point resides.\\
		e.g : Pseudo = "/nfsv4/pseudofs/nfs\_mount\_entry\_\#1" ;

\end{itemize}

	\subsection{The FSAL Block}

	This block provides with general information about the FSAL. A given
	FSAL could use other information, more specific, but not from this
	block which remains generic.

\begin{itemize}

\item  DebugLevel: the level of verbosity for FSAL logs. Acceptable values
			are:
            \begin{itemize}
			\item NIV\_NULL (no logging)
			\item NIV\_MAJ  (log only major events)
			\item NIV\_CRIT (log only erroneous events)
			\item NIV\_EVENT (log only important events)
			\item NIV\_DEBUG (log all events)
			\item NIV\_FULL\_DEBUG (log all internal processing)
            \end{itemize}
		e.g: DebugLevel = "NIV\_EVENT" ;
	
\item   LogFile: The path to use for the logfile. \\
		e.g: LogFile = "/var/log/ganesha.fsal.log" ;

\item  max\_FS\_calls: FSAL provides a way (with a semaphore) to limit the
		number of requests performed by the worker threads. This is useful to
		prevent from 'FS flood via NFS'. If none is specify, no limits
		exists and the semaphore will remain usused.\\
		e.g : max\_FS\_calls = 20 ;

\end{itemize}

    \subsection{The FileSystem Block}

  This block describes the beahavior of the file system on some points.
  
\begin{itemize}

\item  MaxRead: Maximum read buffer size for this filesystem.\\
		e.g. : MaxRead = 1048576 ;

\item  MaxWrite: Maximum write buffer size for this filesystem\\
		e.g. : MaxWrite = 1048576 ;

\item  Umask: If set, this mask is applied on the mode of created objects.
			 This should be an octal value, do not forget the leading 0. \\
		e.g.: Umask = 0002 ;

\item  CanSetTime, Symlink\_support, Link\_support:
		Three flags (TRUE or FALSE) to specify if hard links, symbolic
		links are allowed or if time is settable.

\item  auth\_xdev\_export: A flag (TRUE of FALSE) to specify if crossing
		junction is allowed or not.

\end{itemize}


    \subsection{The HPSS Block (Only for use with HPSS/FSAL)}
	
	This is the items needed for configuring the HPSS CLAPI that 
	the HPSS/FSAL uses for its calls.

\begin{itemize}

\item  AuthMech: CLAPI Authentication mechanism. Should be "krb5" or "unix"\\
		e.g. : AuthMech = "krb5" ;

\item  PrincipalName: The Auth principal to be used by the HPSS/FSAL to run
		CLAPI calls. Principal hpssfs is strongly suggested.\\
		e.g. : PrincipalName = "hpssfs";

\item  KeytabPath: The Keytab file to be use for acquiring principal identify\\
		e.g. : KeytabPath = "/var/hpss/etc/hpss.keytabs" ;

\item  CredentialLifetime : The duration in seconds after which a CLAPI credential is to
		be renewed.\\
		e.g. : CredentialLifetime = 3600 ;

\item  NumRetries: Number of retries in CLAPI for connection failure with
		HPSS. A value of 0 means no retry, a value of -1 means
		"retry forever". This last value may cause troubles
		because worker threads may hang. A value of 100 is a good
		compromise.\\
		e.g : NumRetries = 100 ;

\item  BusyDelay: retry delay (in seconds) if core server is busy\\
		e.g.: BusyDelay = 1 ;

\item  BusyRetries: Number of retries when core is busy (0: no retry, -1: retry forever)\\
		e.g. : BusyRetries = -1;

\item  MaxConnections : Maximum number of connection to the Core Server
		(check HPSS configuration for a coherent value)\\
		e.g. : MaxConnections = 100;

\end{itemize}

    \subsection{The POSIX Block (only for use with a posix filesystem)}

	This block is related to POSIX/FSAL, it contains the necessary items
	to properly init the FSAL built on top of POSIX calls.

\begin{itemize}

\item  DB\_Host : The hostname for the machine running the DB engine\\
		e.g. : DB\_Host = "dbserver.localdomain" ;

\item  DB\_Port : The port to use for contacting the DB engine\\
		e.g. : DB\_Port = 5432;

\item  DB\_Name : The name for the database to use for the POSIX/FSAL\\
		e.g. : DB\_Name = FSAL\_POSIX\_DB ;

\item  DB\_Login : The username to use to connect to the database\\
		e.g.: DB\_Login = DB\_USER ;
	
\item  DB\_keytab : The keytab to use to acquire the 'DB\_Login' identity.\\
		e.g : DB\_keytab = "/var/etc/posix.db.keytab" ;

\end{itemize}
	

    \subsection{The SNMP Block (Only for use with SNMP/FSAL)}
	
	This block is related to SNMP/FSAL, it contains necessary items
	to properly init the FSAL built on top of a SNMP tree.

\begin{itemize}

\item snmp\_version: this indicates the SNMP protocol version that GANESHA
      will use for communicating with SNMP agent. Expected values are
      \texttt{1}, \texttt{2c} or \texttt{3}.
      Default is \texttt{"2c"}.
\item snmp\_server: this is the address of the SNMP master agent. A port number
      can also be specified by adding {}``:<port>'' after the address.
      Default is \texttt{"localhost:161"}.
\item community: this is the SNMP community used for authentication.
      Default is \texttt{"public"}.
\item nb\_retries: number of retries for SNMP requests.
      Default value is\\
      \texttt{SNMP\_DEFAULT\_RETRIES}, defined in net-snmp library.
\item microsec\_timeout: number of microseconds until first timeout,
      then an exponential backoff algorithm is used for next timeouts.
      Default value is \texttt{SNMP\_DEFAULT\_TIMEOUT}, defined in net-snmp library.
\item client\_name: this is the client name that could be used internally
      by net-snmp for its traces.
      Default value is \texttt{"GANESHA"}.
\item snmp\_getbulk\_count: this indicates the number of responses wanted
      for each SNMP GETBULK request.
      Default value is \texttt{64}.

\end{itemize}



    \subsection{The BUDDY\_MALLOC Block}

	GANESHA manages its own memory on its own way. This is pretty useful
	for memory use optimization. The algorithm is the Budd Block
	algorithm. Each thread in GANESHA manages the memory it uses via this
	method.

\begin{itemize}

\item  Page\_Size: The size of a page. This MUST be a power of 2.\\
		e.g. : Page\_Size = 8388608;

\item  Enable\_OnDemand\_Alloc: a flag (TRUE or FALSE) to specify if a thread
		can extend its number of page when it lacks memory. Value of TRUE is
		strongly recommended.\\
		e.g. : Enable\_OnDemand\_Alloc = TRUE ;

\item  Enable\_Extra\_Alloc :  a flag (TRUE or FALSE) to specify if buddy memory manager
  		 allow threads to alloc memory areas that are larger than
		Page\_sSize value.\\
		e.g. : Enable\_Extra\_Alloc = TRUE;

\item  Enable\_GC :  a flag (TRUE or FALSE) to specify if buddy memory manager
  		can release unused pages, according to  GC\_Keep\_Factor and GC\_Keep\_Min parameters.
		TRUE is strongly recommended.\\
		e.g. : Enable\_GC = TRUE;
		
\item  GC\_Keep\_Factor : Buddy's GC must keep at least GC\_Keep\_Factor times the current number of used pages.\\
		e.g. : GC\_Keep\_Factor = 2;

\item  GC\_Keep\_Min: GC must keep at least this number of pages.\\
		e.g. : 	GC\_Keep\_Min = 2;

\item  LogFile: The path of the log file for BUDDY\_MALLOC messages (not very
		verbose, shows only critical behaviours).\\
		e.g. : LogFile: /var/log/ganesha.buddy\_malloc.log ;
\end{itemize}
		

    \subsection{The 'CacheInode\_Hash' block}

	This block defines the behavior of hash table used
	for the internal metadata cache.

	It consists of the following key/value peers:

\begin{itemize}

\item  Index\_Size:
		The size of the hash table.
		This MUST be a prime number, greater enough compared to the number of worker threads
		(specified in the 'NFS\_Core\_Param' block)

\item  Alphabet\_Length:
		A parameter for the hashing algorithm.
		This must be set to the number of possible values
		for each byte of the underlying filesystem's handle, i.e. 256.
		However, if you notice a bad balancing in your hash tables,
		you can try decreasing this value (but it should not exceed 256).

\item  Prealloc\_Node\_Pool\_Size:
		For better performances, each slot of the hash table
		consists of a Red-Black Tree. Thus, this parameter is
		the number of preallocated RBT entries for each worker thread.
		As a result, this must be set to a value that is close to
		the forecasted number of metadata entries, divided by the number
		of worker threads.\\
		e.g.: for 300k entries, and 30 worker threads, we'll have about 10k entries by thread.
		Thus, a value between 2500 and 10000 should be set (it will result in
		1 to 4 memory allocations during the whole life of the thread)
\end{itemize}


    \subsection{The 'CacheInode\_Client' block}

	This block defines the behavior of the metadata cache.
	It consists of the following key/value peers:

\begin{itemize}

\item  LogFile:
		The file where the metadata cache events are logged.

\item  DebugLevel:
		The verbosity level for the metadata cache log.
		The values can be:
        \begin{itemize}
			\item NIV\_NULL (no logging)
			\item NIV\_MAJ  (log only major events)
			\item NIV\_CRIT (log only erroneous events)
			\item NIV\_EVENT (log only important events)
			\item NIV\_DEBUG (log all events)
			\item NIV\_FULL\_DEBUG (log all internal processing)
        \end{itemize}
        
\item  LRU\_Nb\_Call\_Gc\_invalid:
		Each worker maintains a LRU list of the last entries it handled.
		When a worker handles an entry, it sets it invalid in other
		threads' LRU's, in order to make them garbaging it.
		This parameter so defines the periodicity for garbaging invalid
		entries in this list. Thus, a worker will garbage its own list
		after processing this amount of NFS requests.

\item  LRU\_Prealloc\_PoolSize:
		This parameter sets the number of LRU entries that
		are preallocated for each worker thread.
		Given that the total amount of theses entries equals
		the number of cache entries + a certain working set
		(number of working threads * LRU\_Nb\_Call\_Gc\_invalid),
		this parameter must be close to the forecasted number
		of metadata entries divided by the number of worker threads,
		+ the value of LRU\_Nb\_Call\_Gc\_invalid.

\item  Entry\_Prealloc\_PoolSize:
		This parameter is the number of preallocated cache entries
		for each worker thread. It must be close to the forecasted number
		of metadata entries divided by the number of worker threads. 

\item  DirData\_Prealloc\_PoolSize:
		This parameter is the number of preallocated directory cache entries
		for each worker thread. It must be close to the number of directories
		in the exported filesystem divided by the number of worker threads.
  
\item  ParentData\_Prealloc\_PoolSize:
		This parameter is the number (for each thread) of preallocated
		entries' references to their parents (for example, a hard linked
		object can have several parents).\\
		It must be close to the forecasted number of metadata entries
		divided by the number of worker threads, except if the filesystem
		contains a huge amount of hard links (you should then multiply
		this value by the average number of hard links on each object).
 
\item State\_v4\_Prealloc\_PoolSize:
                This parameter is the number (for each thread) of preallocated
                state to be used by each worker for NFSv4 State management.\\
                It is used only if NFSv4 is used.

\item  Attr\_Expiration\_Time:
		The expiration delay (in seconds) for cached attributes.
		A value of "0" disables attributes cache expiration.
  
\item  Symlink\_Expiration\_Time:
		The expiration delay (in seconds) for symbolic link content.
		A value of "0" disables symlink cache expiration.

\item  Directory\_Expiration\_Time:
		The expiration delay (in seconds) for directory content.
		A value of "0" disables directory cache expiration.

\item  Use\_Getattr\_Directory\_Invalidation:  
		This boolean indicates if a cached directory content is invalidated
		when its mtime has changed on the underlying filesystem.
		Setting this parameter to TRUE will result in an extra "getattr" operation
		on the filesystem for each NFS "readdir" call, so it could strongly impact
		the readdir performances.\\
		However, it must be set if your filesystem tree is continuously modified
		by an external actor (another NFS server, ...)
		and if you need a good re-synchronisation of GANESHA's NFS server cache.

\item  Use\_Test\_Access:
		If set to TRUE (strongly recommended), NFS "access" calls will be
		performed according to the cached attributes (mode, group, owner,...).\\
		Else, each NFS "access" operation will result in an "access" call
		to the underlying filesystem.

\item  Use\_OpenClose\_cache:
		If this boolean is set to TRUE, files will not be opened and closed
		at each read/write NFS call: GANESHA will cache a certain amount of
		opened file descriptors for better I/O performances (recommended).

\item  Max\_Fd:
		When datacaching is disabled and 'Use\_OpenClose\_cache' is enabled,
		this parameter indicates the maximum number of files that can be kept
		openned for each thread.\\
		NB: when datacaching is enabled, use the FileContent\_Client::Max\_fd parameter instead

\item  OpenFile\_Retention:
		When datacaching is disabled and 'Use\_OpenClose\_cache' is enabled,
		this parameter indicates the minimum time (in seconds) a file must
		be kept opened.\\
		If the 'max\_fd' limit is reached and all files have been opened
		more recently than the 'OpenFile\_Retention' time, no more file descriptors
		are cached until the previous ones are older than this value.

\item Async\_Op\_Prealloc\_Poolsize (write back cache only):
                The number of structure representing an asynchronous operation
                to be preallocated.

\item Nb\_Synclet (write back cache only):
                The number of synclet to be started.

\end{itemize}

    \subsection{The 'CacheInode\_GC\_Policy' block}

	This section defines the garbage collection policy
	for the metadata cache.

	In GANESHA, each worker thread ensures the garbage collection
	of the cache entries it was the last to deal with.

	The following parameters describe the conditions that must be
	meet for a worker, in order for it to launch a garbage collection
	of its entries.

\begin{itemize}

\item  Nb\_Call\_Before\_GC:
		This parameter indicates, for each worker thread,
		the number of NFS calls it has to process between
		checking for garbage collection conditions.

\item  Runtime\_Interval:
		The period (in seconds) for checking cache high-watermark.
		Thus, a worker thread does not check GC conditions
		until the interval since the last garbage collection
		has not been enlapsed.

\item  NbEntries\_HighWater:
		Garbage collections are launched only if the number of entries
		in the cache is over this value.

\item  NbEntries\_LowWater:
		A garbage collection stops when the number of entries
		in the metadata cache falls to this value.

\item  File\_Lifetime:
		the minimum delay (in seconds) a file has not been accessed,
		for beeing a candidate to metadata cache garbage collection.\\
		A value of "-1" disables files garbage collection.

\item  Directory\_Lifetime:
		the minimum delay (in seconds) a directory has not been accessed,
		for beeing a candidate to metadata cache garbage collection.
		Note that a directory is not garbaged until all its childs
		have not been garbaged before.\\
		A value of "-1" disables directory garbage collection.
\end{itemize}


    \subsection{The 'FileContent\_Client' config block}

	This block sets the behavior for the datacache.
	It consists of the following key/value peers:

\begin{itemize}

\item  LogFile:
		The file where the datacache events are logged.

\item  DebugLevel:
		The verbosity level for the datacache log.
		The values can be:
        \begin{itemize}
			\item NIV\_NULL (no logging)
			\item NIV\_MAJ  (log only major events)
			\item NIV\_CRIT (log only erroneous events)
			\item NIV\_EVENT (log only important events)
			\item NIV\_DEBUG (log all events)
			\item NIV\_FULL\_DEBUG (log all internal processing)
        \end{itemize}

\item  LRU\_Nb\_Call\_Gc\_invalid:
		Each worker maintains a LRU list of the last entries it handled
		for read/write operations through the datacache.
		When a worker accesses an entry, it sets it invalid in other
		threads' LRU's, in order to make them garbaging it.
		This parameter so defines the periodicity for garbaging invalid
		entries in this list. Thus, a worker will garbage its own list
		after processing this amount of read/write operations in the cache.

\item  LRU\_Prealloc\_PoolSize:
		This parameter sets the number of LRU entries that
		are preallocated for each worker thread.
		Given that the total amount of theses entries equals
		the number of datacached entries + a certain working set
		(number of working threads * LRU\_Nb\_Call\_Gc\_invalid),
		this parameter must be close to the number of datacached
		files divided by the number of worker threads,
		+ the value of LRU\_Nb\_Call\_Gc\_invalid.

\item  Entry\_Prealloc\_PoolSize:
		This parameter is the number of preallocated datacache entries
		for each worker thread. It must be close to the forecasted number
		of entries that are to be datacached, divided by the number
		of worker threads.

\item  Cache\_Directory:
		The local directory where the GANESHA's datacache is to be stored.

\item  Refresh\_FSAL\_Force: 
		Force to refresh a datacached file when it has been manually
		modified in the cache ???

\item  Use\_OpenClose\_cache:
		If this boolean is set to TRUE, cached files will not be opened and closed
		at each read/write NFS call: GANESHA will cache a certain amount of
		opened file descriptors for better I/O performances (recommended).

\item  Max\_Fd:
		When datacaching and 'Use\_OpenClose\_cache' are enabled,
		this parameter indicates the maximum number of files that can be kept
		openned for each thread.\\
		NB: when datacaching is disabled, use the CacheContent\_Client::Max\_fd parameter instead

\item  OpenFile\_Retention:
		When datacaching and 'Use\_OpenClose\_cache' are enabled,
		this parameter indicates the minimum time (in seconds) a file must
		be kept opened.\\
		If the 'max\_fd' limit is reached and all files have been opened
		more recently than the 'OpenFile\_Retention' time, no more file descriptors
		are cached until the previous ones are older than this value.

\end{itemize}

    \subsection{The 'FileContent\_GC\_Policy' block}

	This section defines the garbage collection policy
	for the data cache.

\begin{itemize}

\item  Emergency\_Grace\_Delay:
		When doing a global datacache flush (option -F of the ganesha), files who are younger
		than this delay will neither be flushed nor removed from the cache.

\item  Lifetime:
		the minimum delay (in seconds) a file has not been accessed,
		for beeing a candidate to flush and removal from the datacache.\\
		A value of "-1" disables data flushing.

\item  (not implemented yet) Inactivity\_Before\_Flush:
		Will be used for automatic flushing (without removal).

\item  Df\_HighWater:
		When the local datacache filesystem usage is over this value (in percent),
		a garbage collection of the datacache is launched.

\item  Df\_LowWater:
		A datacache garbage collection stops when the filesystem usage
		falls to this value (in percent).

\item  Runtime\_Interval:
		The interval between checking datacache filesystem usage.

\item  Nb\_Call\_Before\_GC:
		This parameter indicates, for each worker thread,
		the number of read/write calls through the datacache
		it has to process between checking for filesystem usage.
\end{itemize}


    \subsection{The 'NFS\_Worker\_Param' block}

	This section consists of the parameters for worker threads.
\begin{itemize}

\item  Pending\_Job\_Prealloc:
		Each worker has a queue of pending jobs (requests that
		have been received from NFS clients, and that have to be
		processed yet).\\
		This parameter indicates the number of preallocated
		pending jobs for each worker thread.
		This must be close to the 'Nb\_Before\_GC' parameter
		+ the potential size of requests floods.

\item  LRU\_Pending\_Job\_Prealloc\_PoolSize:
		The pending jobs are sorted into a LRU list.
		Thus, this parameter should be equal to the 'Pending\_Job\_Prealloc' parameter.

\item  Nb\_Before\_GC:
		When a worker is processing a flood of requests,
		it does not clean its pending jobs immediatly.
		The 'Nb\_Before\_GC' parameter indicates how many requests
		a worker must process before cleaning its queue.

\item  Nb\_DupReq\_Prealloc:
		This indicates the number of preallocated entries
		(for each worker) for the duplicate request cache.
		This must be close to the total number of requests
		stored in it, divided by the number of workers.
		This greatly depends on the number of NFS requests/sec
		and the lifetime of this cache (see the NFS\_Core\_Param::DupReq\_Expiration parameter)

\item  LRU\_DupReq\_Prealloc\_PoolSize:
		The pending requests are sorted into a LRU list.
		Thus, this parameter should be equal to the 'Nb\_DupReq\_Prealloc' parameter.

\item  Nb\_DupReq\_Before\_GC:
		The 'Nb\_DupReq\_Before\_GC' parameter indicates how many requests
		a worker must process before garbaging the duplicate request cache.

\item  Nb\_IP\_Stats\_Prealloc:
		The number of preallocated entries for the IP to statistics cache.
		This must be close to the number of client nodes.

\item  Nb\_Client\_Id\_Prealloc:
		The number of preallocated entries for client information cache in NFSv4.
		This must be close to the number of client nodes.
\end{itemize}


    \subsection{The 'NFS\_Core\_Param' block}

	This section gives general parameters for the NFS deamon.

\begin{itemize}

\item  Nb\_Worker:
		The number of worker threads (threads that process NFS requests)

\item  NFS\_Port:
		The port number for incoming NFS requests (default is 2049)

\item  MNT\_Port:
		The port number for mount protocol (default is any available port)

\item  NFS\_Program:
		The RPC program number for NFS (default is 100003)

\item  MNT\_Program:
		The RPC program number for mount protocol (default is 100005)

\item  Drop\_IO\_Errors:
		This parameter defines the behavior of the server when an EIO error
		is returned by the filesystem.\\
		TRUE indicates that the client request will be dropped, so the client
		will retry it later.\\
		FALSE indicates that an IO error is return to the client (and as a result,
		to the client application).

\item  Drop\_Inval\_Errors:
                This parameter defines the behavior of the server when an EINVAL error
                is returned by the filesystem.\\
                TRUE indicates that the client request will be dropped, so the client
                will retry it later.\\
                FALSE indicates that an IO error is return to the client (and as a result,
                to the client application).


\item  DupReq\_Expiration:
		This defines the lifetime for the duplicate NFS requests cache.

\item  Core\_Dump\_Size:
		This indicates the core size in case of a server crash (default is 0)

\item Nb\_Max\_Fd:
		This value will be used in a {\it setrlimit} call to set {\it RLIMIT\_NOFILE}. This is useful if you want the daemon to 
be abale to use more open file descriptor (useful for use with the FSAL\_PROXY). By default, the OS "max open files" (see {\it ulimit -n} will be used).

\item  Stats\_File\_Path:
		This indicates the file for dumping server's statistics.

\item  Stats\_Update\_Delay:
		This indicates the delay for dumping server's statistics.

\item Dump\_Stats\_Per\_Client:
        Should be set to TRUE or FALSE. If set TRUE, the statistics
        per client will be dumped . (default is FALSE)

\item Stats\_Per\_Client\_Directory:
               If Dump\_Stats\_Per\_Client is TRUE, this directory will contain
               one file per client. (default is /tmp)


\end{itemize}

    \subsection{The 'NFS\_DupReq\_Hash' block}


	Duplicate requests are found from their 'rpcxid' using a hash table.
	This section specifies the parameters for this hastable:

\begin{itemize}

\item  Index\_Size:
		The size of the hash table.
		This MUST be a prime number, greater enough compared to the number of worker threads
		(specified in the 'NFS\_Core\_Param' block)

\item  Alphabet\_Length:
		A parameter for the hashing algorithm.
		This must be set to the number of possible values
		for each byte of the rpcxid, i.e. 256.\\
		However, if you notice a bad balancing in your hash table,
		you can try decreasing this value (but it should not exceed 256).

\item  Prealloc\_Node\_Pool\_Size:
		For better performances, each slot of the hash table
		consists of a Red-Black Tree. Thus, this parameter is
		the number of preallocated RBT entries for each worker thread.
		As a result, this must be set to a value that is close to
		the forecasted number of duplicate requests cache entries,
		divided by the number of worker threads.
		Thus, it must be equal to 'NFS\_Worker\_Param::Nb\_DupReq\_Prealloc'.

\end{itemize}

    \subsection{The 'NFS\_IP\_Name' block}


	Hostnames are found from the associed IP address using a hash table.
	This section specifies the parameters for this hastable:

\begin{itemize}

\item  Index\_Size:
		The size of the hash table.
		This MUST be a prime number, greater enough compared to the number of worker threads
		(specified in the 'NFS\_Core\_Param' block)

\item  Alphabet\_Length:
		A parameter for the hashing algorithm.
		This must be set to the number of possible values
		for each byte of the ip address, i.e. 256.\\
		However, if you notice a bad balancing in your hash table,
		you can try decreasing this value (but it should not exceed 256).

\item  Prealloc\_Node\_Pool\_Size:
		For better performances, each slot of the hash table
		consists of a Red-Black Tree. Thus, this parameter is
		the number of preallocated RBT entries for each worker thread.
		As a result, this must be set to a value that is close to
		the number of client hosts, divided by the number of worker threads.

\item  Map:
		The hash table content can be preloaded when the NFS server is starting
		so that it won't have to issue any DNS requests at runtime.
		This parameter specifies the file that contains the DNS items to be
		preloaded.

\end{itemize}

    \subsection{The 'UidMapper\_Cache' block}


	User names are used for the NFSv4 protocol.
	They are found from the associated uid using a hash table.
	This section specifies the parameters for this hastable:

\begin{itemize}

\item  Index\_Size:
		The size of the hash table.
		This MUST be a prime number, greater enough compared to the number of worker threads
		(specified in the 'NFS\_Core\_Param' block)

\item  Alphabet\_Length:
		A parameter for the hashing algorithm.
		This must be set to the number of possible values
		for each byte of the uid, i.e. 256.\\
		However, if you notice a bad balancing in your hash table,
		you can try decreasing this value (but it should not exceed 256).

\item  Prealloc\_Node\_Pool\_Size:
		For better performances, each slot of the hash table
		consists of a Red-Black Tree. Thus, this parameter is
		the number of preallocated RBT entries for each worker thread.
		As a result, this must be set to a value that is close to
		the total number of users, divided by the number of worker threads.

\item  Map:
		The hash table content can be preloaded when the NFS server is starting
		so that it won't have to issue any ldap/nis request about users at runtime.
		This parameter specifies the file that contains the passwd items to be
		preloaded.
\end{itemize}


    \subsection{The 'GidMapper\_Cache' block}


	Group names are used for the NFSv4 protocol.
	They are found from the associated gid using a hash table.
	This section specifies the parameters for this hastable:

\begin{itemize}

\item  Index\_Size:
		The size of the hash table.
		This MUST be a prime number, greater enough compared to the number of worker threads
		(specified in the 'NFS\_Core\_Param' block)

\item  Alphabet\_Length:
		A parameter for the hashing algorithm.
		This must be set to the number of possible values
		for each byte of the gid, i.e. 256.\\
		However, if you notice a bad balancing in your hash table,
		you can try decreasing this value (but it should not exceed 256).

\item  Prealloc\_Node\_Pool\_Size:
		For better performances, each slot of the hash table
		consists of a Red-Black Tree. Thus, this parameter is
		the number of preallocated RBT entries for each worker thread.
		As a result, this must be set to a value that is close to
		the total number of groups, divided by the number of worker threads.

\item  Map:
		The hash table content can be preloaded when the NFS server is starting
		so that it won't have to issue any ldap/nis request about groups at runtime.
		This parameter specifies the file that contains the group items to be
		preloaded.

\end{itemize}

    \subsection{The 'NFSv4\_ClientId\_Cache' block}


	Client ids are used in NFSv4 protocol, in order to keep informations about clients.
	Those informations are stored into a hashtable.
	This section specifies the parameters for this hastable:

\begin{itemize}

\item  Index\_Size:
		The size of the hash table.
		This MUST be a prime number, greater enough compared to the number of worker threads
		(specified in the 'NFS\_Core\_Param' block)

\item  Alphabet\_Length:
		A parameter for the hashing algorithm.
		This must be set to the number of possible values
		for each byte of the client\_id, i.e. 256.\\
		However, if you notice a bad balancing in your hash table,
		you can try decreasing this value (but it should not exceed 256).

\item  Prealloc\_Node\_Pool\_Size:
		For better performances, each slot of the hash table
		consists of a Red-Black Tree. Thus, this parameter is
		the number of preallocated RBT entries for each worker thread.
		As a result, this must be set to a value that is close to
		the number of client hosts, divided by the number of worker threads.

\end{itemize}

    \subsection{The 'NFSv4\_StateId\_Cache' block}


	State ids are used in NFSv4 protocol, in order to keep informations about clients.
	Those informations are stored into a hashtable.
	This section specifies the parameters for this hastable:

\begin{itemize}

\item  Index\_Size:
		The size of the hash table.
		This MUST be a prime number, greater enough compared to the number of worker threads
		(specified in the 'NFS\_Core\_Param' block)

\item  Alphabet\_Length:
		A parameter for the hashing algorithm.
		This must be set to the number of possible values
		for each byte of the client\_id, i.e. 256.\\
		However, if you notice a bad balancing in your hash table,
		you can try decreasing this value (but it should not exceed 256).

\item  Prealloc\_Node\_Pool\_Size:
		For better performances, each slot of the hash table
		consists of a Red-Black Tree. Thus, this parameter is
		the number of preallocated RBT entries for each worker thread.
		As a result, this must be set to a value that is close to
		the number of client hosts, divided by the number of worker threads.

\end{itemize}

    \subsection{The 'NFS\_KRB5' block}

	This section specifies the parameters for RPCSEC\_GSS authentication.

\begin{itemize}

\item  PrincipalName:
		The principal name the NFS server (default is nfs@localhost.localdomain)

\item  KeytabPath:
		The Kerberos5 keytab for this principal

\end{itemize}

    \subsection{The 'NFSv4' block}

	This block is for NFSv4 specific parameters.

\begin{itemize}

\item  Lease\_Lifetime:
		This specifies the NFSv4 lease time (see RFC 3530 for more details)
\item  FH\_expire:
		This specifies if NFSv4 FH will expire (see RFC 3530 for more details)
\item Returns\_ERR\_FH\_EXPIRED:
                Specifies if the serveur should return NFS4ERR\_FHEXPIRED. This
                will be used only if FH\_expire is TRUE.
\item Use\_OPEN\_CONFIRM:
		This specifies if the server should request the client (via OP4\_OPEN\_CONFIRM) to 
                confirm for the files it opens. Default Value is FALSE
\end{itemize}


\end{document}

