%% LyX 1.3 created this file.  For more info, see http://www.lyx.org/.
%% Do not edit unless you really know what you are doing.
\documentclass[english]{article}
\usepackage[T1]{fontenc}
\usepackage[latin1]{inputenc}

\makeatletter
%%%%%%%%%%%%%%%%%%%%%%%%%%%%%% User specified LaTeX commands.

\usepackage{babel}
\makeatother
\begin{document}

{\center{ \Large \textbf{Exporting FUSE-based filesystems} }}
{\center{ \Large \textbf{with GANESHA} }}


\section{Quick Start}

\subsection{Compiling and installing}

Go to the "src" directory of NFS-GANESHA distribution.
Then run:
\begin{verbatim}
 ./configure --with-fsal=FUSE
 make
 make install
\end{verbatim}

\subsection{Exporting your FUSE-based file system}

\begin{itemize}
\item In your source code, replace \texttt{\#include <fuse.h>} with \texttt{\#include <ganesha\_fuse\_wrap.h>}
\item For linking your program, replace \texttt{-lfuse} with \texttt{-lganeshaNFS}
\end{itemize}

That's done! You can now start your daemon and mount your filesystem (here is an example in NFSv3):
\begin{verbatim}
 ./my_daemon
 mount -o vers=3,udp localhost:/ /mnt
\end{verbatim}

\section{Details about GANESHA FUSE-like binding}

\subsection{Interface}

Ganesha FUSE-like interface provides most FUSE's "high-level" structures and calls (struct fuse\_file\_info, struct fuse\_operations, fuse\_main(), fuse\_get\_context(), ...)

Basically, it supports FUSE-based filesystems using \texttt{<fuse.h>} (at least version 2.6).

\subsection{Filesystem mandatory features}

For being able to export your filesystem with NFS-GANESHA, the following features are mandatory:
\begin{itemize}
\item \textbf{getattr} must be implemented
\item Each entry in your filesystem must have a unic <st\_ino, st\_dev> peer
\item You must set a correct value to the "st\_mode" field of "struct stat" (type and access mode)
\item The "st\_nlink" field of "struct stat" must not be null
\item Deprecated call "getdir" is not supported, replace it with "readdir"
\end{itemize}

\subsection{Command line arguments}

fuse\_main() parameters slightly differ from FUSE implementation.

The expected command line parameters are:
\begin{verbatim}
 Usage: fusexmp [-hds][-L <logfile>][-N <dbg_lvl>][-f <config_file>]
        [-h]                display this help
        [-s]                single-threaded (for MT-unsafe filesystems)
        [-L <logfile>]      set the logfile for the daemon
        [-N <dbg_lvl>]      set the verbosity level
        [-f <config_file>]  set the config file to be used
        [-d]                the daemon starts in background,
                            in a new process group
        [-R]                daemon will manage RPCSEC_GSS
                            (default is no RPCSEC_GSS)
        [-F] <nb_flushers>  flushes the data cache with purge,
                            but do not answer to requests
        [-S] <nb_flushers>  flushes the data cache without purge,
                            but do not answer to requests
\end{verbatim}

\subsection{Example}

FUSE-binding examples are provided in the GANESHA repository (directory src/example-fuse).

These are the same as provided with FUSE distributions, except that \texttt{\#include <fuse.h>} have been changed to \texttt{\#include <ganesha\_fuse\_wrap.h>}.

After compiling GANESHA with FUSE FSAL, you can simply run it the following way:
\begin{verbatim}
 ./fusexmp
\end{verbatim}


\end{document}
