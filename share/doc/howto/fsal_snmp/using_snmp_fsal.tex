%% LyX 1.3 created this file.  For more info, see http://www.lyx.org/.
%% Do not edit unless you really know what you are doing.
\documentclass[english]{article}
\usepackage[T1]{fontenc}
\usepackage[latin1]{inputenc}

\makeatletter
%%%%%%%%%%%%%%%%%%%%%%%%%%%%%% User specified LaTeX commands.

\usepackage{babel}
\makeatother
\begin{document}

{\Large \textbf{Exporting a SNMP tree with GANESHA} }\\

\section{Overview}

Thanks to its backend modules called {}``File System Abstraction Layers'' (FSAL),
GANESHA NFS server makes it possible to export any sets of data
organized as trees, where each entry has a name and path.

In SNMP (Simple Network Management Protocol), data are organized as a tree
where all objects are addressed by their OID, which is the object's path
in this tree. For example, {}``.iso'' can be considered has a directory
identified by the OID {}``.1'', and the leaf {}``.iso.org.dod.internet.mgmt.mib-2.system.sysDescr.0''
(which corresponds to system description) can be considered as a file
whose data is something like {}``Linux node\_name 2.6.9-22.ELsmp \#1 SMP Mon Sep 19 18:32:14 EDT 2005 i686''.

As a result, exporting such a tree through NFS makes it possible
to have a filesystem similar to {}``/proc'', where administrators can
read statistics about a system or an equipment (switch, router, ...)
simply using \texttt{cat} command, and modify them easily, using \texttt{vi} or \texttt{echo "xxx" > file}.

This is what you can do using the SNMP FSAL. To use it, simply build GANESHA using the configure arg \texttt{--with-fsal=SNMP}:
\begin{verbatim}
cd src
./configure --with-fsal=SNMP
make
\end{verbatim}

Net-snmp library and includes must be installed on your system.

\section{SNMP relative options}

\subsection{The SNMP block}

\subsubsection{Parameters description}

For configurating GANESHA's SNMP access, you have to set
some options in the configuration file: this in done in a {}``\texttt{SNMP}''
configuration block.
\\

In such a block, you can set the following values:

\begin{itemize}
\item snmp\_version: this indicates the SNMP protocol version that GANESHA
      will use for communicating with SNMP agent. Expected values are
      \texttt{1}, \texttt{2c} or \texttt{3}.
      Default is \texttt{"2c"}.
\item snmp\_server: this is the address of the SNMP master agent. A port number
      can also be specified by adding {}``:<port>'' after the address.
      Default is \texttt{"localhost:161"}.
\item nb\_retries: number of retries for SNMP requests.
      Default value is\\
      \texttt{SNMP\_DEFAULT\_RETRIES}, defined in net-snmp library.
\item microsec\_timeout: number of microseconds until first timeout,
      then an exponential backoff algorithm is used for next timeouts.
      Default value is \texttt{SNMP\_DEFAULT\_TIMEOUT}, defined in net-snmp library.
\item client\_name: this is the client name that could be used internally
      by net-snmp for its traces.
      Default value is \texttt{"GANESHA"}.
\item snmp\_getbulk\_count: this indicates the number of responses wanted
      for each SNMP GETBULK request.
      Default value is \texttt{64}.
\end{itemize}

SNMP v1 and v2 specific parameters:
\begin{itemize}
\item community: this is the SNMP community used for authentication.
      Default is \texttt{"public"}.
\end{itemize}

SNMP v3 specific parameters:
\begin{itemize}
\item auth\_proto: the authentication protocol (MD5 or SHA).
      Default is \texttt{"MD5"}.
\item enc\_proto: the privacy protocol (DES or AES).
      Default is \texttt{"DES"}.
\item username: the security name (or user name).
      This a private information: check rights on this config file!
\item auth\_phrase: authentifaction passphrase (>=8 char).
      This a private information: check rights on this config file!
\item enc\_phrase: authentifaction passphrase (>=8 char).
      This a private information: check rights on this config file!
\end{itemize}

\subsubsection{A simple SNMP v2c example}
\begin{verbatim}
SNMP
{
    snmp_version = 2c;
    snmp_server = "snmp_master.my_net";
    community = "public";
}
\end{verbatim}

\subsubsection{A simple SNMP v3 example}
\begin{verbatim}
SNMP
{
    snmp_version = 3;
    snmp_server  = "snmp_master.my_net";
    username     = "snmpadm";
    auth_phrase  = "p4ssw0rd!";
    enc_phrase   = "p455w0rd?";
}
\end{verbatim}


\subsection{Export entries}

For defining the path of an export entry, you must replace traditional
SNMP dot separators `.' by slashes `/'. For example, you should
set export path to `/iso/org' instead of `.iso.org'.

Note that you can give slash separated numerical OIDs for exports. Thus,
exporting `/1/3/6/1/2/1' is equivalent to `/iso/org/dod/internet/mgmt/mib-2'.


\section{Specific behaviors}

Even if it is possible to export a SNMP tree as if it was a standard
filesystem, SNMP however has some specificities that don't match
POSIX or NFS semantics and features.

\subsection{Creating and removing objects}

The goal of exporting SNMP with NFS is to make it possible
for an administrator to browse statistics, to read them,
and modify some agent's configuration variables.

Thus, we don't have any interest in creating new entries
(register some new values to SNMP agent)
nor removing existing entries (unregister agent's objects).
That would be very complex to operate and it could result
in agent's problems or crashes...

Another problem in SNMP is that object types are very
different from NFS types: an object can be an integer,
a string, a counter, a timetick, so it is difficult
to know the type of data a NFS client is going to write
to a newly created file...

What's more, directories don't have a proper existence;
they only exist if a child of them exist. As a result,
creating or deleting an empty directory would
have no sense in SNMP.

For all those reasons, \textbf{create and remove operations}
(create, mkdir, remove, rmdir) \textbf{are not supported}
and return an \texttt{EROFS} error.

\subsection{Objects' attributes}


\subsubsection{Returned attributes}

SNMP objects don't have the same metadata as NFS.
They have no modification or access time,
no owner, no inode number, etc... As a result,
it has been necessary to emulate some
of these attributes, to make the SNMP FSAL compatible
with the NFS protocol.

\begin{itemize}

\item type: all SNMP objects are considered as regular files
            and their parent nodes as directories.

\item size: unlike `\texttt{/proc}' file system (where are file sizes
            are shown as null), the returned size for a SNMP object is
            the size of its current string representation.

\item inode number: SNMP objects have no unic 32 or 64 bits identifier.
            However, their OID is unic and constant, like inode number might be.
            So, the returned inode number is a hash of this OID.

\item owner and group: there is no such attributes in SNMP. Thus,
            owners and groups are set to 0 (root).
            
\item mode: it is not always possible to determinate the access rights
            for an SNMP object. Thus, the SNMP FSAL makes its best
            to determine access rights from MIBs info, but
            sometimes it doesn't know, so it returns a default mode
            \texttt{rw-rw-rw} (0666) for files, and \texttt{r-xr-xr-x} (0555)
            for directories. As a result, you may get an \texttt{EACCES} error
            when writing or reading a file, even if you seemed to have the
            correct rights.

\item numlinks: there is no hardlinks in SNMP, so link count is always 1 for
            regular files. What's more, for a given directory, il would be
            costful to count the number of child directories that would have
            a virtual `..' entry. As a result, the numlink is also set to 1
            for directories.

\item access, modification and change time: SNMP has no memory of the time
            when a variable has been created, read or modified.
            So, the SNMP FSAL always return current time for those attributes.
            What's more, given that objects' data can change continuously,
            this behavior forces clients to clear their datacaches.

\end{itemize}

\subsubsection{Setting attributes}

Given that the only attributes that have an equivalence in SNMP
are static (read-only), \textbf{no attributes are supported for setattr operations}.
Thus, trying to change an attribute will result in a \texttt{EINVAL} error.


\subsection{Objects' data}

\subsubsection{String representation}

SNMP objects divide in many types: integer, counter, gauge,
string, opaque buffer, OID, timeticks, network address, etc.
In order to make it easy to interpret, the data returned
by a `read' operation is a string representation that depends
on the object type.

Indeed, even if integer, timeticks or IPv4 address are both
32 bits values, the integer will be displayed as a numerical
value, timeticks will be displayed as days/hours/minutes/seconds,
and the network address will be displayed in the classical dot-separated
notation.
\\

Some examples:
\begin{verbatim}
$ cd /mnt/iso/org/dod/internet/mgmt/mib-2

$ cat host/hrSystem/hrSystemUptime/0
528224338 (61 days, 03:17:23.38)

$ cat udp/udpInDatagrams/0
820798

$ cat udp/udpTable/udpEntry/udpLocalAddress/1/0/0/127/123
127.0.0.1
\end{verbatim}


\subsubsection{Modifying data}

To modify a value, you just have to write a string repesentation to the associated file,
using the way you want: a program, a script, or just the \texttt{echo} command, ...

For example:
\begin{verbatim}
$ cd /mnt/iso/org/dod/internet/mgmt/mib-2
$ echo "my_new_hostname" > system/sysName/0
\end{verbatim}

But be careful: in SNMP, there are strong type constraints: you cannot
write a string to an integer, etc. What's more, for a given object, the agent
may also do extra checks (for example, if a configuration value
is supposed to be a percentage, its SNMP type is unsigned integer,
but the agent may return an error if you set a value greater than 100).

In general, you have to rewrite data in the format of the value read in the file.
The only exception is for timeticks: you must only write the value in 100th of seconds
(not its translation to days, hours, minutes, seconds).

An issue when developping the SNMP FSAL has been to translate {}``invalid type''
SNMP errors to an appropriate NFS error code. Indeed, in NFS, there is
no concept of invalid data content ! Returning NFSERR\_INVAL or NFSERR\_IO
may result in a bad interpretation by the client, and it would be difficult
for the client to distinguish the error from a real EIO or EINVAL error.
So, it has been decided to convert such an error to NFSERR\_DQUOT (that has no other sense with SNMP)
so the user can know, with no ambiguity, that the value is not in the correct format,
or is out of expected range.


\subsubsection{Unsupported data operations}

\begin{paragraph}{Read at offset>0}
In the SNMP protocol, an object's data is read in a single GET request.
As a result, a `read' NFS operation must always read data from offset 0,
and with a buffer big enough for writting the whole variable content.
If the buffer is too small, the data representation will be truncated.

Thus, after a seek, a \texttt{read} operation may return an EINVAL error.
The same error is returned for \texttt{pread} operation with a non-null offset.
\end{paragraph}

\begin{paragraph}{Write at offset>0}
To write in a file, it is the same: in the SNMP protocol,
object's data is written in a single SET request. As a result,
the NFS `write' operation must give all the object data, starting at offset 0.

Thus, after a \texttt{seek} operation, a \texttt{write} operation may return EINVAL.
The same error is returned for a \texttt{pwrite} operation with a non-null offset.
\end{paragraph}

\begin{paragraph}{Truncate}
A truncate operation on an object that is not a string has no sense.
For example, what about truncating an integer? Indeed, its string
representation always contains at least 3 characters, e.g. for representing
zero: "0$\backslash{}$n$\backslash{}$0".

However, we must not return an error for truncate operations,
because when user do something like `\texttt{echo "0" > an\_integer\_object}',
the first system operation is:\\
\texttt{open( "an\_integer\_object", O\_CREAT|O\_TRUNC|O\_WRONLY );}\\
which will be translated to the NFS request \texttt{SETATTR(size=0)} (i.e. truncate).
As a result, if we returned an error for the truncate operation,
the user could not write data to files...

Thus, we decided to ignore `truncate' operations, given that the object's
data is totaly replaced at the following `write'.

The only border side effect is that '\texttt{cp /dev/null}' or `\texttt{truncate}'
on a file has no effect (to truncate a string, you must write an empty string to it).

\end{paragraph}



\end{document}
