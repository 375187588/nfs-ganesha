%% LyX 1.3 created this file.  For more info, see http://www.lyx.org/.
%% Do not edit unless you really know what you are doing.
\documentclass[english]{article}
\usepackage[T1]{fontenc}
\usepackage[latin1]{inputenc}
\usepackage{graphicx}

\makeatletter
%%%%%%%%%%%%%%%%%%%%%%%%%%%%%% User specified LaTeX commands.

\usepackage{babel}
\makeatother
\begin{document}

\author{jbongio@us.ibm.com}
\title{\center{ \Large \textbf{Overview of GANESHA's Logging Mechanisms} }}
\maketitle

GANESHA is capable of fine granularity filtering of log messages, support for
multiple log destinations (files, console, syslog, etc.), support for sending
different log messages to different log destinations, and many ways of setting
log parameters.

These options allow GANESHA the flexibility to be used in multiple
professional environments while the default settings as well as the 

\section{Overview of Logging}

Each log message in GANESHA relates to a specific component and severity
level. A component is a label created with the intent of capturing a sectin of
code or a specific function in GANESHA. Severity level determines the importance
of a log message. The verbosity of logging can be changed per component as well
as by the severity level of messages. Log messages can be sent to either the
console, syslog, or an alternative file. The default logging destination is
syslog and the default log level for all components is NIV\_EVENT.

A typical log message in syslog will look like the following:
\begin{verbatim}
Sep  8 02:31:05 localhost nfs-ganesha[11555]: [main] :NFS STARTUP: EVENT: \
Configuration file successfully parsed
\end{verbatim}

The first bit with the date, hostname, process name, and process id are all
prepended to the log message by syslog. \emph{[main]} is the thread that printed
the log message. \emph{NFS STARTUP} is the component. \emph{EVENT} is the
severity of the message. Finally \emph{Configuration file successfully parsed}
is the message itself. Each log message in GANESHA follows this format.

\subsection{Logging Components}
\label{listofcomponents}
Listed below are all of the components that a log message can belong to

\begin{itemize}
\item COMPONENT\_ALL - 
\item COMPONENT\_LOG - Reports changes to log destination or log level.
\item COMPONENT\_MEMALLOC - Used for tracking memory allocation through the
  buddy malloc system.
\item COMPONENT\_STATES - Mainly used with severity level NIV\_FULL\_DEBUG to
  debug NFSV4 states.
\item COMPONENT\_MEMLEAKS - Mainly used with severity level
  NIV\_DEBUG and NIV\_FULL\_DEBUG. It is used to detect and debug memory leaks.
\item COMPONENT\_FSAL - Messages that relate to FSAL specific functions.
\item COMPONENT\_NFSPROTO - Messages that relate to the NFSV2 and NFSV3
  protocol.
\item COMPONENT\_NFSV4 - Messages relating to the NFSV4 protocol.
\item COMPONENT\_NFSV4\_PSEUDO - Message relating to the NFSV4 pseudo file
  system.
\item COMPONENT\_FILEHANDLE - Messages relating to the conversion from a file
  handle (used by clients) to an FSAL specific handle (used by the server).
\item COMPONENT\_NFS\_SHELL - Messages relating to the GANESHA shell.
\item COMPONENT\_DISPATCH - Messages relating to the dispatch thread which is
  responsible for receiving requests from clients and assigning to a worker
  thread for processing.
\item COMPONENT\_CACHE\_CONTENT - Messages related to cached file data that
  makes client requests faster.
\item COMPONENT\_CACHE\_INODE - 
\item COMPONENT\_CACHE\_INODE\_GC - 
\item COMPONENT\_HASHTABLE - 
\item COMPONENT\_LRU - 
\item COMPONENT\_DUPREQ - GANESHA (and other NFS servers) keep a cache of recent
  requests and their replies. If a duplicate request is received GANESHA will
  send the same reply rather than process the request again. The messages
  relating to this duplicate request cache fall under this component.
\item COMPONENT\_RPCSEC\_GSS - 
\item COMPONENT\_INIT - Most messages logged when GANESHA is starting will be
  logged under this component.
\item COMPONENT\_MAIN - 
\item COMPONENT\_IDMAPPER - 
\item COMPONENT\_NFS\_READDIR - 
\item COMPONENT\_NFSV4\_LOCK - Debugs file locking for NFSV4.
\item COMPONENT\_NFSV4\_XATTR - Debugs extended attributes for NFSV4.
\item COMPONENT\_NFSV4\_REFERRAL - Debugs NFSV4's ability to refer clients to
  another NFSV4 server.
\item COMPONENT\_MEMCORRUPT - 
\item COMPONENT\_CONFIG - 
\item COMPONENT\_CLIENT\_ID\_COMPUTE - 
\item COMPONENT\_STDOUT - 
\item COMPONENT\_OPEN\_OWNER\_HASH - 
\item COMPONENT\_SESSIONS - 
\item COMPONENT\_PNFS - Message related to the ability for NFS to work in
  parallel to other NFS servers. Parallel NFS is a compile time option and won't
  be seen unless it was compiled.
\item COMPONENT\_RPC\_CACHE - 
\end{itemize}$\\$

\subsection{Log Severity Levels}

The log levels (debug levels, severity levels) and their order of severity are
as follows:
\begin{itemize}
\item NIV\_NULL - no messages are printed.
\item NIV\_MAJ - only major messages are printed.
\item NIV\_CRIT - critical messages or higher are printed.
\item NIV\_EVENT - event messages or higher printed. 
\item NIV\_DEBUG - debug messages or higher are printed. This should only be
  used if a user suspects they have found a bug or are developing GANESHA.
\item NIV\_FULL\_DEBUG - extremely verbose debug messages or higher are printed.
\end{itemize}

The default log level is NIV\_EVENT. 

\subsection{Log Location}

The location where log messages will be printed is configurable when GANESHA
first starts. Once GANESHA is running, the log destination cannot be changed.

Section \ref{loggingmethods} covers the different mechanisms for configuring the
log destination upon startup. Each method accepts a string as the log
destination. The valid strings are as follows:
\begin{itemize}
\item ``SYSLOG'' - Log messages will go to syslog. Where the log messages go
  after being sent to the syslog daemon depends on how syslog was configured,
  but the messages will probably go to ``/var/log/messages''. Syslog is the
  default log location.
\item ``STDOUT'' - Log messages will be sent to the console that started GANESHA
  as stdout output. This can sometimes be useful for debugging purposes but is
  not generally used.
\item ``STDERR'' - Log messages will be sent to the console that started GANESHA
  as stderr output. This can sometimes be useful for debugging purposes but is
  not generally used.
\item ``/some/path/to/file'' - If none of the above strings are given, GANESHA
  will assume the string refers to a file path. Log messages will be appended to
  the filepath. All of the directories in this path must already
  exist. Otherwise an error will be thrown and GANESHA will quit.
\end{itemize}

%% To clear the current messages in syslog execute the following command:
%% \begin{verbatim}
%% $ > /var/log/messages
%% \end{verbatim}

\subsection{Changing the Log Destination}

There are two ways to change where log messages will go, through the commandline
or through configuration file variables. It is recommended people use the
default of syslog, but there are may be some scenarios where separating the
logging of certain components will be useful.

\subsection{Commandline Arguments}

\begin{verbatim}
/usr/bin/gpfs.ganesha.nfsd -d -f /etc/ganesha/gpfs.ganesha.nfsd.conf -N \
NIV_EVENT -L SYSLOG
\end{verbatim}

\subsection{Configuration File Variables}

Table \ref{confifile} lists the configuration file stanzas that contain a
variable for the location of its log files. Setting the variable for that
stanza, displayed in the second column of the table, will set the log
destination for a specific logging component, displayed in the third column.

\begin{figure}[h]
\center
  \begin{tabular}{ | l | l | l |}
    \hline Configuration file stanza & configuration variable & logging component\\
    \hline FSAL & LogFile & COMPONENT\_FSAL \\
    \hline CacheInode\_Client & LogFile & COMPONENT\_CACHE\_INODE\\
    \hline FileContent\_Client & LogFile & COMPONENT\_CACHE\_CONTENT\\
    \hline BUDDY\_MALLOC & LogFile & COMPONENT\_MEMALLOC \\ && COMPONENT\_MEMLEAKS\\
    \hline
  \end{tabular}
  \caption{Configuration File Variables for Changing Log Destination}
  \label{confifile}
\end{figure}
Each stanza uses the variable name of ``LogFile''. So an
example of setting the FSAL stanza to using stdout output is:
\begin{verbatim}
FSAL
{
    LogFile="STDOUT";
}
\end{verbatim}

\section{Methods for Controlling the Log Level of Components}
\label{loggingmethods}

There are a total of six different ways that the log level of components is
set. The order of priority for the six is shown in Figure
\ref{priorities}. First there is a default log level for all components. Second,
environment variables can be defined 

\begin{figure}[h]
  \center
  $\textnormal{compiled defaults} \rightarrow \textnormal{environment variables}
  \rightarrow \textnormal{command-line arguments} \rightarrow
  \textnormal{configuration file parameters} \rightarrow
  \textnormal{signal handlers} \leftrightarrow \textnormal{snmp}$
  \caption{Order of Priority of Mechanisms for Changing Log Levels}
  \label{priorities}
\end{figure}


\subsection{Default Log Levels}

The default log level of each component is NIV\_EVENT. If no other mechanism
is used, the log level will remain at NIV\_EVENT.

\subsection{Environment Variables}
The name of each component, listed in Subsection \ref{listofcomponents}, can be
used as the name of an environment variable to set the debug level of that
component. 

This option is not recommended for use in a GANESHA deployment, but meant for
fine granularity tuning of debug messages for development purposes.

\subsection{Commandline Arguments}

\begin{verbatim}
/usr/bin/gpfs.ganesha.nfsd -d -f /etc/ganesha/gpfs.ganesha.nfsd.conf -N \
NIV_EVENT -L SYSLOG
\end{verbatim}

\subsection{Configuration File Variables}

\begin{figure}[h]
\center
  \begin{tabular}{ | l | l | l |}
    \hline Configuration file stanza & configuration variable & logging component\\
    \hline FSAL & DebugLevel & COMPONENT\_FSAL \\
    \hline CacheInode\_Client & DebugLevel & COMPONENT\_CACHE\_INODE\\
    \hline FileContent\_Client & DebugLevel & COMPONENT\_CACHE\_CONTENT\\
    \hline
  \end{tabular}
  \caption{Configuration File Variables for Changing Log Levels}
  \label{confifilelevel}
\end{figure}
Each stanza uses the variable name of ``DebugLevel''. An
example of setting the FSAL stanza to using NIV\_DEBUG is:
\begin{verbatim}
FSAL
{
    DebugLevel="NIV_DEBUG";
}
\end{verbatim}

\subsection{Signal Handlers}

To increment log level of all components:
\begin{verbatim}
$ killall -s SIGUSR1 /usr/bin/gpfs.ganesha.nfsd
$ tail -1 /var/log/messages
Sep  8 02:28:47 localhost nfs-ganesha[11287]: [main] :LOG: SIGUSR1 Increasing \
log level for all components to NIV_DEBUG
\end{verbatim}
To decrement log level of all components:
\begin{verbatim}
$ killall -s SIGUSR2 /usr/bin/gpfs.ganesha.nfsd
$ tail -1 /var/log/messages
Sep  8 02:29:00 localhost nfs-ganesha[11287]: [main] :LOG: SIGUSR2 Decreasing \
log level for all components to NIV_EVENT
\end{verbatim}

\subsection{SNMP}

It is necessary to setup an SNMP server in order to use the SNMP support in
GANESHA. For instructions on how to install and configure this server as well as
how to access GANESHA through SNMP, refer to the GANESHA SNMP ADMINISTRATION
guide. Once SNMP is properly installed, the tool presented in Section
\ref{snmptool} can be used for easy local and remote administration.

\section{GANESHA Tool for Reading/Modifying Log Levels with SNMP}
\label{snmptool}

GANESHA contains a program written in Perl that provides easy access to log
level information. From the root directory of the GANESHA code base, the path to
the program is \emph{src/cmdline\_tools/ganesha\_log\_level} and, if installed
through the rpm, the program is installed to
\emph{/usr/bin/ganesha\_log\_level}.

\emph{ganesha\_log\_level} requires a configuration file to locate and access
the SNMP server. A sample configuration file is installed to
\emph{/etc/ganesha/snmp.conf}. The file is very simple:
\begin{verbatim}
# This is a sample snmp.conf to be used with ganesha to allow
# the ganesha_log_level tool to work. Put it in /etc/ganesha

community_string = public
\end{verbatim}
It is essential that the community\_string parameter matches the community
column in \emph{/etc/snmp/snmpd.conf} that looks like the following:
\begin{verbatim}
##       sec.name  source          community
com2sec  local     localhost       ganesha
com2sec  mynetwork 9.47.69.0/24    ganesha
\end{verbatim}

\subsection{Examining Log Levels with \emph{ganesha\_log\_level}}

List possible log levels:
\begin{verbatim}
$ ganesha_log_level -L
Valid Log Levels (less to more messages):
  NIV_NULL
  NIV_MAJOR
  NIV_CRIT
  NIV_EVENT
  NIV_DEBUG
  NIV_FULL_DEBUG
\end{verbatim}
List current log levels on components:
\begin{verbatim}
$ ganesha_log_level -l
Current Log Levels:
  COMPONENT_ALL                   NIV_EVENT
  COMPONENT_CACHE_CONTENT         NIV_EVENT
  COMPONENT_CACHE_INODE           NIV_EVENT
  COMPONENT_CACHE_INODE_GC        NIV_EVENT
  COMPONENT_CLIENT_ID_COMPUTE     NIV_EVENT
  COMPONENT_CONFIG                NIV_EVENT
  COMPONENT_DISPATCH              NIV_EVENT
<output continues ...>
\end{verbatim}
List possible components:
\begin{verbatim}
$ ganesha_log_level -C
Valid Components:
  COMPONENT_ALL                 
  COMPONENT_CACHE_CONTENT       
  COMPONENT_CACHE_INODE         
  COMPONENT_CACHE_INODE_GC      
  COMPONENT_CLIENT_ID_COMPUTE   
  COMPONENT_CONFIG              
  COMPONENT_DISPATCH  
<output continues ...>
\end{verbatim}
Get the log level for one or more components
\begin{verbatim}
$ ganesha_log_level -g COMPONENT_LRU
NIV_EVENT
\end{verbatim}

\subsection{Changing Log Levels with \emph{ganesha\_log\_level}}

Set the log level for one or more components:
\begin{verbatim}
$ ganesha_log_level -s NIV_DEBUG COMPONENT_LRU
$ ganesha_log_level -g COMPONENT_LRU
NIV_DEBUG
\end{verbatim}

\end{document}


