
This example is based on Keith Packard's 2003 paper for
the Linux Symposium Proceedings.

The easiest way to do a bibliography is to use BiBTeX.
In the body of the paper, you \cite{} various references.
The citation name is the first name following the opening
curly brace in the .bib file.  For example, with the list below,
I could \cite{autoconf} and \cite{freetype2}.

Near the end of your main .tex file, you include a section like so:
\begin{flushleft}
\bibliography{keithp}
\bibliographystyle{plain}
\end{flushleft}
(this comes *before* \end{document}.)

And in a separate file whose name matches the \bibliography{}
declaration above (e.g., keithp.bib in this case), you define all
the references.  Note that \url is a valid way to typeset web
references.  

Note that the makefiles are already set up to process this form
of bibliography, so using it is indeed easy.  (It's also one
reason why the input files are processed multiple times, though.)

Here are some sample entries for various types
of publications:

@book{autoconf,
 title		= "GNU Autoconf, Automake and Libtool",
 author		= "Gary V. Vaughan and Ben Elliston and Tom Tromey and Ian Lance Taylor",
 publisher 	= "New Riders",
 year		= 2000,
 note		= {ISBN 1-57870-190-2},	},

@article{blinn:1994,
 title		= "Compositing Theory",
 author		= "Jim Blinn",
 journal	= "IEEE Computer Graphics and Applications",
 year		= 1994,
 month		= "September",
 note		= "Republished in~\cite{blinn:1998}"	}
 
@book{blinn:1998,
 title		= "{Jim Blinn's Corner: Dirty Pixels}",
 author		= "Jim Blinn",
 year		= 1998,
 publisher	= "Morgan Kaufmann",
 isbn		= "1-55860-455-3",	}

@techreport{dbe,
 title		= "{Double Buffer Extension Protocol}",
 author		= "Ian Elliott and David P. Wiggins",
 institution	= "X Consortium, Inc.",
 type		= "X Consortium Standard",
 year		= 1994,			}
@manual{dc,
 title		= "DC - An Interactive Desk Calculator",
 author		= "Robert Morris and Lorinda Cherry",
 organization	= "AT\&T Bell Laboratories",
 note		= "Unix Programmer's Manual Volume 2, 7th Edition",
 year		= 1978,			},

@misc{freetype2,
 title		= "The design of {FreeType} 2",
 author		= "David Turner and The FreeType Development Team",
 year		= 2000,
 note		= "\url{http://www.freetype.org/freetype2/docs/design/}",
},

@inproceedings{gj,
 title		= "Making the future safe for the past: Adding Genericity to the Java Programming Language",
 author		= "Gilad Bracha and Martin Odersky and David Stoutamire and Phillip Wadler",
 month		= "October",
 booktitle	= "Conference on Object-Oriented Programing systems, Languages and Applications (OOPSLA '98)",
 year		= 1998,
 publisher	= "ACM",
 organization	= "SIGPLAN",	}

@phdthesis{Hobby85,
 author = {John D. Hobby},
 title = {Digitized Brush Trajectories},
 school = {Stanford University},
 year = {1985},
 note = {Also {\it Stanford Report STAN-CS-85-1070}}
}

@article{itsy,
 title          = "{Itsy: Stretching the Bounds of Mobile Computing}",
 author         = "William R. Hamburgen and Deborah A. Wallach and Marc A. Viredaz and Lawrence S. Brakmo and Carl A. Waldspurger and Joel F. Bartlett and Timothy Mann and Keith I. Farkas",
 journal        = "IEEE Computer",
 year           = 2001,
 publisher      = "Institute of Electrical and Electronics Engineers, Inc.",
 volume         = 34,
 number         = 4,
 month          = "April",
 pages          = "28-35",              }
  
@inproceedings{lbx:1993,
 title = "{An Update on Low Bandwidth X (LBX): A Standard For X and Serial Lines}",
 author		= "Jim Fulton and Chris Kent Kantarjiev",
 booktitle	= "Proceedings of the Seventh Annual X Technical Conference",
 month		= "January",
 year		= 1993,
 pages		= "251-266",
 address	= "Boston, MA",
 organization	= "MIT X Consortium",	
},

@inproceedings{lmbench:1996,
 title 		= "{lmbench: Portable tools for performance analysis}",
 author		= "Larry McVoy and Carl Staelin",
 booktitle	= "Technical Conference Proceedings",
 month		= "January",
 year		= 1996,
 pages		= "279-284",
 address	= "San Diego, CA",
 organization	= "USENIX",		}
 
@Article{Nistnet00,
  author =       "NIST Internetworking Technology Group",
  title =        "{NISTNet} network emulation package",
  journal =      "\url{http://www.antd.nist.gov/itg/nistnet/}",
  month =        jun,
  year =         "2000",
  bibdate =      "Thursday, June 29, 2000 at 16:40:15 (MEST)",
  submitter =    "Katarina Asplund",
}

@TechReport{AMD:2000:XTW,
  author =       "{AMD Corporation}",
  title =        "{x86-64$^{\mathrm{TM}}$ Technology White Paper}",
  institution =  "{AMD Corporation}",
  address =      "One AMD Place, Sunnyvale, CA 94088, USA",
  pages =        "12",
  day =          "17",
  month =        aug,
  year =         "2000",
  bibdate =      "Fri May 04 12:53:45 2001",
  bibsource =    "\url{http://www.amd.com/products/cpg/64bit/index.html}",
  URL = "\url{http://www.amd.com/products/cpg/64bit/pdf/x86-64_wp.pdf};
                 \url{http://www1.amd.com/products/cpg/x8664bit/faq}",
  acknowledgement = ack-nhfb,
  annote =       "The x86-64 architecture is definitely not an IA-64
                 implementation, but rather, an extension of IA-32 by
                 widening the integer registers to 64-bits.",
}

@unpublished{pinzari,
 author		= "Gian Filippo Pinzari",
 title		= "The NX X Protocol Compressor",
 note		= "Electronic Communication",
 month		= "March",
 year		= "2003",
 }
 
@inproceedings{Gettys:2002,
  title = "{The Future is Coming, Where the X Window System Should Go}",
  author        = "James Gettys",
  booktitle	= "FREENIX Track, 2002 Usenix Annual Technical Conference",
  month		= "June",
  year		= 2002,
  organization	= "USENIX",
  address       = "Monterey, CA",
  url = "\url{http://www.usenix.org/publications/library/proceedings/usenix02/tech/freenix/full_papers/gettys/gettys_html/index.html}",
}

@misc{ewing,
 title          = "Linux 2.0 Penguins",
 author         = "Larry Ewing",
 note           = "\url{http://www.isc.tamu.edu/~lewing/linux}",
}

@misc{gimp,
 title          = "The {GIMP}: The {GNU} Image Manipulation Program",
 author         = "Peter Mattis and Spencer Kimball and the GIMP developers",
 note           = "\url{http://www.gimp.org}",
}

