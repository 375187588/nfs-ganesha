\documentclass[twocolumn,12pt]{article}
\usepackage{ols}
\ifpdf
\usepackage[pdftex]{epsfig}
\else
\usepackage{epsfig}
\fi
\input{ols-fonts}

% These packages are Proceedings-friendly.
\usepackage{cprog}
\usepackage[nolineno,norules]{lgrind}
\usepackage[hang,scriptsize]{subfigure}

% These ones are only suitable for standalone
\usepackage{subfigure}
%%% both of these break the Proceedings and are thus evil
\usepackage{listings}
\input{llvm.lst}    % Get listing support for llvm code
%%%%


\begin{document}

\date{}

%make title bold and 14 pt font (Latex default is non-bold, 16 pt)
\title{Architecture for a Next-Generation GCC}

\author{
Chris Lattner \hspace*{0.5in} Vikram Adve\\
\emph{University of Illinois at Urbana, Champaign}\\
\texttt{\em\normalsize \{lattner, vadve\}@cs.uiuc.edu}\\
\emph{\normalsize \url{http://llvm.cs.uiuc.edu}}} 

\maketitle

% You have to do this to suppress page numbers.  Don't ask.
\thispagestyle{empty}

Formatting team's note:  The two figures here illustrate two ways of presenting
the same information, and are hopefully more complex
than you'll require.  The first is set using Proceedings-friendly
packages; the second works only as a standalone paper.

%%% Figure typeset in a Proceedings-friendly fashion
%%% (thanks to Diego Novillo for inspiration)
\begin{figure*}[t]
\scriptsize
%%% \centering
\subfigure[Example function]{%
\label{figure:example_c}
\parbox{0.65\columnwidth}{\begin{cprog}
typedef struct QuadTree {
  double Data;
  struct QuadTree 
     *Children[4];
} QT;

void Sum3rdChildren(QT *T,
           double *Result) {
  double Ret;
  if (T == 0) { Ret = 0;
  } else {
    QT *Child3 =
      T[0].Children[3];
    double V;
    Sum3rdChildren(Child3, 
                   &V);
    Ret = V + T[0].Data;
  }
  *Result = Ret;
}
\end{cprog} 
}
}\hspace*{5pt}\vrule\hspace*{5pt}
\subfigure[Corresponding LLVM code] {%
\label{figure:example_llvm}
\parbox{1.35\columnwidth}{\begin{verbatim}
%struct.QuadTree = type { double, [4 x %QT*] }
%QT = type %struct.QuadTree

void %Sum3rdChildren(%QT* %T, double* %Result) {
entry: %V = alloca double            ;; %V is type 'double*'
       %tmp.0 = seteq %QT* %T, null  ;; type 'bool'
       br bool %tmp.0, label %endif, label %else

else:  ;;tmp.1 = &T[0].Children[3]  'Children' = Field #1
       %tmp.1 = getelementptr %QT* %T, long 0, ubyte 1, long 3
       %Child3 = load %QT** %tmp.1
       call void %Sum3rdChildren(%QT* %Child3, double* %V)
       %tmp.2 = load double* %V
       %tmp.3 = getelementptr %QT* %T, long 0, ubyte 0
       %tmp.4 = load double* %tmp.3
       %tmp.5 = add double %tmp.2, %tmp.4
       br label %endif

endif: %Ret = phi double [ %tmp.5, %else ], [ 0.0, %entry ]
       store double %Ret, double* %Result
       ret void  ;; Return with no value
}
\end{verbatim}
}}
%%% }%
\caption{C and LLVM code for a function}
\label{figure:example}
\end{figure*}

%%===------------------------
%  Code example figure
%
\begin{figure*} [t]
\scriptsize
\centering
\subfigure[Example function] {
\label{figure2:example_c}
\lstset{language=c}
\lstinputlisting{Figures/example.c}
}\hspace*{5pt}\vrule\hspace*{5pt}
\subfigure[Corresponding LLVM code] {
\label{figure2:example_llvm}
\lstset{language=LLVM}
\lstinputlisting{Figures/example.ll}
}%
\caption{C and LLVM code for a function}
\label{figure2:example}
\end{figure*}
%
%%===------------------------


\end{document}


