%% LyX 1.3 created this file.  For more info, see http://www.lyx.org/.
%% Do not edit unless you really know what you are doing.
\documentclass[english]{article}
\usepackage[T1]{fontenc}
\usepackage[latin1]{inputenc}

\makeatletter
%%%%%%%%%%%%%%%%%%%%%%%%%%%%%% User specified LaTeX commands.
\usepackage{babel}

\usepackage{babel}
\makeatother
\begin{document}

\subparagraph{GANESHA, a multi-usage with large cache NFSv4 server. Work In Progress
Submission}

The NFSv4 protocol is quite different from former versions of the
NFS protocol: it provides enhanced caching capability and is generic
enough in its semantic to support many different filesystem. GANESHA
is a NFS server that runs in the user space. Its design is oriented
to take advantage of these two aspects: aggressive caching and file
system genericity.

Caching is done for data and metadata. The metadata cache is fully
built within memory, it is based on a associative addressing made
via hash tables developed for that purpose. These hash tables are
using sets of RBT (Red Black Trees) and have performance that scales
like O( log( n ) ) where n is the number of entries stored. First
feedback showed that caching up to a million of inodes records with
GANESHA's metadata cache was possible without loosing performances.
The GANESHA server runs in user space which makes it able of allocating
huge piece of memory (several Gigabytes). Ambition of the product
is to keep several days of {}``active inodes'' in memory for several
days.

GANESHA uses backend modules called {}``File System Abstraction Layers''
or FSAL. A FSAL provide with all the necessary calls to access a file
system. NFSv4 requires only a few {}``madatory'' attributes for
files and directory, which makes its possible to address very basic
file system. Due to this, FSAL can be built on top of sets of data
organized as trees, where each entry has a name. SNMP and LDAP for
example meets the requirement for building a FSAL on top of them and
being {}``plugged'' into GANESHA.

For now, GANESHA supports FSAL to address the following set of data:

\begin{itemize}
\item The HPSS Namespace (HPSS is an HSM developed by the DOE and IBM Government
Systems). GANESHA with the FSAL/HPSS module makes the NFS support
for HPSS 
\item {}``Posix FSAL'' which provides a way to export via NFS any file
systems that have no {}``handle based'' API but only the standard
POSIX interface within libc 
\end{itemize}
The following FSAL are under development:

\begin{itemize}
\item {}``NFSv4 Client FSAL'': having a FSAL that is a NFSv4 client, with
GANESHA's aggressive caching will turn GANESHA into a NFSv4 proxy 
\item SNMP FSAL: this FSAL would makes its possible to browse SNMP information
through GANESHA via a mount point, with something similar to the /proc
file system 
\item LDAP FSAL: for mounting LDAP tree and parse it via a NFS mount point,
with something similar to /proc 
\end{itemize}
GANESHA is used on our site in production for one year. A website
and a submission to Freshmeat and Sourceforge should be done in a
few weeks. \textbf{Contact: Philippe Deniel (philippe.deniel@cea.fr).} 
\end{document}
