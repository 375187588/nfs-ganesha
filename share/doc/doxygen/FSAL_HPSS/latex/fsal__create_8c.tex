\section{fsal\_\-create.c File Reference}
\label{fsal__create_8c}\index{fsal\_\-create.c@{fsal\_\-create.c}}
Filesystem objects creation functions.  


{\tt \#include \char`\"{}fsal.h\char`\"{}}\par
{\tt \#include \char`\"{}fsal\_\-internal.h\char`\"{}}\par
{\tt \#include \char`\"{}fsal\_\-convert.h\char`\"{}}\par
{\tt \#include \char`\"{}fsal\_\-common.h\char`\"{}}\par
{\tt \#include \char`\"{}HPSSclapiExt/hpssclapiext.h\char`\"{}}\par
\subsection*{Functions}
\begin{CompactItemize}
\item 
fsal\_\-status\_\-t {\bf FSAL\_\-create} (fsal\_\-handle\_\-t $\ast$parent\_\-directory\_\-handle, fsal\_\-name\_\-t $\ast$p\_\-filename, fsal\_\-op\_\-context\_\-t $\ast$p\_\-context, fsal\_\-accessmode\_\-t accessmode, fsal\_\-handle\_\-t $\ast$object\_\-handle, fsal\_\-attrib\_\-list\_\-t $\ast$object\_\-attributes)
\item 
fsal\_\-status\_\-t {\bf FSAL\_\-mkdir} (fsal\_\-handle\_\-t $\ast$parent\_\-directory\_\-handle, fsal\_\-name\_\-t $\ast$p\_\-dirname, fsal\_\-op\_\-context\_\-t $\ast$p\_\-context, fsal\_\-accessmode\_\-t accessmode, fsal\_\-handle\_\-t $\ast$object\_\-handle, fsal\_\-attrib\_\-list\_\-t $\ast$object\_\-attributes)
\item 
fsal\_\-status\_\-t {\bf FSAL\_\-link} (fsal\_\-handle\_\-t $\ast$target\_\-handle, fsal\_\-handle\_\-t $\ast$dir\_\-handle, fsal\_\-name\_\-t $\ast$p\_\-link\_\-name, fsal\_\-op\_\-context\_\-t $\ast$p\_\-context, fsal\_\-attrib\_\-list\_\-t $\ast$attributes)
\item 
fsal\_\-status\_\-t {\bf FSAL\_\-mknode} (fsal\_\-handle\_\-t $\ast$parentdir\_\-handle, fsal\_\-name\_\-t $\ast$p\_\-node\_\-name, fsal\_\-op\_\-context\_\-t $\ast$p\_\-context, fsal\_\-accessmode\_\-t accessmode, fsal\_\-nodetype\_\-t nodetype, fsal\_\-dev\_\-t $\ast$dev, fsal\_\-handle\_\-t $\ast$p\_\-object\_\-handle, fsal\_\-attrib\_\-list\_\-t $\ast$node\_\-attributes)
\end{CompactItemize}


\subsection{Detailed Description}
Filesystem objects creation functions. 

\begin{Desc}
\item[Author:]\end{Desc}
\begin{Desc}
\item[Author]leibovic \end{Desc}
\begin{Desc}
\item[Date:]\end{Desc}
\begin{Desc}
\item[Date]2006/01/24 13:45:36 \end{Desc}
\begin{Desc}
\item[Version:]\end{Desc}
\begin{Desc}
\item[Revision]1.18 \end{Desc}


Definition in file {\bf fsal\_\-create.c}.

\subsection{Function Documentation}
\index{fsal\_\-create.c@{fsal\_\-create.c}!FSAL\_\-create@{FSAL\_\-create}}
\index{FSAL\_\-create@{FSAL\_\-create}!fsal_create.c@{fsal\_\-create.c}}
\subsubsection[{FSAL\_\-create}]{\setlength{\rightskip}{0pt plus 5cm}fsal\_\-status\_\-t FSAL\_\-create (fsal\_\-handle\_\-t $\ast$ {\em parent\_\-directory\_\-handle}, \/  fsal\_\-name\_\-t $\ast$ {\em p\_\-filename}, \/  fsal\_\-op\_\-context\_\-t $\ast$ {\em p\_\-context}, \/  fsal\_\-accessmode\_\-t {\em accessmode}, \/  fsal\_\-handle\_\-t $\ast$ {\em object\_\-handle}, \/  fsal\_\-attrib\_\-list\_\-t $\ast$ {\em object\_\-attributes})}\label{fsal__create_8c_b32064718b92ea123ea5cc7a6ff7cacd}


FSAL\_\-create: Create a regular file.

\begin{Desc}
\item[Parameters:]
\begin{description}
\item[{\em parent\_\-directory\_\-handle}](input): Handle of the parent directory where the file is to be created. \item[{\em p\_\-filename}](input): Pointer to the name of the file to be created. \item[{\em cred}](input): Authentication context for the operation (user, export...). \item[{\em accessmode}](input): Mode for the file to be created. (the umask defined into the FSAL configuration file will be applied on it). \item[{\em object\_\-handle}](output): Pointer to the handle of the created file. \item[{\em object\_\-attributes}](optionnal input/output): The postop attributes of the created file. As input, it defines the attributes that the caller wants to retrieve (by positioning flags into this structure) and the output is built considering this input (it fills the structure according to the flags it contains). Can be NULL.\end{description}
\end{Desc}
\begin{Desc}
\item[Returns:]Major error codes :\begin{itemize}
\item ERR\_\-FSAL\_\-NO\_\-ERROR (no error)\item ERR\_\-FSAL\_\-STALE (parent\_\-directory\_\-handle does not address an existing object)\item ERR\_\-FSAL\_\-FAULT (a NULL pointer was passed as mandatory argument)\item Other error codes can be returned : ERR\_\-FSAL\_\-ACCESS, ERR\_\-FSAL\_\-EXIST, ERR\_\-FSAL\_\-IO, ...\end{itemize}
\end{Desc}
NB: if getting postop attributes failed, the function does not return an error but the FSAL\_\-ATTR\_\-RDATTR\_\-ERR bit is set in the object\_\-attributes-$>$asked\_\-attributes field. 

Definition at line 62 of file fsal\_\-create.c.

References fsal2unix\_\-mode(), hpss2fsal\_\-attributes(), hpss2fsal\_\-error(), and TakeTokenFSCall().\index{fsal\_\-create.c@{fsal\_\-create.c}!FSAL\_\-link@{FSAL\_\-link}}
\index{FSAL\_\-link@{FSAL\_\-link}!fsal_create.c@{fsal\_\-create.c}}
\subsubsection[{FSAL\_\-link}]{\setlength{\rightskip}{0pt plus 5cm}fsal\_\-status\_\-t FSAL\_\-link (fsal\_\-handle\_\-t $\ast$ {\em target\_\-handle}, \/  fsal\_\-handle\_\-t $\ast$ {\em dir\_\-handle}, \/  fsal\_\-name\_\-t $\ast$ {\em p\_\-link\_\-name}, \/  fsal\_\-op\_\-context\_\-t $\ast$ {\em p\_\-context}, \/  fsal\_\-attrib\_\-list\_\-t $\ast$ {\em attributes})}\label{fsal__create_8c_b835d0f160323a78d0512a5254e9c109}


FSAL\_\-link: Create a hardlink.

\begin{Desc}
\item[Parameters:]
\begin{description}
\item[{\em target\_\-handle}](input): Handle of the target object. \item[{\em dir\_\-handle}](input): Pointer to the directory handle where the hardlink is to be created. \item[{\em p\_\-link\_\-name}](input): Pointer to the name of the hardlink to be created. \item[{\em cred}](input): Authentication context for the operation (user,...). \item[{\em accessmode}](input): Mode for the directory to be created. (the umask defined into the FSAL configuration file will be applied on it). \item[{\em attributes}](optionnal input/output): The post\_\-operation attributes of the linked object. As input, it defines the attributes that the caller wants to retrieve (by positioning flags into this structure) and the output is built considering this input (it fills the structure according to the flags it contains). May be NULL.\end{description}
\end{Desc}
\begin{Desc}
\item[Returns:]Major error codes :\begin{itemize}
\item ERR\_\-FSAL\_\-NO\_\-ERROR (no error)\item ERR\_\-FSAL\_\-STALE (target\_\-handle or dir\_\-handle does not address an existing object)\item ERR\_\-FSAL\_\-FAULT (a NULL pointer was passed as mandatory argument)\item Other error codes can be returned : ERR\_\-FSAL\_\-ACCESS, ERR\_\-FSAL\_\-EXIST, ERR\_\-FSAL\_\-IO, ...\end{itemize}
\end{Desc}
NB: if getting postop attributes failed, the function does not return an error but the FSAL\_\-ATTR\_\-RDATTR\_\-ERR bit is set in the attributes-$>$asked\_\-attributes field. 

Definition at line 363 of file fsal\_\-create.c.

References FSAL\_\-getattrs(), hpss2fsal\_\-error(), and TakeTokenFSCall().\index{fsal\_\-create.c@{fsal\_\-create.c}!FSAL\_\-mkdir@{FSAL\_\-mkdir}}
\index{FSAL\_\-mkdir@{FSAL\_\-mkdir}!fsal_create.c@{fsal\_\-create.c}}
\subsubsection[{FSAL\_\-mkdir}]{\setlength{\rightskip}{0pt plus 5cm}fsal\_\-status\_\-t FSAL\_\-mkdir (fsal\_\-handle\_\-t $\ast$ {\em parent\_\-directory\_\-handle}, \/  fsal\_\-name\_\-t $\ast$ {\em p\_\-dirname}, \/  fsal\_\-op\_\-context\_\-t $\ast$ {\em p\_\-context}, \/  fsal\_\-accessmode\_\-t {\em accessmode}, \/  fsal\_\-handle\_\-t $\ast$ {\em object\_\-handle}, \/  fsal\_\-attrib\_\-list\_\-t $\ast$ {\em object\_\-attributes})}\label{fsal__create_8c_8d7790c659b715e2e87bc57595a2cd71}


FSAL\_\-mkdir: Create a directory.

\begin{Desc}
\item[Parameters:]
\begin{description}
\item[{\em parent\_\-directory\_\-handle}](input): Handle of the parent directory where the subdirectory is to be created. \item[{\em p\_\-dirname}](input): Pointer to the name of the directory to be created. \item[{\em cred}](input): Authentication context for the operation (user,...). \item[{\em accessmode}](input): Mode for the directory to be created. (the umask defined into the FSAL configuration file will be applied on it). \item[{\em object\_\-handle}](output): Pointer to the handle of the created directory. \item[{\em object\_\-attributes}](optionnal input/output): The attributes of the created directory. As input, it defines the attributes that the caller wants to retrieve (by positioning flags into this structure) and the output is built considering this input (it fills the structure according to the flags it contains). May be NULL.\end{description}
\end{Desc}
\begin{Desc}
\item[Returns:]Major error codes :\begin{itemize}
\item ERR\_\-FSAL\_\-NO\_\-ERROR (no error)\item ERR\_\-FSAL\_\-STALE (parent\_\-directory\_\-handle does not address an existing object)\item ERR\_\-FSAL\_\-FAULT (a NULL pointer was passed as mandatory argument)\item Other error codes can be returned : ERR\_\-FSAL\_\-ACCESS, ERR\_\-FSAL\_\-EXIST, ERR\_\-FSAL\_\-IO, ...\end{itemize}
\end{Desc}
NB: if getting postop attributes failed, the function does not return an error but the FSAL\_\-ATTR\_\-RDATTR\_\-ERR bit is set in the object\_\-attributes-$>$asked\_\-attributes field. 

Definition at line 232 of file fsal\_\-create.c.

References fsal2unix\_\-mode(), hpss2fsal\_\-attributes(), hpss2fsal\_\-error(), and TakeTokenFSCall().\index{fsal\_\-create.c@{fsal\_\-create.c}!FSAL\_\-mknode@{FSAL\_\-mknode}}
\index{FSAL\_\-mknode@{FSAL\_\-mknode}!fsal_create.c@{fsal\_\-create.c}}
\subsubsection[{FSAL\_\-mknode}]{\setlength{\rightskip}{0pt plus 5cm}fsal\_\-status\_\-t FSAL\_\-mknode (fsal\_\-handle\_\-t $\ast$ {\em parentdir\_\-handle}, \/  fsal\_\-name\_\-t $\ast$ {\em p\_\-node\_\-name}, \/  fsal\_\-op\_\-context\_\-t $\ast$ {\em p\_\-context}, \/  fsal\_\-accessmode\_\-t {\em accessmode}, \/  fsal\_\-nodetype\_\-t {\em nodetype}, \/  fsal\_\-dev\_\-t $\ast$ {\em dev}, \/  fsal\_\-handle\_\-t $\ast$ {\em p\_\-object\_\-handle}, \/  fsal\_\-attrib\_\-list\_\-t $\ast$ {\em node\_\-attributes})}\label{fsal__create_8c_344a6fd2ad23cc03509f7e7f3bf1fd1c}


FSAL\_\-mknode: Create a special object in the filesystem. Not supported upon HPSS.

\begin{Desc}
\item[Returns:]ERR\_\-FSAL\_\-NOTSUPP. \end{Desc}


Definition at line 453 of file fsal\_\-create.c.