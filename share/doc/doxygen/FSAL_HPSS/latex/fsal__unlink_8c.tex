\section{fsal\_\-unlink.c File Reference}
\label{fsal__unlink_8c}\index{fsal\_\-unlink.c@{fsal\_\-unlink.c}}
object removing function.  


{\tt \#include \char`\"{}fsal.h\char`\"{}}\par
{\tt \#include \char`\"{}fsal\_\-internal.h\char`\"{}}\par
{\tt \#include \char`\"{}fsal\_\-convert.h\char`\"{}}\par
\subsection*{Functions}
\begin{CompactItemize}
\item 
fsal\_\-status\_\-t {\bf FSAL\_\-unlink} (fsal\_\-handle\_\-t $\ast$parentdir\_\-handle, fsal\_\-name\_\-t $\ast$p\_\-object\_\-name, fsal\_\-op\_\-context\_\-t $\ast$p\_\-context, fsal\_\-attrib\_\-list\_\-t $\ast$parentdir\_\-attributes)
\end{CompactItemize}


\subsection{Detailed Description}
object removing function. 

\begin{Desc}
\item[Author:]\end{Desc}
\begin{Desc}
\item[Author]leibovic \end{Desc}
\begin{Desc}
\item[Date:]\end{Desc}
\begin{Desc}
\item[Date]2006/01/24 13:45:37 \end{Desc}
\begin{Desc}
\item[Version:]\end{Desc}
\begin{Desc}
\item[Revision]1.9 \end{Desc}


Definition in file {\bf fsal\_\-unlink.c}.

\subsection{Function Documentation}
\index{fsal\_\-unlink.c@{fsal\_\-unlink.c}!FSAL\_\-unlink@{FSAL\_\-unlink}}
\index{FSAL\_\-unlink@{FSAL\_\-unlink}!fsal_unlink.c@{fsal\_\-unlink.c}}
\subsubsection[{FSAL\_\-unlink}]{\setlength{\rightskip}{0pt plus 5cm}fsal\_\-status\_\-t FSAL\_\-unlink (fsal\_\-handle\_\-t $\ast$ {\em parentdir\_\-handle}, \/  fsal\_\-name\_\-t $\ast$ {\em p\_\-object\_\-name}, \/  fsal\_\-op\_\-context\_\-t $\ast$ {\em p\_\-context}, \/  fsal\_\-attrib\_\-list\_\-t $\ast$ {\em parentdir\_\-attributes})}\label{fsal__unlink_8c_313314120a0ca14f6b41aea997f47285}


FSAL\_\-unlink: Remove a filesystem object .

\begin{Desc}
\item[Parameters:]
\begin{description}
\item[{\em parentdir\_\-handle}](input): Handle of the parent directory of the object to be deleted. \item[{\em p\_\-object\_\-name}](input): Name of the object to be removed. \item[{\em p\_\-context}](input): Authentication context for the operation (user,...). \item[{\em parentdir\_\-attributes}](optionnal input/output): Post operation attributes of the parent directory. As input, it defines the attributes that the caller wants to retrieve (by positioning flags into this structure) and the output is built considering this input (it fills the structure according to the flags it contains). May be NULL.\end{description}
\end{Desc}
\begin{Desc}
\item[Returns:]Major error codes :\begin{itemize}
\item ERR\_\-FSAL\_\-NO\_\-ERROR (no error)\item ERR\_\-FSAL\_\-STALE (parentdir\_\-handle does not address an existing object)\item ERR\_\-FSAL\_\-NOTDIR (parentdir\_\-handle does not address a directory)\item ERR\_\-FSAL\_\-NOENT (the object designated by p\_\-object\_\-name does not exist)\item ERR\_\-FSAL\_\-NOTEMPTY (tried to remove a non empty directory)\item ERR\_\-FSAL\_\-FAULT (a NULL pointer was passed as mandatory argument)\item Other error codes can be returned : ERR\_\-FSAL\_\-ACCESS, ERR\_\-FSAL\_\-IO, ... \end{itemize}
\end{Desc}


Definition at line 52 of file fsal\_\-unlink.c.

References FSAL\_\-getattrs(), FSAL\_\-lookup(), hpss2fsal\_\-error(), and TakeTokenFSCall().