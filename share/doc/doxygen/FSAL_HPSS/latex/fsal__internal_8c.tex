\section{fsal\_\-internal.c File Reference}
\label{fsal__internal_8c}\index{fsal\_\-internal.c@{fsal\_\-internal.c}}
Defines the datas that are to be accessed as extern by the fsal modules.  


{\tt \#include \char`\"{}fsal.h\char`\"{}}\par
{\tt \#include \char`\"{}fsal\_\-internal.h\char`\"{}}\par
{\tt \#include \char`\"{}stuff\_\-alloc.h\char`\"{}}\par
{\tt \#include \char`\"{}SemN.h\char`\"{}}\par
{\tt \#include $<$pthread.h$>$}\par
\subsection*{Defines}
\begin{CompactItemize}
\item 
\#define \textbf{FSAL\_\-INTERNAL\_\-C}\label{fsal__internal_8c_cfa63689895f5e17e205bd65f0178c0d}

\item 
\#define \textbf{SET\_\-INTEGER\_\-PARAM}(cfg, p\_\-init\_\-info, \_\-field)
\item 
\#define \textbf{SET\_\-BITMAP\_\-PARAM}(cfg, p\_\-init\_\-info, \_\-field)
\item 
\#define \textbf{SET\_\-BOOLEAN\_\-PARAM}(cfg, p\_\-init\_\-info, \_\-field)
\end{CompactItemize}
\subsection*{Functions}
\begin{CompactItemize}
\item 
void {\bf fsal\_\-increment\_\-nbcall} (int function\_\-index, fsal\_\-status\_\-t status)
\item 
void {\bf fsal\_\-internal\_\-getstats} (fsal\_\-statistics\_\-t $\ast$output\_\-stats)
\item 
void {\bf fsal\_\-internal\_\-SetCredentialLifetime} (fsal\_\-uint\_\-t lifetime\_\-in)
\item 
void {\bf fsal\_\-internal\_\-SetReturnInconsistentDirent} (fsal\_\-uint\_\-t bool\_\-in)
\item 
void {\bf TakeTokenFSCall} ()
\item 
void \textbf{ReleaseTokenFSCall} ()\label{fsal__internal_8c_3ebf10f203cc6b718170d47f661b101a}

\item 
fsal\_\-status\_\-t \textbf{fsal\_\-internal\_\-init\_\-global} (fsal\_\-init\_\-info\_\-t $\ast$fsal\_\-info, fs\_\-common\_\-initinfo\_\-t $\ast$fs\_\-common\_\-info)\label{fsal__internal_8c_806e3a89735ee912440d4b6e60a20dae}

\item 
fsal\_\-boolean\_\-t {\bf fsal\_\-do\_\-log} (fsal\_\-status\_\-t status)
\end{CompactItemize}
\subsection*{Variables}
\begin{CompactItemize}
\item 
fsal\_\-uint\_\-t \textbf{CredentialLifetime} = 3600\label{fsal__internal_8c_f3cb7380eea92f3a001fbcbe517e8e1c}

\item 
fsal\_\-uint\_\-t \textbf{ReturnInconsistentDirent} = FALSE\label{fsal__internal_8c_c3af4311d2c1075371c73ef185dcdda3}

\item 
fsal\_\-staticfsinfo\_\-t \textbf{global\_\-fs\_\-info}\label{fsal__internal_8c_2c43caebcfbbfc95b5d4917b99d70fa1}

\item 
log\_\-t \textbf{fsal\_\-log}\label{fsal__internal_8c_40064e582b2112b48cbb377868b840b2}

\item 
semaphore\_\-t \textbf{sem\_\-fs\_\-calls}\label{fsal__internal_8c_d249483e6d6223333f220241279dc57d}

\end{CompactItemize}


\subsection{Detailed Description}
Defines the datas that are to be accessed as extern by the fsal modules. 

\begin{Desc}
\item[Author:]\end{Desc}
\begin{Desc}
\item[Author]leibovic \end{Desc}
\begin{Desc}
\item[Date:]\end{Desc}
\begin{Desc}
\item[Date]2006/02/08 12:46:59 \end{Desc}
\begin{Desc}
\item[Version:]\end{Desc}
\begin{Desc}
\item[Revision]1.25 \end{Desc}


Definition in file {\bf fsal\_\-internal.c}.

\subsection{Define Documentation}
\index{fsal\_\-internal.c@{fsal\_\-internal.c}!SET\_\-BITMAP\_\-PARAM@{SET\_\-BITMAP\_\-PARAM}}
\index{SET\_\-BITMAP\_\-PARAM@{SET\_\-BITMAP\_\-PARAM}!fsal_internal.c@{fsal\_\-internal.c}}
\subsubsection[{SET\_\-BITMAP\_\-PARAM}]{\setlength{\rightskip}{0pt plus 5cm}\#define SET\_\-BITMAP\_\-PARAM(cfg, \/  p\_\-init\_\-info, \/  \_\-field)}\label{fsal__internal_8c_ba08a3bf394576d1a142b9f4b81867bf}


\textbf{Value:}

\begin{Code}\begin{verbatim}switch( (p_init_info)->behaviors._field ){                    \
    case FSAL_INIT_FORCE_VALUE :                                  \
        /* force the value in any case */                         \
        cfg._field = (p_init_info)->values._field;                \
        break;                                                    \
    case FSAL_INIT_MAX_LIMIT :                                    \
      /* proceed a bit AND */                                     \
      cfg._field &= (p_init_info)->values._field ;                \
      break;                                                      \
    case FSAL_INIT_MIN_LIMIT :                                    \
      /* proceed a bit OR */                                      \
      cfg._field |= (p_init_info)->values._field ;                \
      break;                                                      \
    /* In the other cases, we keep the default value. */          \
    }
\end{verbatim}
\end{Code}


Definition at line 311 of file fsal\_\-internal.c.\index{fsal\_\-internal.c@{fsal\_\-internal.c}!SET\_\-BOOLEAN\_\-PARAM@{SET\_\-BOOLEAN\_\-PARAM}}
\index{SET\_\-BOOLEAN\_\-PARAM@{SET\_\-BOOLEAN\_\-PARAM}!fsal_internal.c@{fsal\_\-internal.c}}
\subsubsection[{SET\_\-BOOLEAN\_\-PARAM}]{\setlength{\rightskip}{0pt plus 5cm}\#define SET\_\-BOOLEAN\_\-PARAM(cfg, \/  p\_\-init\_\-info, \/  \_\-field)}\label{fsal__internal_8c_2d4e102fe0da3b9340d2448907db42c6}


\textbf{Value:}

\begin{Code}\begin{verbatim}switch( (p_init_info)->behaviors._field ){                    \
    case FSAL_INIT_FORCE_VALUE :                                  \
        /* force the value in any case */                         \
        cfg._field = (p_init_info)->values._field;                \
        break;                                                    \
    case FSAL_INIT_MAX_LIMIT :                                    \
      /* proceed a boolean AND */                                 \
      cfg._field = cfg._field && (p_init_info)->values._field ;   \
      break;                                                      \
    case FSAL_INIT_MIN_LIMIT :                                    \
      /* proceed a boolean OR */                                  \
      cfg._field = cfg._field && (p_init_info)->values._field ;   \
      break;                                                      \
    /* In the other cases, we keep the default value. */          \
    }
\end{verbatim}
\end{Code}


Definition at line 329 of file fsal\_\-internal.c.\index{fsal\_\-internal.c@{fsal\_\-internal.c}!SET\_\-INTEGER\_\-PARAM@{SET\_\-INTEGER\_\-PARAM}}
\index{SET\_\-INTEGER\_\-PARAM@{SET\_\-INTEGER\_\-PARAM}!fsal_internal.c@{fsal\_\-internal.c}}
\subsubsection[{SET\_\-INTEGER\_\-PARAM}]{\setlength{\rightskip}{0pt plus 5cm}\#define SET\_\-INTEGER\_\-PARAM(cfg, \/  p\_\-init\_\-info, \/  \_\-field)}\label{fsal__internal_8c_1338c45d33ed60654787c72596f36adc}


\textbf{Value:}

\begin{Code}\begin{verbatim}switch( (p_init_info)->behaviors._field ){                    \
    case FSAL_INIT_FORCE_VALUE :                                  \
      /* force the value in any case */                           \
      cfg._field = (p_init_info)->values._field;                  \
      break;                                                      \
    case FSAL_INIT_MAX_LIMIT :                                    \
      /* check the higher limit */                                \
      if ( cfg._field > (p_init_info)->values._field )            \
        cfg._field = (p_init_info)->values._field ;               \
      break;                                                      \
    case FSAL_INIT_MIN_LIMIT :                                    \
      /* check the lower limit */                                 \
      if ( cfg._field < (p_init_info)->values._field )            \
        cfg._field = (p_init_info)->values._field ;               \
      break;                                                      \
    /* In the other cases, we keep the default value. */          \
    }
\end{verbatim}
\end{Code}


Definition at line 291 of file fsal\_\-internal.c.

\subsection{Function Documentation}
\index{fsal\_\-internal.c@{fsal\_\-internal.c}!fsal\_\-do\_\-log@{fsal\_\-do\_\-log}}
\index{fsal\_\-do\_\-log@{fsal\_\-do\_\-log}!fsal_internal.c@{fsal\_\-internal.c}}
\subsubsection[{fsal\_\-do\_\-log}]{\setlength{\rightskip}{0pt plus 5cm}fsal\_\-boolean\_\-t fsal\_\-do\_\-log (fsal\_\-status\_\-t {\em status})}\label{fsal__internal_8c_07976458ca604abb5159c050190c53d4}


fsal\_\-do\_\-log: Indicates if an FSAL error has to be traced into its log file in the NIV\_\-EVENT level. (in the other cases, return codes are only logged in the NIV\_\-FULL\_\-DEBUG logging lovel).

\begin{Desc}
\item[Parameters:]
\begin{description}
\item[{\em status(input),:}]The fsal status that is to be tested.\end{description}
\end{Desc}
\begin{Desc}
\item[Returns:]- TRUE if the error is to be traced.\begin{itemize}
\item FALSE if the error must not be traced except in NIV\_\-FULL\_\-DEBUG level. \end{itemize}
\end{Desc}


Definition at line 465 of file fsal\_\-internal.c.\index{fsal\_\-internal.c@{fsal\_\-internal.c}!fsal\_\-increment\_\-nbcall@{fsal\_\-increment\_\-nbcall}}
\index{fsal\_\-increment\_\-nbcall@{fsal\_\-increment\_\-nbcall}!fsal_internal.c@{fsal\_\-internal.c}}
\subsubsection[{fsal\_\-increment\_\-nbcall}]{\setlength{\rightskip}{0pt plus 5cm}void fsal\_\-increment\_\-nbcall (int {\em function\_\-index}, \/  fsal\_\-status\_\-t {\em status})}\label{fsal__internal_8c_79dcb2ba7f283a1ba5c6010d76fa88f1}


fsal\_\-increment\_\-nbcall: Updates fonction call statistics.

\begin{Desc}
\item[Parameters:]
\begin{description}
\item[{\em function\_\-index}](input): Index of the function whom number of call is to be incremented. \item[{\em status}](input): Status the function returned.\end{description}
\end{Desc}
\begin{Desc}
\item[Returns:]Nothing. \end{Desc}


Definition at line 107 of file fsal\_\-internal.c.

References fsal\_\-is\_\-retryable().\index{fsal\_\-internal.c@{fsal\_\-internal.c}!fsal\_\-internal\_\-getstats@{fsal\_\-internal\_\-getstats}}
\index{fsal\_\-internal\_\-getstats@{fsal\_\-internal\_\-getstats}!fsal_internal.c@{fsal\_\-internal.c}}
\subsubsection[{fsal\_\-internal\_\-getstats}]{\setlength{\rightskip}{0pt plus 5cm}void fsal\_\-internal\_\-getstats (fsal\_\-statistics\_\-t $\ast$ {\em output\_\-stats})}\label{fsal__internal_8c_6de88949985cac9da4cd3fb302fb80a7}


fsal\_\-internal\_\-getstats: (For internal use in the FSAL). Retrieve call statistics for current thread.

\begin{Desc}
\item[Parameters:]
\begin{description}
\item[{\em output\_\-stats}](output): Pointer to the call statistics structure.\end{description}
\end{Desc}
\begin{Desc}
\item[Returns:]Nothing. \end{Desc}


Definition at line 186 of file fsal\_\-internal.c.

Referenced by FSAL\_\-get\_\-stats().\index{fsal\_\-internal.c@{fsal\_\-internal.c}!fsal\_\-internal\_\-SetCredentialLifetime@{fsal\_\-internal\_\-SetCredentialLifetime}}
\index{fsal\_\-internal\_\-SetCredentialLifetime@{fsal\_\-internal\_\-SetCredentialLifetime}!fsal_internal.c@{fsal\_\-internal.c}}
\subsubsection[{fsal\_\-internal\_\-SetCredentialLifetime}]{\setlength{\rightskip}{0pt plus 5cm}void fsal\_\-internal\_\-SetCredentialLifetime (fsal\_\-uint\_\-t {\em lifetime\_\-in})}\label{fsal__internal_8c_52fbd60f6f8f222868f133a1ca585bd1}


Set credential lifetime. (For internal use in the FSAL). Set the period for thread's credential renewal.

\begin{Desc}
\item[Parameters:]
\begin{description}
\item[{\em lifetime\_\-in}](input): The period for thread's credential renewal.\end{description}
\end{Desc}
\begin{Desc}
\item[Returns:]Nothing. \end{Desc}


Definition at line 243 of file fsal\_\-internal.c.

Referenced by FSAL\_\-Init().\index{fsal\_\-internal.c@{fsal\_\-internal.c}!fsal\_\-internal\_\-SetReturnInconsistentDirent@{fsal\_\-internal\_\-SetReturnInconsistentDirent}}
\index{fsal\_\-internal\_\-SetReturnInconsistentDirent@{fsal\_\-internal\_\-SetReturnInconsistentDirent}!fsal_internal.c@{fsal\_\-internal.c}}
\subsubsection[{fsal\_\-internal\_\-SetReturnInconsistentDirent}]{\setlength{\rightskip}{0pt plus 5cm}void fsal\_\-internal\_\-SetReturnInconsistentDirent (fsal\_\-uint\_\-t {\em bool\_\-in})}\label{fsal__internal_8c_c9a6a99ea408b452eb6f232ae3ed5271}


Set behavior when detecting a MD inconsistency in readdir. (For internal use in the FSAL).

\begin{Desc}
\item[Parameters:]
\begin{description}
\item[{\em lifetime\_\-in}](input): The period for thread's credential renewal.\end{description}
\end{Desc}
\begin{Desc}
\item[Returns:]Nothing. \end{Desc}


Definition at line 258 of file fsal\_\-internal.c.

Referenced by FSAL\_\-Init().\index{fsal\_\-internal.c@{fsal\_\-internal.c}!TakeTokenFSCall@{TakeTokenFSCall}}
\index{TakeTokenFSCall@{TakeTokenFSCall}!fsal_internal.c@{fsal\_\-internal.c}}
\subsubsection[{TakeTokenFSCall}]{\setlength{\rightskip}{0pt plus 5cm}void TakeTokenFSCall ()}\label{fsal__internal_8c_880a1463c400047bfd0401f0b9c431a7}


Used to limit the number of simultaneous calls to Filesystem. 

Definition at line 268 of file fsal\_\-internal.c.

Referenced by FSAL\_\-access(), FSAL\_\-close(), FSAL\_\-create(), FSAL\_\-dynamic\_\-fsinfo(), FSAL\_\-getattrs(), FSAL\_\-link(), FSAL\_\-lookup(), FSAL\_\-lookupJunction(), FSAL\_\-mkdir(), FSAL\_\-open(), FSAL\_\-read(), FSAL\_\-readdir(), FSAL\_\-readlink(), FSAL\_\-rename(), FSAL\_\-setattrs(), FSAL\_\-symlink(), FSAL\_\-truncate(), FSAL\_\-unlink(), and FSAL\_\-write().