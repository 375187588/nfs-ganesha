\section{fsal\_\-attrs.c File Reference}
\label{fsal__attrs_8c}\index{fsal\_\-attrs.c@{fsal\_\-attrs.c}}
Attributes functions.  


{\tt \#include \char`\"{}fsal.h\char`\"{}}\par
{\tt \#include \char`\"{}fsal\_\-internal.h\char`\"{}}\par
{\tt \#include \char`\"{}fsal\_\-convert.h\char`\"{}}\par
{\tt \#include \char`\"{}HPSSclapiExt/hpssclapiext.h\char`\"{}}\par
\subsection*{Functions}
\begin{CompactItemize}
\item 
fsal\_\-status\_\-t {\bf FSAL\_\-getattrs} (fsal\_\-handle\_\-t $\ast$filehandle, fsal\_\-op\_\-context\_\-t $\ast$p\_\-context, fsal\_\-attrib\_\-list\_\-t $\ast$object\_\-attributes)
\item 
fsal\_\-status\_\-t {\bf FSAL\_\-setattrs} (fsal\_\-handle\_\-t $\ast$filehandle, fsal\_\-op\_\-context\_\-t $\ast$p\_\-context, fsal\_\-attrib\_\-list\_\-t $\ast$attrib\_\-set, fsal\_\-attrib\_\-list\_\-t $\ast$object\_\-attributes)
\end{CompactItemize}


\subsection{Detailed Description}
Attributes functions. 

\begin{Desc}
\item[Author:]\end{Desc}
\begin{Desc}
\item[Author]leibovic \end{Desc}
\begin{Desc}
\item[Date:]\end{Desc}
\begin{Desc}
\item[Date]2005/09/09 15:22:49 \end{Desc}
\begin{Desc}
\item[Version:]\end{Desc}
\begin{Desc}
\item[Revision]1.19 \end{Desc}


Definition in file {\bf fsal\_\-attrs.c}.

\subsection{Function Documentation}
\index{fsal\_\-attrs.c@{fsal\_\-attrs.c}!FSAL\_\-getattrs@{FSAL\_\-getattrs}}
\index{FSAL\_\-getattrs@{FSAL\_\-getattrs}!fsal_attrs.c@{fsal\_\-attrs.c}}
\subsubsection[{FSAL\_\-getattrs}]{\setlength{\rightskip}{0pt plus 5cm}fsal\_\-status\_\-t FSAL\_\-getattrs (fsal\_\-handle\_\-t $\ast$ {\em filehandle}, \/  fsal\_\-op\_\-context\_\-t $\ast$ {\em p\_\-context}, \/  fsal\_\-attrib\_\-list\_\-t $\ast$ {\em object\_\-attributes})}\label{fsal__attrs_8c_1a28c6ab788c37b4658aeb6ed9536d7b}


FSAL\_\-getattrs: Get attributes for the object specified by its filehandle.

\begin{Desc}
\item[Parameters:]
\begin{description}
\item[{\em filehandle}](input): The handle of the object to get parameters. \item[{\em p\_\-context}](input): Authentication context for the operation (user, export...). \item[{\em object\_\-attributes}](mandatory input/output): The retrieved attributes for the object. As input, it defines the attributes that the caller wants to retrieve (by positioning flags into this structure) and the output is built considering this input (it fills the structure according to the flags it contains).\end{description}
\end{Desc}
\begin{Desc}
\item[Returns:]Major error codes :\begin{itemize}
\item ERR\_\-FSAL\_\-NO\_\-ERROR (no error)\item ERR\_\-FSAL\_\-STALE (object\_\-handle does not address an existing object)\item ERR\_\-FSAL\_\-FAULT (a NULL pointer was passed as mandatory argument)\item Another error code if an error occured. \end{itemize}
\end{Desc}


Definition at line 44 of file fsal\_\-attrs.c.

References hpss2fsal\_\-attributes(), hpss2fsal\_\-error(), and TakeTokenFSCall().

Referenced by FSAL\_\-access(), FSAL\_\-GetXAttrAttrs(), FSAL\_\-link(), FSAL\_\-ListXAttrs(), FSAL\_\-lookup(), FSAL\_\-readlink(), FSAL\_\-rename(), FSAL\_\-setattrs(), FSAL\_\-truncate(), and FSAL\_\-unlink().\index{fsal\_\-attrs.c@{fsal\_\-attrs.c}!FSAL\_\-setattrs@{FSAL\_\-setattrs}}
\index{FSAL\_\-setattrs@{FSAL\_\-setattrs}!fsal_attrs.c@{fsal\_\-attrs.c}}
\subsubsection[{FSAL\_\-setattrs}]{\setlength{\rightskip}{0pt plus 5cm}fsal\_\-status\_\-t FSAL\_\-setattrs (fsal\_\-handle\_\-t $\ast$ {\em filehandle}, \/  fsal\_\-op\_\-context\_\-t $\ast$ {\em p\_\-context}, \/  fsal\_\-attrib\_\-list\_\-t $\ast$ {\em attrib\_\-set}, \/  fsal\_\-attrib\_\-list\_\-t $\ast$ {\em object\_\-attributes})}\label{fsal__attrs_8c_c0463632f2c08701c012b7885b1d9012}


FSAL\_\-setattrs: Set attributes for the object specified by its filehandle.

\begin{Desc}
\item[Parameters:]
\begin{description}
\item[{\em filehandle}](input): The handle of the object to get parameters. \item[{\em p\_\-context}](input): Authentication context for the operation (user,...). \item[{\em attrib\_\-set}](mandatory input): The attributes to be set for the object. It defines the attributes that the caller wants to set and their values. \item[{\em object\_\-attributes}](optionnal input/output): The post operation attributes for the object. As input, it defines the attributes that the caller wants to retrieve (by positioning flags into this structure) and the output is built considering this input (it fills the structure according to the flags it contains). May be NULL.\end{description}
\end{Desc}
\begin{Desc}
\item[Returns:]Major error codes :\begin{itemize}
\item ERR\_\-FSAL\_\-NO\_\-ERROR (no error)\item ERR\_\-FSAL\_\-STALE (object\_\-handle does not address an existing object)\item ERR\_\-FSAL\_\-INVAL (tried to modify a read-only attribute)\item ERR\_\-FSAL\_\-ATTRNOTSUPP (tried to modify a non-supported attribute)\item ERR\_\-FSAL\_\-FAULT (a NULL pointer was passed as mandatory argument)\item Another error code if an error occured. NB: if getting postop attributes failed, the function does not return an error but the FSAL\_\-ATTR\_\-RDATTR\_\-ERR bit is set in the object\_\-attributes-$>$asked\_\-attributes field. \end{itemize}
\end{Desc}


\begin{Desc}
\item[{\bf Todo}]: chown restricted seems to be OK. \end{Desc}


\begin{Desc}
\item[{\bf Todo}]voir pourquoi hpss\_\-fattr\_\-out ne contient pas ce qu'il devrait contenir \end{Desc}


Definition at line 138 of file fsal\_\-attrs.c.

References fsal2hpss\_\-attribset(), FSAL\_\-getattrs(), hpss2fsal\_\-attributes(), hpss2fsal\_\-error(), and TakeTokenFSCall().