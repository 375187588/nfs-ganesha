\section{fsal\_\-fsinfo.c File Reference}
\label{fsal__fsinfo_8c}\index{fsal_fsinfo.c@{fsal\_\-fsinfo.c}}
functions for retrieving filesystem info. 

{\tt \#include \char`\"{}fsal.h\char`\"{}}\par
{\tt \#include \char`\"{}fsal\_\-internal.h\char`\"{}}\par
{\tt \#include \char`\"{}fsal\_\-convert.h\char`\"{}}\par
\subsection*{Functions}
\begin{CompactItemize}
\item 
fsal\_\-status\_\-t {\bf FSAL\_\-static\_\-fsinfo} (fsal\_\-handle\_\-t $\ast$filehandle, fsal\_\-op\_\-context\_\-t $\ast$p\_\-context, fsal\_\-staticfsinfo\_\-t $\ast$staticinfo)
\item 
fsal\_\-status\_\-t {\bf FSAL\_\-dynamic\_\-fsinfo} (fsal\_\-handle\_\-t $\ast$filehandle, fsal\_\-op\_\-context\_\-t $\ast$p\_\-context, fsal\_\-dynamicfsinfo\_\-t $\ast$dynamicinfo)
\end{CompactItemize}


\subsection{Detailed Description}
functions for retrieving filesystem info. 

\begin{Desc}
\item[Author:]\begin{Desc}
\item[Author]leibovic \end{Desc}
\end{Desc}
\begin{Desc}
\item[Date:]\begin{Desc}
\item[Date]2006/02/16 08:20:22 \end{Desc}
\end{Desc}
\begin{Desc}
\item[Version:]\begin{Desc}
\item[Revision]1.12 \end{Desc}
\end{Desc}


Definition in file {\bf fsal\_\-fsinfo.c}.

\subsection{Function Documentation}
\index{fsal_fsinfo.c@{fsal\_\-fsinfo.c}!FSAL_dynamic_fsinfo@{FSAL\_\-dynamic\_\-fsinfo}}
\index{FSAL_dynamic_fsinfo@{FSAL\_\-dynamic\_\-fsinfo}!fsal_fsinfo.c@{fsal\_\-fsinfo.c}}
\subsubsection{\setlength{\rightskip}{0pt plus 5cm}fsal\_\-status\_\-t FSAL\_\-dynamic\_\-fsinfo (fsal\_\-handle\_\-t $\ast$ {\em filehandle}, fsal\_\-op\_\-context\_\-t $\ast$ {\em p\_\-context}, fsal\_\-dynamicfsinfo\_\-t $\ast$ {\em dynamicinfo})}\label{fsal__fsinfo_8c_a1}


FSAL\_\-dynamic\_\-fsinfo: Return dynamic filesystem info such as used size, free size, number of objects...

\begin{Desc}
\item[Parameters:]
\begin{description}
\item[{\em filehandle}](input): Handle of an object in the filesystem whom info is to be retrieved. \item[{\em p\_\-context}](input): Authentication context for the operation (user,...). \item[{\em dynamicinfo}](output): Pointer to the static info of the filesystem.\end{description}
\end{Desc}
\begin{Desc}
\item[Returns:]Major error codes:\begin{itemize}
\item ERR\_\-FSAL\_\-NO\_\-ERROR (no error)\item ERR\_\-FSAL\_\-FAULT (a NULL pointer was passed as mandatory argument)\item Other error codes can be returned : ERR\_\-FSAL\_\-IO, ... \end{itemize}
\end{Desc}


Definition at line 85 of file fsal\_\-fsinfo.c.

References hpss2fsal\_\-error(), and Take\-Token\-FSCall().\index{fsal_fsinfo.c@{fsal\_\-fsinfo.c}!FSAL_static_fsinfo@{FSAL\_\-static\_\-fsinfo}}
\index{FSAL_static_fsinfo@{FSAL\_\-static\_\-fsinfo}!fsal_fsinfo.c@{fsal\_\-fsinfo.c}}
\subsubsection{\setlength{\rightskip}{0pt plus 5cm}fsal\_\-status\_\-t FSAL\_\-static\_\-fsinfo (fsal\_\-handle\_\-t $\ast$ {\em filehandle}, fsal\_\-op\_\-context\_\-t $\ast$ {\em p\_\-context}, fsal\_\-staticfsinfo\_\-t $\ast$ {\em staticinfo})}\label{fsal__fsinfo_8c_a0}


FSAL\_\-static\_\-fsinfo: Return static filesystem info such as behavior, configuration, supported operations...

\begin{Desc}
\item[Parameters:]
\begin{description}
\item[{\em filehandle}](input): Handle of an object in the filesystem whom info is to be retrieved. \item[{\em cred}](input): Authentication context for the operation (user,...). \item[{\em staticinfo}](output): Pointer to the static info of the filesystem.\end{description}
\end{Desc}
\begin{Desc}
\item[Returns:]Major error codes:\begin{itemize}
\item ERR\_\-FSAL\_\-NO\_\-ERROR (no error)\item ERR\_\-FSAL\_\-FAULT (a NULL pointer was passed as mandatory argument)\item Other error codes can be returned : ERR\_\-FSAL\_\-IO, ... \end{itemize}
\end{Desc}


Definition at line 47 of file fsal\_\-fsinfo.c.