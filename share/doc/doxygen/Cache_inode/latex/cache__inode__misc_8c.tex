\section{cache\_\-inode\_\-misc.c File Reference}
\label{cache__inode__misc_8c}\index{cache\_\-inode\_\-misc.c@{cache\_\-inode\_\-misc.c}}
{\tt \#include \char`\"{}LRU\_\-List.h\char`\"{}}\par
{\tt \#include \char`\"{}log\_\-functions.h\char`\"{}}\par
{\tt \#include \char`\"{}HashData.h\char`\"{}}\par
{\tt \#include \char`\"{}HashTable.h\char`\"{}}\par
{\tt \#include \char`\"{}fsal.h\char`\"{}}\par
{\tt \#include \char`\"{}cache\_\-inode.h\char`\"{}}\par
{\tt \#include \char`\"{}cache\_\-content.h\char`\"{}}\par
{\tt \#include \char`\"{}stuff\_\-alloc.h\char`\"{}}\par
{\tt \#include $<$unistd.h$>$}\par
{\tt \#include $<$sys/types.h$>$}\par
{\tt \#include $<$sys/param.h$>$}\par
{\tt \#include $<$time.h$>$}\par
{\tt \#include $<$pthread.h$>$}\par
{\tt \#include $<$string.h$>$}\par
\subsection*{Functions}
\begin{CompactItemize}
\item 
int {\bf cache\_\-inode\_\-compare\_\-key\_\-fsal} (hash\_\-buffer\_\-t $\ast$buff1, hash\_\-buffer\_\-t $\ast$buff2)
\item 
int {\bf cache\_\-inode\_\-fsaldata\_\-2\_\-key} (hash\_\-buffer\_\-t $\ast$pkey, cache\_\-inode\_\-fsal\_\-data\_\-t $\ast$pfsdata, cache\_\-inode\_\-client\_\-t $\ast$pclient)
\item 
void {\bf cache\_\-inode\_\-release\_\-fsaldata\_\-key} (hash\_\-buffer\_\-t $\ast$pkey, cache\_\-inode\_\-client\_\-t $\ast$pclient)
\item 
cache\_\-entry\_\-t $\ast$ {\bf cache\_\-inode\_\-new\_\-entry} (cache\_\-inode\_\-fsal\_\-data\_\-t $\ast$pfsdata, fsal\_\-attrib\_\-list\_\-t $\ast$pfsal\_\-attr, cache\_\-inode\_\-file\_\-type\_\-t type, cache\_\-inode\_\-create\_\-arg\_\-t $\ast$pcreate\_\-arg, cache\_\-entry\_\-t $\ast$pentry\_\-dir\_\-prev, hash\_\-table\_\-t $\ast$ht, cache\_\-inode\_\-client\_\-t $\ast$pclient, fsal\_\-op\_\-context\_\-t $\ast$pcontext, unsigned int create\_\-flag, cache\_\-inode\_\-status\_\-t $\ast$pstatus)
\item 
cache\_\-inode\_\-status\_\-t {\bf cache\_\-inode\_\-clean\_\-entry} (cache\_\-entry\_\-t $\ast$pentry)
\item 
cache\_\-inode\_\-status\_\-t {\bf cache\_\-inode\_\-error\_\-convert} (fsal\_\-status\_\-t fsal\_\-status)
\item 
cache\_\-inode\_\-status\_\-t {\bf cache\_\-inode\_\-valid} (cache\_\-entry\_\-t $\ast$pentry, cache\_\-inode\_\-op\_\-t op, cache\_\-inode\_\-client\_\-t $\ast$pclient)
\item 
void {\bf cache\_\-inode\_\-get\_\-attributes} (cache\_\-entry\_\-t $\ast$pentry, fsal\_\-attrib\_\-list\_\-t $\ast$pattr)
\item 
void {\bf cache\_\-inode\_\-set\_\-attributes} (cache\_\-entry\_\-t $\ast$pentry, fsal\_\-attrib\_\-list\_\-t $\ast$pattr)
\item 
cache\_\-inode\_\-file\_\-type\_\-t {\bf cache\_\-inode\_\-fsal\_\-type\_\-convert} (fsal\_\-nodetype\_\-t type)
\item 
fsal\_\-handle\_\-t $\ast$ {\bf cache\_\-inode\_\-get\_\-fsal\_\-handle} (cache\_\-entry\_\-t $\ast$pentry, cache\_\-inode\_\-status\_\-t $\ast$pstatus)
\item 
int {\bf cache\_\-inode\_\-type\_\-are\_\-rename\_\-compatible} (cache\_\-entry\_\-t $\ast$pentry\_\-src, cache\_\-entry\_\-t $\ast$pentry\_\-dest)
\item 
void {\bf cache\_\-inode\_\-mutex\_\-destroy} (cache\_\-entry\_\-t $\ast$pentry)
\item 
void {\bf cache\_\-inode\_\-print\_\-dir} (cache\_\-entry\_\-t $\ast$cache\_\-entry\_\-root)
\item 
cache\_\-inode\_\-status\_\-t {\bf cache\_\-inode\_\-dump\_\-content} (char $\ast$path, cache\_\-entry\_\-t $\ast$pentry)
\item 
cache\_\-inode\_\-status\_\-t {\bf cache\_\-inode\_\-reload\_\-content} (char $\ast$path, cache\_\-entry\_\-t $\ast$pentry)
\item 
cache\_\-inode\_\-status\_\-t {\bf cache\_\-inode\_\-kill\_\-entry} (cache\_\-entry\_\-t $\ast$pentry, hash\_\-table\_\-t $\ast$ht, cache\_\-inode\_\-client\_\-t $\ast$pclient, cache\_\-inode\_\-status\_\-t $\ast$pstatus)
\end{CompactItemize}


\subsection{Function Documentation}
\index{cache\_\-inode\_\-misc.c@{cache\_\-inode\_\-misc.c}!cache\_\-inode\_\-clean\_\-entry@{cache\_\-inode\_\-clean\_\-entry}}
\index{cache\_\-inode\_\-clean\_\-entry@{cache\_\-inode\_\-clean\_\-entry}!cache_inode_misc.c@{cache\_\-inode\_\-misc.c}}
\subsubsection[{cache\_\-inode\_\-clean\_\-entry}]{\setlength{\rightskip}{0pt plus 5cm}cache\_\-inode\_\-status\_\-t cache\_\-inode\_\-clean\_\-entry (cache\_\-entry\_\-t $\ast$ {\em pentry})}\label{cache__inode__misc_8c_efe5de533cbaea1af0370e4393c74c4c}


cache\_\-clean\_\-entry: cleans an entry when it is garbagge collected.

Cleans an entry when it is garbagge collected.

\begin{Desc}
\item[Parameters:]
\begin{description}
\item[{\em pentry}][INOUT] the entry to be cleaned.\end{description}
\end{Desc}
\begin{Desc}
\item[Returns:]CACHE\_\-INODE\_\-SUCCESS in all cases. \end{Desc}


Definition at line 693 of file cache\_\-inode\_\-misc.c.\index{cache\_\-inode\_\-misc.c@{cache\_\-inode\_\-misc.c}!cache\_\-inode\_\-compare\_\-key\_\-fsal@{cache\_\-inode\_\-compare\_\-key\_\-fsal}}
\index{cache\_\-inode\_\-compare\_\-key\_\-fsal@{cache\_\-inode\_\-compare\_\-key\_\-fsal}!cache_inode_misc.c@{cache\_\-inode\_\-misc.c}}
\subsubsection[{cache\_\-inode\_\-compare\_\-key\_\-fsal}]{\setlength{\rightskip}{0pt plus 5cm}int cache\_\-inode\_\-compare\_\-key\_\-fsal (hash\_\-buffer\_\-t $\ast$ {\em buff1}, \/  hash\_\-buffer\_\-t $\ast$ {\em buff2})}\label{cache__inode__misc_8c_b2ed61f21523805c300eae3d843ada57}


cache\_\-inode\_\-compare\_\-key\_\-fsal: Compares two keys used in cache inode.

Compare two keys used in cache inode. These keys are basically made from FSAL related information.

\begin{Desc}
\item[Parameters:]
\begin{description}
\item[{\em buff1}][IN] first key \item[{\em buff2}][IN] second key \end{description}
\end{Desc}
\begin{Desc}
\item[Returns:]0 if keys are the same, 1 otherwise\end{Desc}
\begin{Desc}
\item[See also:]FSAL\_\-handlecmp \end{Desc}


Definition at line 127 of file cache\_\-inode\_\-misc.c.\index{cache\_\-inode\_\-misc.c@{cache\_\-inode\_\-misc.c}!cache\_\-inode\_\-dump\_\-content@{cache\_\-inode\_\-dump\_\-content}}
\index{cache\_\-inode\_\-dump\_\-content@{cache\_\-inode\_\-dump\_\-content}!cache_inode_misc.c@{cache\_\-inode\_\-misc.c}}
\subsubsection[{cache\_\-inode\_\-dump\_\-content}]{\setlength{\rightskip}{0pt plus 5cm}cache\_\-inode\_\-status\_\-t cache\_\-inode\_\-dump\_\-content (char $\ast$ {\em path}, \/  cache\_\-entry\_\-t $\ast$ {\em pentry})}\label{cache__inode__misc_8c_d4bd35e3d32d663afea3499d1a29199e}


cache\_\-inode\_\-dump\_\-content: dumps the content of a pentry to a local file (used for File Content index files).

Dumps the content of a pentry to a local file (used for File Content index files).

\begin{Desc}
\item[Parameters:]
\begin{description}
\item[{\em path}][IN] the full path to the file that will contain the data. \item[{\em pentry}][IN] the input pentry.\end{description}
\end{Desc}
\begin{Desc}
\item[Returns:]CACHE\_\-INODE\_\-BAD\_\-TYPE if pentry is not related to a REGULAR\_\-FILE \par
 

CACHE\_\-INODE\_\-INVALID\_\-ARGUMENT if path is inconsistent \par
 

CACHE\_\-INODE\_\-SUCCESS if operation succeded. \end{Desc}


Definition at line 1286 of file cache\_\-inode\_\-misc.c.\index{cache\_\-inode\_\-misc.c@{cache\_\-inode\_\-misc.c}!cache\_\-inode\_\-error\_\-convert@{cache\_\-inode\_\-error\_\-convert}}
\index{cache\_\-inode\_\-error\_\-convert@{cache\_\-inode\_\-error\_\-convert}!cache_inode_misc.c@{cache\_\-inode\_\-misc.c}}
\subsubsection[{cache\_\-inode\_\-error\_\-convert}]{\setlength{\rightskip}{0pt plus 5cm}cache\_\-inode\_\-status\_\-t cache\_\-inode\_\-error\_\-convert (fsal\_\-status\_\-t {\em fsal\_\-status})}\label{cache__inode__misc_8c_aa50487b9b403fd85b4a845de1a3df69}


cache\_\-inode\_\-error\_\-convert: converts an FSAL error to the corresponding cache\_\-inode error.

Converts an FSAL error to the corresponding cache\_\-inode error.

\begin{Desc}
\item[Parameters:]
\begin{description}
\item[{\em fsal\_\-status}][IN] fsal error to be converted.\end{description}
\end{Desc}
\begin{Desc}
\item[Returns:]the result of the conversion. \end{Desc}


Definition at line 715 of file cache\_\-inode\_\-misc.c.\index{cache\_\-inode\_\-misc.c@{cache\_\-inode\_\-misc.c}!cache\_\-inode\_\-fsal\_\-type\_\-convert@{cache\_\-inode\_\-fsal\_\-type\_\-convert}}
\index{cache\_\-inode\_\-fsal\_\-type\_\-convert@{cache\_\-inode\_\-fsal\_\-type\_\-convert}!cache_inode_misc.c@{cache\_\-inode\_\-misc.c}}
\subsubsection[{cache\_\-inode\_\-fsal\_\-type\_\-convert}]{\setlength{\rightskip}{0pt plus 5cm}cache\_\-inode\_\-file\_\-type\_\-t cache\_\-inode\_\-fsal\_\-type\_\-convert (fsal\_\-nodetype\_\-t {\em type})}\label{cache__inode__misc_8c_8f6934bbf38300379caf1342b6a63b26}


cache\_\-inode\_\-fsal\_\-type\_\-convert: converts an FSAL type to the cache\_\-inode type to be used.

Converts an FSAL type to the cache\_\-inode type to be used.

\begin{Desc}
\item[Parameters:]
\begin{description}
\item[{\em type}][IN] the input FSAL type.\end{description}
\end{Desc}
\begin{Desc}
\item[Returns:]the result of the conversion. \end{Desc}


Definition at line 1048 of file cache\_\-inode\_\-misc.c.\index{cache\_\-inode\_\-misc.c@{cache\_\-inode\_\-misc.c}!cache\_\-inode\_\-fsaldata\_\-2\_\-key@{cache\_\-inode\_\-fsaldata\_\-2\_\-key}}
\index{cache\_\-inode\_\-fsaldata\_\-2\_\-key@{cache\_\-inode\_\-fsaldata\_\-2\_\-key}!cache_inode_misc.c@{cache\_\-inode\_\-misc.c}}
\subsubsection[{cache\_\-inode\_\-fsaldata\_\-2\_\-key}]{\setlength{\rightskip}{0pt plus 5cm}int cache\_\-inode\_\-fsaldata\_\-2\_\-key (hash\_\-buffer\_\-t $\ast$ {\em pkey}, \/  cache\_\-inode\_\-fsal\_\-data\_\-t $\ast$ {\em pfsdata}, \/  cache\_\-inode\_\-client\_\-t $\ast$ {\em pclient})}\label{cache__inode__misc_8c_7bf5883d643f1a7940f2290a6d2d9686}


cache\_\-inode\_\-fsaldata\_\-2\_\-key: builds a key from the FSAL data.

Builds a key from the FSAL data. If the key is used for reading and stay local to the function pclient can be NULL (psfsdata in the scope of the current calling routine is used). If the key must survive after the end of the calling routine, a new key is allocated and ressource in $\ast$pclient are used

\begin{Desc}
\item[Parameters:]
\begin{description}
\item[{\em pkey}][OUT] computed key \item[{\em pfsdata}][IN] FSAL data to be used to compute the key \item[{\em pclient}][INOUT] if NULL, pfsdata is used to build the key (that stay local), if not pool\_\-key is used to allocate a new key \end{description}
\end{Desc}
\begin{Desc}
\item[Returns:]0 if keys if successfully build, 1 otherwise \end{Desc}


Definition at line 171 of file cache\_\-inode\_\-misc.c.\index{cache\_\-inode\_\-misc.c@{cache\_\-inode\_\-misc.c}!cache\_\-inode\_\-get\_\-attributes@{cache\_\-inode\_\-get\_\-attributes}}
\index{cache\_\-inode\_\-get\_\-attributes@{cache\_\-inode\_\-get\_\-attributes}!cache_inode_misc.c@{cache\_\-inode\_\-misc.c}}
\subsubsection[{cache\_\-inode\_\-get\_\-attributes}]{\setlength{\rightskip}{0pt plus 5cm}void cache\_\-inode\_\-get\_\-attributes (cache\_\-entry\_\-t $\ast$ {\em pentry}, \/  fsal\_\-attrib\_\-list\_\-t $\ast$ {\em pattr})}\label{cache__inode__misc_8c_8b74ff8be25748ce96f1d6258409d27d}


cache\_\-inode\_\-get\_\-attributes: gets the attributes cached in the entry.

Gets the attributes cached in the entry.

\begin{Desc}
\item[Parameters:]
\begin{description}
\item[{\em pentry}][IN] the entry to deal with. \item[{\em pattr}][OUT] the attributes for this entry.\end{description}
\end{Desc}
\begin{Desc}
\item[Returns:]nothing (void function). \end{Desc}


Definition at line 955 of file cache\_\-inode\_\-misc.c.\index{cache\_\-inode\_\-misc.c@{cache\_\-inode\_\-misc.c}!cache\_\-inode\_\-get\_\-fsal\_\-handle@{cache\_\-inode\_\-get\_\-fsal\_\-handle}}
\index{cache\_\-inode\_\-get\_\-fsal\_\-handle@{cache\_\-inode\_\-get\_\-fsal\_\-handle}!cache_inode_misc.c@{cache\_\-inode\_\-misc.c}}
\subsubsection[{cache\_\-inode\_\-get\_\-fsal\_\-handle}]{\setlength{\rightskip}{0pt plus 5cm}fsal\_\-handle\_\-t$\ast$ cache\_\-inode\_\-get\_\-fsal\_\-handle (cache\_\-entry\_\-t $\ast$ {\em pentry}, \/  cache\_\-inode\_\-status\_\-t $\ast$ {\em pstatus})}\label{cache__inode__misc_8c_625ee0acb63c4e7f1b342389c5dd7a0b}


cache\_\-inode\_\-get\_\-fsal\_\-handle: gets the FSAL handle from a pentry.

Gets the FSAL handle from a pentry. The entry should be lock BEFORE this call is done (no lock is managed in this function). All DIR\_\-BEGINNING and DIR\_\-CONTINUE involved in the same dir\_\-chain will return the same handle.

\begin{Desc}
\item[Parameters:]
\begin{description}
\item[{\em pentry}][IN] the input pentry. \item[{\em pstatus}][OUT] the status for the extraction (If not CACHE\_\-INODE\_\-SUCCESS, there is an error).\end{description}
\end{Desc}
\begin{Desc}
\item[Returns:]the result of the conversion. NULL shows an error. \end{Desc}


Definition at line 1103 of file cache\_\-inode\_\-misc.c.\index{cache\_\-inode\_\-misc.c@{cache\_\-inode\_\-misc.c}!cache\_\-inode\_\-kill\_\-entry@{cache\_\-inode\_\-kill\_\-entry}}
\index{cache\_\-inode\_\-kill\_\-entry@{cache\_\-inode\_\-kill\_\-entry}!cache_inode_misc.c@{cache\_\-inode\_\-misc.c}}
\subsubsection[{cache\_\-inode\_\-kill\_\-entry}]{\setlength{\rightskip}{0pt plus 5cm}cache\_\-inode\_\-status\_\-t cache\_\-inode\_\-kill\_\-entry (cache\_\-entry\_\-t $\ast$ {\em pentry}, \/  hash\_\-table\_\-t $\ast$ {\em ht}, \/  cache\_\-inode\_\-client\_\-t $\ast$ {\em pclient}, \/  cache\_\-inode\_\-status\_\-t $\ast$ {\em pstatus})}\label{cache__inode__misc_8c_ed59b090dbed68c7eca3a1b5514a11c5}


cache\_\-inode\_\-kill\_\-entry: force removing an entry from the cache\_\-inode. This is used in case of a 'stale' entry.

Force removing an entry from the cache\_\-inode. This is used in case of a 'stale' entry.

\begin{Desc}
\item[Parameters:]
\begin{description}
\item[{\em pentry}][IN] the input pentry (supposed to be staled). \item[{\em ht}][INOUT] the related hash table for the cache\_\-inode cache. \item[{\em pclient}][INOUT] related cache\_\-inode client. \item[{\em pstatus}][OUT] status for the operation.\end{description}
\end{Desc}
\begin{Desc}
\item[Returns:]CACHE\_\-INODE\_\-BAD\_\-TYPE if pentry is not related a REGULAR\_\-FILE or DIR\_\-BEGINNING \par
 

CACHE\_\-INODE\_\-SUCCESS if operation succeded. \end{Desc}


Definition at line 1463 of file cache\_\-inode\_\-misc.c.\index{cache\_\-inode\_\-misc.c@{cache\_\-inode\_\-misc.c}!cache\_\-inode\_\-mutex\_\-destroy@{cache\_\-inode\_\-mutex\_\-destroy}}
\index{cache\_\-inode\_\-mutex\_\-destroy@{cache\_\-inode\_\-mutex\_\-destroy}!cache_inode_misc.c@{cache\_\-inode\_\-misc.c}}
\subsubsection[{cache\_\-inode\_\-mutex\_\-destroy}]{\setlength{\rightskip}{0pt plus 5cm}void cache\_\-inode\_\-mutex\_\-destroy (cache\_\-entry\_\-t $\ast$ {\em pentry})}\label{cache__inode__misc_8c_5c72fb619678af886176784ba7250b3c}


cache\_\-inode\_\-mutex\_\-destroy: destroys the pthread\_\-mutex associated with a pentry when it is put back to the spool.

Destroys the pthread\_\-mutex associated with a pentry when it is put back to the spool

\begin{Desc}
\item[Parameters:]
\begin{description}
\item[{\em pentry}][INOUT] the input pentry.\end{description}
\end{Desc}
\begin{Desc}
\item[Returns:]nothing (void function) \end{Desc}


Definition at line 1212 of file cache\_\-inode\_\-misc.c.\index{cache\_\-inode\_\-misc.c@{cache\_\-inode\_\-misc.c}!cache\_\-inode\_\-new\_\-entry@{cache\_\-inode\_\-new\_\-entry}}
\index{cache\_\-inode\_\-new\_\-entry@{cache\_\-inode\_\-new\_\-entry}!cache_inode_misc.c@{cache\_\-inode\_\-misc.c}}
\subsubsection[{cache\_\-inode\_\-new\_\-entry}]{\setlength{\rightskip}{0pt plus 5cm}cache\_\-entry\_\-t$\ast$ cache\_\-inode\_\-new\_\-entry (cache\_\-inode\_\-fsal\_\-data\_\-t $\ast$ {\em pfsdata}, \/  fsal\_\-attrib\_\-list\_\-t $\ast$ {\em pfsal\_\-attr}, \/  cache\_\-inode\_\-file\_\-type\_\-t {\em type}, \/  cache\_\-inode\_\-create\_\-arg\_\-t $\ast$ {\em pcreate\_\-arg}, \/  cache\_\-entry\_\-t $\ast$ {\em pentry\_\-dir\_\-prev}, \/  hash\_\-table\_\-t $\ast$ {\em ht}, \/  cache\_\-inode\_\-client\_\-t $\ast$ {\em pclient}, \/  fsal\_\-op\_\-context\_\-t $\ast$ {\em pcontext}, \/  unsigned int {\em create\_\-flag}, \/  cache\_\-inode\_\-status\_\-t $\ast$ {\em pstatus})}\label{cache__inode__misc_8c_8392006e0fb4aea7143cfd950e91a74a}


cache\_\-inode\_\-new\_\-entry: adds a new entry to the cache\_\-inode.

adds a new entry to the cache\_\-inode. These function os used to allocate entries of any kind. Some parameter are meaningless for some types or used for others.

\begin{Desc}
\item[Parameters:]
\begin{description}
\item[{\em pfsdata}][IN] FSAL data for the entry to be created (used to build the key) \item[{\em pfsal\_\-attr}][in] attributes for the entry (unused if value == NULL). Used for caching. \item[{\em type}][IN] type of the entry to be created. \item[{\em link\_\-content}][IN] if type == SYMBOLIC\_\-LINK, this is the content of the link. Unused otherwise \item[{\em pentry\_\-dir\_\-prev}][IN] if type == DIR\_\-CONTINUE, this is the previous entry in the dir\_\-chain. Unused otherwise. \item[{\em ht}][INOUT] hash table used for the cache. \item[{\em pclient}][INOUT]ressource allocated by the client for the nfs management. \item[{\em pcontext}][IN] FSAL credentials for the operation. \item[{\em create\_\-flag}][IN] a flag which shows if the entry is newly created or not \item[{\em pstatus}][OUT] returned status.\end{description}
\end{Desc}
\begin{Desc}
\item[Returns:]the same as $\ast$pstatus \end{Desc}


Definition at line 255 of file cache\_\-inode\_\-misc.c.\index{cache\_\-inode\_\-misc.c@{cache\_\-inode\_\-misc.c}!cache\_\-inode\_\-print\_\-dir@{cache\_\-inode\_\-print\_\-dir}}
\index{cache\_\-inode\_\-print\_\-dir@{cache\_\-inode\_\-print\_\-dir}!cache_inode_misc.c@{cache\_\-inode\_\-misc.c}}
\subsubsection[{cache\_\-inode\_\-print\_\-dir}]{\setlength{\rightskip}{0pt plus 5cm}void cache\_\-inode\_\-print\_\-dir (cache\_\-entry\_\-t $\ast$ {\em cache\_\-entry\_\-root})}\label{cache__inode__misc_8c_b405ebfdd83c07441a5f617862fd789a}


cache\_\-inode\_\-print\_\-dir: prints the content of a pentry that is a directory segment.

Prints the content of a pentry that is a DIR\_\-BEGINNING or a DIR\_\-CONTINUE. /!$\backslash$ This function is provided for debugging purpose only, it makes no sanity check on the arguments.

\begin{Desc}
\item[Parameters:]
\begin{description}
\item[{\em pentry}][IN] the input pentry.\end{description}
\end{Desc}
\begin{Desc}
\item[Returns:]nothing (void function) \end{Desc}


Definition at line 1230 of file cache\_\-inode\_\-misc.c.\index{cache\_\-inode\_\-misc.c@{cache\_\-inode\_\-misc.c}!cache\_\-inode\_\-release\_\-fsaldata\_\-key@{cache\_\-inode\_\-release\_\-fsaldata\_\-key}}
\index{cache\_\-inode\_\-release\_\-fsaldata\_\-key@{cache\_\-inode\_\-release\_\-fsaldata\_\-key}!cache_inode_misc.c@{cache\_\-inode\_\-misc.c}}
\subsubsection[{cache\_\-inode\_\-release\_\-fsaldata\_\-key}]{\setlength{\rightskip}{0pt plus 5cm}void cache\_\-inode\_\-release\_\-fsaldata\_\-key (hash\_\-buffer\_\-t $\ast$ {\em pkey}, \/  cache\_\-inode\_\-client\_\-t $\ast$ {\em pclient})}\label{cache__inode__misc_8c_85615042ce506cf412216455b73ada68}


cache\_\-inode\_\-release\_\-fsaldata\_\-key: release a fsal key used to access the cache inode

Release a fsal key used to access the cache inode.

\begin{Desc}
\item[Parameters:]
\begin{description}
\item[{\em pkey}][IN] pointer to the key to be freed \item[{\em pclient}][INOUT] ressource allocated by the client for the nfs management.\end{description}
\end{Desc}
\begin{Desc}
\item[Returns:]nothing (void function) \end{Desc}


Definition at line 225 of file cache\_\-inode\_\-misc.c.\index{cache\_\-inode\_\-misc.c@{cache\_\-inode\_\-misc.c}!cache\_\-inode\_\-reload\_\-content@{cache\_\-inode\_\-reload\_\-content}}
\index{cache\_\-inode\_\-reload\_\-content@{cache\_\-inode\_\-reload\_\-content}!cache_inode_misc.c@{cache\_\-inode\_\-misc.c}}
\subsubsection[{cache\_\-inode\_\-reload\_\-content}]{\setlength{\rightskip}{0pt plus 5cm}cache\_\-inode\_\-status\_\-t cache\_\-inode\_\-reload\_\-content (char $\ast$ {\em path}, \/  cache\_\-entry\_\-t $\ast$ {\em pentry})}\label{cache__inode__misc_8c_e0a8ae792a80e2143f3ae5cc1bffe95d}


cache\_\-inode\_\-reload\_\-content: reloads the content of a pentry from a local file (used File Content crash recovery).

Reloeads the content of a pentry from a local file (used File Content crash recovery).

\begin{Desc}
\item[Parameters:]
\begin{description}
\item[{\em path}][IN] the full path to the file that will contain the metadata. \item[{\em pentry}][IN] the input pentry.\end{description}
\end{Desc}
\begin{Desc}
\item[Returns:]CACHE\_\-INODE\_\-BAD\_\-TYPE if pentry is not related to a REGULAR\_\-FILE \par
 

CACHE\_\-INODE\_\-SUCCESS if operation succeded. \end{Desc}


Definition at line 1327 of file cache\_\-inode\_\-misc.c.\index{cache\_\-inode\_\-misc.c@{cache\_\-inode\_\-misc.c}!cache\_\-inode\_\-set\_\-attributes@{cache\_\-inode\_\-set\_\-attributes}}
\index{cache\_\-inode\_\-set\_\-attributes@{cache\_\-inode\_\-set\_\-attributes}!cache_inode_misc.c@{cache\_\-inode\_\-misc.c}}
\subsubsection[{cache\_\-inode\_\-set\_\-attributes}]{\setlength{\rightskip}{0pt plus 5cm}void cache\_\-inode\_\-set\_\-attributes (cache\_\-entry\_\-t $\ast$ {\em pentry}, \/  fsal\_\-attrib\_\-list\_\-t $\ast$ {\em pattr})}\label{cache__inode__misc_8c_7da7b52764bdb6c1a6d60a2a979322aa}


cache\_\-inode\_\-set\_\-attributes: sets the attributes cached in the entry.

Sets the attributes cached in the entry.

\begin{Desc}
\item[Parameters:]
\begin{description}
\item[{\em pentry}][OUT] the entry to deal with. \item[{\em pattr}][IN] the attributes to be set for this entry.\end{description}
\end{Desc}
\begin{Desc}
\item[Returns:]nothing (void function). \end{Desc}


Definition at line 1003 of file cache\_\-inode\_\-misc.c.\index{cache\_\-inode\_\-misc.c@{cache\_\-inode\_\-misc.c}!cache\_\-inode\_\-type\_\-are\_\-rename\_\-compatible@{cache\_\-inode\_\-type\_\-are\_\-rename\_\-compatible}}
\index{cache\_\-inode\_\-type\_\-are\_\-rename\_\-compatible@{cache\_\-inode\_\-type\_\-are\_\-rename\_\-compatible}!cache_inode_misc.c@{cache\_\-inode\_\-misc.c}}
\subsubsection[{cache\_\-inode\_\-type\_\-are\_\-rename\_\-compatible}]{\setlength{\rightskip}{0pt plus 5cm}int cache\_\-inode\_\-type\_\-are\_\-rename\_\-compatible (cache\_\-entry\_\-t $\ast$ {\em pentry\_\-src}, \/  cache\_\-entry\_\-t $\ast$ {\em pentry\_\-dest})}\label{cache__inode__misc_8c_276225d5d7db59f532f94ca6cf32ab53}


cache\_\-inode\_\-type\_\-are\_\-rename\_\-compatible: test if an existing entry could be scrtached during a rename.

test if an existing entry could be scrtached during a rename. No mutext management.

\begin{Desc}
\item[Parameters:]
\begin{description}
\item[{\em pentry\_\-src}][IN] the source pentry (the one to be renamed) \item[{\em pentry\_\-dest}][IN] the dest pentry (the one to be scratched during the rename)\end{description}
\end{Desc}
\begin{Desc}
\item[Returns:]TRUE if rename if allowed (types are compatible), FALSE if not. \end{Desc}


Definition at line 1169 of file cache\_\-inode\_\-misc.c.\index{cache\_\-inode\_\-misc.c@{cache\_\-inode\_\-misc.c}!cache\_\-inode\_\-valid@{cache\_\-inode\_\-valid}}
\index{cache\_\-inode\_\-valid@{cache\_\-inode\_\-valid}!cache_inode_misc.c@{cache\_\-inode\_\-misc.c}}
\subsubsection[{cache\_\-inode\_\-valid}]{\setlength{\rightskip}{0pt plus 5cm}cache\_\-inode\_\-status\_\-t cache\_\-inode\_\-valid (cache\_\-entry\_\-t $\ast$ {\em pentry}, \/  cache\_\-inode\_\-op\_\-t {\em op}, \/  cache\_\-inode\_\-client\_\-t $\ast$ {\em pclient})}\label{cache__inode__misc_8c_a34ae3526aa5eb8bc0f96e54a3fe8861}


cache\_\-inode\_\-valid: validates an entry to update its garbagge status.

Validates an error to update its garbagge status. Entry is supposed to be locked when this function is called !!

\begin{Desc}
\item[Parameters:]
\begin{description}
\item[{\em pentry}][INOUT] entry to be validated. \item[{\em op}][IN] can be set to CACHE\_\-INODE\_\-OP\_\-GET or CACHE\_\-INODE\_\-OP\_\-SET to show the type of operation done. \item[{\em pclient}][INOUT] ressource allocated by the client for the nfs management.\end{description}
\end{Desc}
\begin{Desc}
\item[Returns:]CACHE\_\-INODE\_\-SUCCESS if successful \par
 

CACHE\_\-INODE\_\-LRU\_\-ERROR if an errorr occured in LRU management. \end{Desc}


Definition at line 811 of file cache\_\-inode\_\-misc.c.