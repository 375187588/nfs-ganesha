\section{cache\_\-inode\_\-lookupp.c File Reference}
\label{cache__inode__lookupp_8c}\index{cache_inode_lookupp.c@{cache\_\-inode\_\-lookupp.c}}
Perform lookup through the cache to get the parent entry for a directory. 

{\tt \#include \char`\"{}LRU\_\-List.h\char`\"{}}\par
{\tt \#include \char`\"{}log\_\-functions.h\char`\"{}}\par
{\tt \#include \char`\"{}Hash\-Data.h\char`\"{}}\par
{\tt \#include \char`\"{}Hash\-Table.h\char`\"{}}\par
{\tt \#include \char`\"{}fsal.h\char`\"{}}\par
{\tt \#include \char`\"{}cache\_\-inode.h\char`\"{}}\par
{\tt \#include \char`\"{}stuff\_\-alloc.h\char`\"{}}\par
{\tt \#include $<$unistd.h$>$}\par
{\tt \#include $<$sys/types.h$>$}\par
{\tt \#include $<$sys/param.h$>$}\par
{\tt \#include $<$time.h$>$}\par
{\tt \#include $<$pthread.h$>$}\par
\subsection*{Functions}
\begin{CompactItemize}
\item 
cache\_\-entry\_\-t $\ast$ {\bf cache\_\-inode\_\-lookupp\_\-sw} (cache\_\-entry\_\-t $\ast$pentry, hash\_\-table\_\-t $\ast$ht, cache\_\-inode\_\-client\_\-t $\ast$pclient, fsal\_\-op\_\-context\_\-t $\ast$pcontext, cache\_\-inode\_\-status\_\-t $\ast$pstatus, int use\_\-mutex)
\item 
cache\_\-entry\_\-t $\ast$ {\bf cache\_\-inode\_\-lookupp} (cache\_\-entry\_\-t $\ast$pentry, hash\_\-table\_\-t $\ast$ht, cache\_\-inode\_\-client\_\-t $\ast$pclient, fsal\_\-op\_\-context\_\-t $\ast$pcontext, cache\_\-inode\_\-status\_\-t $\ast$pstatus)
\item 
cache\_\-entry\_\-t $\ast$ {\bf cache\_\-inode\_\-lookupp\_\-no\_\-mutex} (cache\_\-entry\_\-t $\ast$pentry, hash\_\-table\_\-t $\ast$ht, cache\_\-inode\_\-client\_\-t $\ast$pclient, fsal\_\-op\_\-context\_\-t $\ast$pcontext, cache\_\-inode\_\-status\_\-t $\ast$pstatus)
\end{CompactItemize}


\subsection{Detailed Description}
Perform lookup through the cache to get the parent entry for a directory. 

\begin{Desc}
\item[Author:]\begin{Desc}
\item[Author]deniel \end{Desc}
\end{Desc}
\begin{Desc}
\item[Date:]\begin{Desc}
\item[Date]2005/11/28 17:02:26 \end{Desc}
\end{Desc}
\begin{Desc}
\item[Version:]\begin{Desc}
\item[Revision]1.5 \end{Desc}
\end{Desc}
{\bf cache\_\-inode\_\-lookupp.c}{\rm (p.\,\pageref{cache__inode__lookupp_8c})} : Perform lookup through the cache to get the parent entry for a directory.

Definition in file {\bf cache\_\-inode\_\-lookupp.c}.

\subsection{Function Documentation}
\index{cache_inode_lookupp.c@{cache\_\-inode\_\-lookupp.c}!cache_inode_lookupp@{cache\_\-inode\_\-lookupp}}
\index{cache_inode_lookupp@{cache\_\-inode\_\-lookupp}!cache_inode_lookupp.c@{cache\_\-inode\_\-lookupp.c}}
\subsubsection{\setlength{\rightskip}{0pt plus 5cm}cache\_\-entry\_\-t$\ast$ cache\_\-inode\_\-lookupp (cache\_\-entry\_\-t $\ast$ {\em pentry}, hash\_\-table\_\-t $\ast$ {\em ht}, cache\_\-inode\_\-client\_\-t $\ast$ {\em pclient}, fsal\_\-op\_\-context\_\-t $\ast$ {\em pcontext}, cache\_\-inode\_\-status\_\-t $\ast$ {\em pstatus})}\label{cache__inode__lookupp_8c_a1}


cache\_\-inode\_\-lookupp: looks up (and caches) the parent directory for a directory.

Looks up (and caches) the parent directory for a directory.

\begin{Desc}
\item[Parameters:]
\begin{description}
\item[{\em pentry}][IN] entry whose parent is to be obtained. \item[{\em ht}][IN] hash table used for the cache, unused in this call. \item[{\em pclient}][INOUT] ressource allocated by the client for the nfs management. \item[{\em pcontext}][IN] FSAL credentials \item[{\em pstatus}][OUT] returned status.\end{description}
\end{Desc}
\begin{Desc}
\item[Returns:]CACHE\_\-INODE\_\-SUCCESS if operation is a success \par
 

CACHE\_\-INODE\_\-LRU\_\-ERROR if allocation error occured when validating the entry \end{Desc}


Definition at line 251 of file cache\_\-inode\_\-lookupp.c.

References cache\_\-inode\_\-lookupp\_\-sw().\index{cache_inode_lookupp.c@{cache\_\-inode\_\-lookupp.c}!cache_inode_lookupp_no_mutex@{cache\_\-inode\_\-lookupp\_\-no\_\-mutex}}
\index{cache_inode_lookupp_no_mutex@{cache\_\-inode\_\-lookupp\_\-no\_\-mutex}!cache_inode_lookupp.c@{cache\_\-inode\_\-lookupp.c}}
\subsubsection{\setlength{\rightskip}{0pt plus 5cm}cache\_\-entry\_\-t$\ast$ cache\_\-inode\_\-lookupp\_\-no\_\-mutex (cache\_\-entry\_\-t $\ast$ {\em pentry}, hash\_\-table\_\-t $\ast$ {\em ht}, cache\_\-inode\_\-client\_\-t $\ast$ {\em pclient}, fsal\_\-op\_\-context\_\-t $\ast$ {\em pcontext}, cache\_\-inode\_\-status\_\-t $\ast$ {\em pstatus})}\label{cache__inode__lookupp_8c_a2}


cache\_\-inode\_\-lookupp\_\-no\_\-mutex: looks up (and caches) the parent directory for a directory. No mutex management

Looks up (and caches) the parent directory for a directory.

\begin{Desc}
\item[Parameters:]
\begin{description}
\item[{\em pentry}][IN] entry whose parent is to be obtained. \item[{\em ht}][IN] hash table used for the cache, unused in this call. \item[{\em pclient}][INOUT] ressource allocated by the client for the nfs management. \item[{\em pcontext}][IN] FSAL credentials \item[{\em pstatus}][OUT] returned status.\end{description}
\end{Desc}
\begin{Desc}
\item[Returns:]CACHE\_\-INODE\_\-SUCCESS if operation is a success \par
 

CACHE\_\-INODE\_\-LRU\_\-ERROR if allocation error occured when validating the entry \end{Desc}


Definition at line 281 of file cache\_\-inode\_\-lookupp.c.

References cache\_\-inode\_\-lookupp\_\-sw().

Referenced by cache\_\-inode\_\-lookup\_\-sw().\index{cache_inode_lookupp.c@{cache\_\-inode\_\-lookupp.c}!cache_inode_lookupp_sw@{cache\_\-inode\_\-lookupp\_\-sw}}
\index{cache_inode_lookupp_sw@{cache\_\-inode\_\-lookupp\_\-sw}!cache_inode_lookupp.c@{cache\_\-inode\_\-lookupp.c}}
\subsubsection{\setlength{\rightskip}{0pt plus 5cm}cache\_\-entry\_\-t$\ast$ cache\_\-inode\_\-lookupp\_\-sw (cache\_\-entry\_\-t $\ast$ {\em pentry}, hash\_\-table\_\-t $\ast$ {\em ht}, cache\_\-inode\_\-client\_\-t $\ast$ {\em pclient}, fsal\_\-op\_\-context\_\-t $\ast$ {\em pcontext}, cache\_\-inode\_\-status\_\-t $\ast$ {\em pstatus}, int {\em use\_\-mutex})}\label{cache__inode__lookupp_8c_a0}


cache\_\-inode\_\-lookupp\_\-sw: looks up (and caches) the parent directory for a directory. A switches tells is mutex are use.

Looks up (and caches) the parent directory for a directory.

\begin{Desc}
\item[Parameters:]
\begin{description}
\item[{\em pentry}][IN] entry whose parent is to be obtained. \item[{\em ht}][IN] hash table used for the cache, unused in this call. \item[{\em pclient}][INOUT] ressource allocated by the client for the nfs management. \item[{\em pcontext}][IN] FSAL credentials \item[{\em pstatus}][OUT] returned status. \item[{\em use\_\-mutex}][IN] if TRUE mutex are use, not otherwise.\end{description}
\end{Desc}
\begin{Desc}
\item[Returns:]CACHE\_\-INODE\_\-SUCCESS if operation is a success \par
 

CACHE\_\-INODE\_\-LRU\_\-ERROR if allocation error occured when validating the entry \end{Desc}


Definition at line 122 of file cache\_\-inode\_\-lookupp.c.

References cache\_\-inode\_\-error\_\-convert(), cache\_\-inode\_\-get(), cache\_\-inode\_\-kill\_\-entry(), cache\_\-inode\_\-renew\_\-entry(), and cache\_\-inode\_\-valid().

Referenced by cache\_\-inode\_\-lookupp(), and cache\_\-inode\_\-lookupp\_\-no\_\-mutex().