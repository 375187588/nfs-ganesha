\section{cache\_\-inode\_\-truncate.c File Reference}
\label{cache__inode__truncate_8c}\index{cache_inode_truncate.c@{cache\_\-inode\_\-truncate.c}}
Truncates a regular file. 

{\tt \#include \char`\"{}fsal.h\char`\"{}}\par
{\tt \#include \char`\"{}LRU\_\-List.h\char`\"{}}\par
{\tt \#include \char`\"{}log\_\-functions.h\char`\"{}}\par
{\tt \#include \char`\"{}Hash\-Data.h\char`\"{}}\par
{\tt \#include \char`\"{}Hash\-Table.h\char`\"{}}\par
{\tt \#include \char`\"{}cache\_\-inode.h\char`\"{}}\par
{\tt \#include \char`\"{}cache\_\-content.h\char`\"{}}\par
{\tt \#include $<$unistd.h$>$}\par
{\tt \#include $<$sys/types.h$>$}\par
{\tt \#include $<$sys/param.h$>$}\par
{\tt \#include $<$time.h$>$}\par
{\tt \#include $<$pthread.h$>$}\par
\subsection*{Functions}
\begin{CompactItemize}
\item 
cache\_\-inode\_\-status\_\-t {\bf cache\_\-inode\_\-truncate\_\-sw} (cache\_\-entry\_\-t $\ast$pentry, fsal\_\-size\_\-t length, fsal\_\-attrib\_\-list\_\-t $\ast$pattr, hash\_\-table\_\-t $\ast$ht, cache\_\-inode\_\-client\_\-t $\ast$pclient, fsal\_\-op\_\-context\_\-t $\ast$pcontext, cache\_\-inode\_\-status\_\-t $\ast$pstatus, int use\_\-mutex)
\item 
cache\_\-inode\_\-status\_\-t {\bf cache\_\-inode\_\-truncate\_\-no\_\-mutex} (cache\_\-entry\_\-t $\ast$pentry, fsal\_\-size\_\-t length, fsal\_\-attrib\_\-list\_\-t $\ast$pattr, hash\_\-table\_\-t $\ast$ht, cache\_\-inode\_\-client\_\-t $\ast$pclient, fsal\_\-op\_\-context\_\-t $\ast$pcontext, cache\_\-inode\_\-status\_\-t $\ast$pstatus)
\item 
cache\_\-inode\_\-status\_\-t {\bf cache\_\-inode\_\-truncate} (cache\_\-entry\_\-t $\ast$pentry, fsal\_\-size\_\-t length, fsal\_\-attrib\_\-list\_\-t $\ast$pattr, hash\_\-table\_\-t $\ast$ht, cache\_\-inode\_\-client\_\-t $\ast$pclient, fsal\_\-op\_\-context\_\-t $\ast$pcontext, cache\_\-inode\_\-status\_\-t $\ast$pstatus)
\end{CompactItemize}


\subsection{Detailed Description}
Truncates a regular file. 

\begin{Desc}
\item[Author:]\begin{Desc}
\item[Author]deniel \end{Desc}
\end{Desc}
\begin{Desc}
\item[Date:]\begin{Desc}
\item[Date]2005/11/28 17:02:28 \end{Desc}
\end{Desc}
\begin{Desc}
\item[Version:]\begin{Desc}
\item[Revision]1.19 \end{Desc}
\end{Desc}
{\bf cache\_\-inode\_\-truncate.c}{\rm (p.\,\pageref{cache__inode__truncate_8c})} : Truncates a regular file.

Definition in file {\bf cache\_\-inode\_\-truncate.c}.

\subsection{Function Documentation}
\index{cache_inode_truncate.c@{cache\_\-inode\_\-truncate.c}!cache_inode_truncate@{cache\_\-inode\_\-truncate}}
\index{cache_inode_truncate@{cache\_\-inode\_\-truncate}!cache_inode_truncate.c@{cache\_\-inode\_\-truncate.c}}
\subsubsection{\setlength{\rightskip}{0pt plus 5cm}cache\_\-inode\_\-status\_\-t cache\_\-inode\_\-truncate (cache\_\-entry\_\-t $\ast$ {\em pentry}, fsal\_\-size\_\-t {\em length}, fsal\_\-attrib\_\-list\_\-t $\ast$ {\em pattr}, hash\_\-table\_\-t $\ast$ {\em ht}, cache\_\-inode\_\-client\_\-t $\ast$ {\em pclient}, fsal\_\-op\_\-context\_\-t $\ast$ {\em pcontext}, cache\_\-inode\_\-status\_\-t $\ast$ {\em pstatus})}\label{cache__inode__truncate_8c_a2}


cache\_\-inode\_\-truncate: truncates a regular file specified by its cache entry.

Truncates a regular file specified by its cache entry.

\begin{Desc}
\item[Parameters:]
\begin{description}
\item[{\em pentry}][INOUT] entry pointer for the fs object to be truncated. \item[{\em length}][IN] wanted length for the file. \item[{\em pattr}][OUT] attrtibutes for the file after the operation. \item[{\em ht}][INOUT] hash table used for the cache. Unused in this call (kept for protototype's homogeneity). \item[{\em pclient}][INOUT] ressource allocated by the client for the nfs management. \item[{\em pcontext}][IN] FSAL credentials \item[{\em pstatus}][OUT] returned status.\end{description}
\end{Desc}
\begin{Desc}
\item[Returns:]CACHE\_\-INODE\_\-SUCCESS if operation is a success \par
 

CACHE\_\-INODE\_\-LRU\_\-ERROR if allocation error occured when validating the entry \end{Desc}


Definition at line 298 of file cache\_\-inode\_\-truncate.c.

References cache\_\-inode\_\-truncate\_\-sw().\index{cache_inode_truncate.c@{cache\_\-inode\_\-truncate.c}!cache_inode_truncate_no_mutex@{cache\_\-inode\_\-truncate\_\-no\_\-mutex}}
\index{cache_inode_truncate_no_mutex@{cache\_\-inode\_\-truncate\_\-no\_\-mutex}!cache_inode_truncate.c@{cache\_\-inode\_\-truncate.c}}
\subsubsection{\setlength{\rightskip}{0pt plus 5cm}cache\_\-inode\_\-status\_\-t cache\_\-inode\_\-truncate\_\-no\_\-mutex (cache\_\-entry\_\-t $\ast$ {\em pentry}, fsal\_\-size\_\-t {\em length}, fsal\_\-attrib\_\-list\_\-t $\ast$ {\em pattr}, hash\_\-table\_\-t $\ast$ {\em ht}, cache\_\-inode\_\-client\_\-t $\ast$ {\em pclient}, fsal\_\-op\_\-context\_\-t $\ast$ {\em pcontext}, cache\_\-inode\_\-status\_\-t $\ast$ {\em pstatus})}\label{cache__inode__truncate_8c_a1}


cache\_\-inode\_\-truncate\_\-no\_\-mutex: truncates a regular file specified by its cache entry (no mutex management).

Truncates a regular file specified by its cache entry.

\begin{Desc}
\item[Parameters:]
\begin{description}
\item[{\em pentry}][INOUT] entry pointer for the fs object to be truncated. \item[{\em length}][IN] wanted length for the file. \item[{\em pattr}][OUT] attrtibutes for the file after the operation. \item[{\em ht}][INOUT] hash table used for the cache. Unused in this call (kept for protototype's homogeneity). \item[{\em pclient}][INOUT] ressource allocated by the client for the nfs management. \item[{\em pcontext}][IN] FSAL credentials \item[{\em pstatus}][OUT] returned status.\end{description}
\end{Desc}
\begin{Desc}
\item[Returns:]CACHE\_\-INODE\_\-SUCCESS if operation is a success \par
 

CACHE\_\-INODE\_\-LRU\_\-ERROR if allocation error occured when validating the entry \end{Desc}


Definition at line 262 of file cache\_\-inode\_\-truncate.c.

References cache\_\-inode\_\-truncate\_\-sw().\index{cache_inode_truncate.c@{cache\_\-inode\_\-truncate.c}!cache_inode_truncate_sw@{cache\_\-inode\_\-truncate\_\-sw}}
\index{cache_inode_truncate_sw@{cache\_\-inode\_\-truncate\_\-sw}!cache_inode_truncate.c@{cache\_\-inode\_\-truncate.c}}
\subsubsection{\setlength{\rightskip}{0pt plus 5cm}cache\_\-inode\_\-status\_\-t cache\_\-inode\_\-truncate\_\-sw (cache\_\-entry\_\-t $\ast$ {\em pentry}, fsal\_\-size\_\-t {\em length}, fsal\_\-attrib\_\-list\_\-t $\ast$ {\em pattr}, hash\_\-table\_\-t $\ast$ {\em ht}, cache\_\-inode\_\-client\_\-t $\ast$ {\em pclient}, fsal\_\-op\_\-context\_\-t $\ast$ {\em pcontext}, cache\_\-inode\_\-status\_\-t $\ast$ {\em pstatus}, int {\em use\_\-mutex})}\label{cache__inode__truncate_8c_a0}


cache\_\-inode\_\-truncate\_\-sw: truncates a regular file specified by its cache entry.

Truncates a regular file specified by its cache entry.

\begin{Desc}
\item[Parameters:]
\begin{description}
\item[{\em pentry}][INOUT] entry pointer for the fs object to be truncated. \item[{\em length}][IN] wanted length for the file. \item[{\em pattr}][OUT] attrtibutes for the file after the operation. \item[{\em ht}][INOUT] hash table used for the cache. Unused in this call (kept for protototype's homogeneity). \item[{\em pclient}][INOUT] ressource allocated by the client for the nfs management. \item[{\em pcontext}][IN] FSAL credentials \item[{\em pstatus}][OUT] returned status. \item[{\em use\_\-mutex}][IN] if TRUE, mutex management is done, not if equal to FALSE.\end{description}
\end{Desc}
\begin{Desc}
\item[Returns:]CACHE\_\-INODE\_\-SUCCESS if operation is a success \par
 

CACHE\_\-INODE\_\-LRU\_\-ERROR if allocation error occured when validating the entry \end{Desc}


Definition at line 126 of file cache\_\-inode\_\-truncate.c.

References cache\_\-inode\_\-error\_\-convert(), cache\_\-inode\_\-kill\_\-entry(), and cache\_\-inode\_\-valid().

Referenced by cache\_\-inode\_\-truncate(), and cache\_\-inode\_\-truncate\_\-no\_\-mutex().