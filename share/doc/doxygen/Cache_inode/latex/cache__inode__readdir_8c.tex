\section{cache\_\-inode\_\-readdir.c File Reference}
\label{cache__inode__readdir_8c}\index{cache_inode_readdir.c@{cache\_\-inode\_\-readdir.c}}
Reads the content of a directory. 

{\tt \#include \char`\"{}LRU\_\-List.h\char`\"{}}\par
{\tt \#include \char`\"{}log\_\-functions.h\char`\"{}}\par
{\tt \#include \char`\"{}Hash\-Data.h\char`\"{}}\par
{\tt \#include \char`\"{}Hash\-Table.h\char`\"{}}\par
{\tt \#include \char`\"{}stuff\_\-alloc.h\char`\"{}}\par
{\tt \#include \char`\"{}fsal.h\char`\"{}}\par
{\tt \#include \char`\"{}cache\_\-inode.h\char`\"{}}\par
{\tt \#include $<$unistd.h$>$}\par
{\tt \#include $<$sys/types.h$>$}\par
{\tt \#include $<$sys/param.h$>$}\par
{\tt \#include $<$time.h$>$}\par
{\tt \#include $<$pthread.h$>$}\par
\subsection*{Functions}
\begin{CompactItemize}
\item 
cache\_\-entry\_\-t $\ast$ {\bf cache\_\-inode\_\-operate\_\-cached\_\-dirent} (cache\_\-entry\_\-t $\ast$pentry\_\-parent, fsal\_\-name\_\-t $\ast$pname, fsal\_\-name\_\-t $\ast$newname, cache\_\-inode\_\-dirent\_\-op\_\-t dirent\_\-op, cache\_\-inode\_\-status\_\-t $\ast$pstatus)
\item 
cache\_\-inode\_\-status\_\-t {\bf cache\_\-inode\_\-add\_\-cached\_\-dirent} (cache\_\-entry\_\-t $\ast$pentry\_\-parent, fsal\_\-name\_\-t $\ast$pname, cache\_\-entry\_\-t $\ast$pentry\_\-added, cache\_\-entry\_\-t $\ast$$\ast$ppentry\_\-next, hash\_\-table\_\-t $\ast$ht, cache\_\-inode\_\-client\_\-t $\ast$pclient, fsal\_\-op\_\-context\_\-t $\ast$pcontext, cache\_\-inode\_\-status\_\-t $\ast$pstatus)
\item 
cache\_\-inode\_\-status\_\-t {\bf cache\_\-inode\_\-invalidate\_\-all\_\-cached\_\-dirent} (cache\_\-entry\_\-t $\ast$pentry\_\-dir, hash\_\-table\_\-t $\ast$ht, cache\_\-inode\_\-client\_\-t $\ast$pclient, cache\_\-inode\_\-status\_\-t $\ast$pstatus)
\item 
cache\_\-inode\_\-status\_\-t {\bf cache\_\-inode\_\-remove\_\-cached\_\-dirent} (cache\_\-entry\_\-t $\ast$pentry\_\-parent, fsal\_\-name\_\-t $\ast$pname, hash\_\-table\_\-t $\ast$ht, cache\_\-inode\_\-client\_\-t $\ast$pclient, cache\_\-inode\_\-status\_\-t $\ast$pstatus)
\item 
cache\_\-inode\_\-status\_\-t {\bf cache\_\-inode\_\-readdir\_\-populate} (cache\_\-entry\_\-t $\ast$pentry\_\-dir, hash\_\-table\_\-t $\ast$ht, cache\_\-inode\_\-client\_\-t $\ast$pclient, fsal\_\-op\_\-context\_\-t $\ast$pcontext, cache\_\-inode\_\-status\_\-t $\ast$pstatus)
\item 
cache\_\-inode\_\-status\_\-t {\bf cache\_\-inode\_\-readdir} (cache\_\-entry\_\-t $\ast$dir\_\-pentry, unsigned int cookie, unsigned int nbwanted, unsigned int $\ast$pnbfound, unsigned int $\ast$pend\_\-cookie, cache\_\-inode\_\-endofdir\_\-t $\ast$peod\_\-met, cache\_\-inode\_\-dir\_\-entry\_\-t $\ast$dirent\_\-array, unsigned int $\ast$cookie\_\-array, hash\_\-table\_\-t $\ast$ht, cache\_\-inode\_\-client\_\-t $\ast$pclient, fsal\_\-op\_\-context\_\-t $\ast$pcontext, cache\_\-inode\_\-status\_\-t $\ast$pstatus)
\end{CompactItemize}


\subsection{Detailed Description}
Reads the content of a directory. 

\begin{Desc}
\item[Author:]\begin{Desc}
\item[Author]deniel \end{Desc}
\end{Desc}
\begin{Desc}
\item[Date:]\begin{Desc}
\item[Date]2006/01/24 11:43:05 \end{Desc}
\end{Desc}
\begin{Desc}
\item[Version:]\begin{Desc}
\item[Revision]1.50 \end{Desc}
\end{Desc}
{\bf cache\_\-inode\_\-readdir.c}{\rm (p.\,\pageref{cache__inode__readdir_8c})} : Reads the content of a directory. Contains also the needed function for directory browsing.

Definition in file {\bf cache\_\-inode\_\-readdir.c}.

\subsection{Function Documentation}
\index{cache_inode_readdir.c@{cache\_\-inode\_\-readdir.c}!cache_inode_add_cached_dirent@{cache\_\-inode\_\-add\_\-cached\_\-dirent}}
\index{cache_inode_add_cached_dirent@{cache\_\-inode\_\-add\_\-cached\_\-dirent}!cache_inode_readdir.c@{cache\_\-inode\_\-readdir.c}}
\subsubsection{\setlength{\rightskip}{0pt plus 5cm}cache\_\-inode\_\-status\_\-t cache\_\-inode\_\-add\_\-cached\_\-dirent (cache\_\-entry\_\-t $\ast$ {\em pentry\_\-parent}, fsal\_\-name\_\-t $\ast$ {\em pname}, cache\_\-entry\_\-t $\ast$ {\em pentry\_\-added}, cache\_\-entry\_\-t $\ast$$\ast$ {\em ppentry\_\-next}, hash\_\-table\_\-t $\ast$ {\em ht}, cache\_\-inode\_\-client\_\-t $\ast$ {\em pclient}, fsal\_\-op\_\-context\_\-t $\ast$ {\em pcontext}, cache\_\-inode\_\-status\_\-t $\ast$ {\em pstatus})}\label{cache__inode__readdir_8c_a1}


cache\_\-inode\_\-add\_\-cached\_\-dirent: Adds a directory entry to a cached directory.

Adds a directory entry to a cached directory. This is use when creating a new entry through nfs and keep it to the cache. It also allocates and caches the entry. This function can be call iteratively, within a loop (like what is done in cache\_\-inode\_\-readdir\_\-populate). In this case, pentry\_\-parent should be set to the value returned in $\ast$pentry\_\-next. This function should never be used for managing a junction.

\begin{Desc}
\item[Parameters:]
\begin{description}
\item[{\em pentry\_\-parent}][INOUT] cache entry representing the directory to be managed. \item[{\em name}][IN] name of the entry to add. \item[{\em pentry\_\-added}][IN] the pentry added to the dirent array \item[{\em pentry\_\-next}][OUT] the next pentry to use for next call. \item[{\em ht}][IN] hash table used for the cache, unused in this call. \item[{\em pclient}][INOUT] ressource allocated by the client for the nfs management. \item[{\em pstatus}][OUT] returned status.\end{description}
\end{Desc}
\begin{Desc}
\item[Returns:]the DIR\_\-CONTINUE or DIR\_\-BEGINNING that contain this entry in its array\_\-dirent\par
 

NULL if failed, see $\ast$pstatus for error's meaning. \end{Desc}


Definition at line 343 of file cache\_\-inode\_\-readdir.c.

References cache\_\-inode\_\-new\_\-entry().

Referenced by cache\_\-inode\_\-create(), cache\_\-inode\_\-link(), cache\_\-inode\_\-lookup\_\-sw(), cache\_\-inode\_\-readdir\_\-populate(), and cache\_\-inode\_\-rename().\index{cache_inode_readdir.c@{cache\_\-inode\_\-readdir.c}!cache_inode_invalidate_all_cached_dirent@{cache\_\-inode\_\-invalidate\_\-all\_\-cached\_\-dirent}}
\index{cache_inode_invalidate_all_cached_dirent@{cache\_\-inode\_\-invalidate\_\-all\_\-cached\_\-dirent}!cache_inode_readdir.c@{cache\_\-inode\_\-readdir.c}}
\subsubsection{\setlength{\rightskip}{0pt plus 5cm}cache\_\-inode\_\-status\_\-t cache\_\-inode\_\-invalidate\_\-all\_\-cached\_\-dirent (cache\_\-entry\_\-t $\ast$ {\em pentry\_\-dir}, hash\_\-table\_\-t $\ast$ {\em ht}, cache\_\-inode\_\-client\_\-t $\ast$ {\em pclient}, cache\_\-inode\_\-status\_\-t $\ast$ {\em pstatus})}\label{cache__inode__readdir_8c_a2}




Definition at line 608 of file cache\_\-inode\_\-readdir.c.

Referenced by cache\_\-inode\_\-readdir\_\-populate().\index{cache_inode_readdir.c@{cache\_\-inode\_\-readdir.c}!cache_inode_operate_cached_dirent@{cache\_\-inode\_\-operate\_\-cached\_\-dirent}}
\index{cache_inode_operate_cached_dirent@{cache\_\-inode\_\-operate\_\-cached\_\-dirent}!cache_inode_readdir.c@{cache\_\-inode\_\-readdir.c}}
\subsubsection{\setlength{\rightskip}{0pt plus 5cm}cache\_\-entry\_\-t$\ast$ cache\_\-inode\_\-operate\_\-cached\_\-dirent (cache\_\-entry\_\-t $\ast$ {\em pentry\_\-parent}, fsal\_\-name\_\-t $\ast$ {\em pname}, fsal\_\-name\_\-t $\ast$ {\em newname}, cache\_\-inode\_\-dirent\_\-op\_\-t {\em dirent\_\-op}, cache\_\-inode\_\-status\_\-t $\ast$ {\em pstatus})}\label{cache__inode__readdir_8c_a0}


cache\_\-inode\_\-operate\_\-cached\_\-dirent: locates a dirent in the cached dirent, and perform an operation on it.

Looks up for an dirent in the cached dirent. Thus function searches only in the entries listed in the dir\_\-entries array. Some entries may be missing but existing and not be cached (it no readdir was never performed on the entry for example. This function provides a way to operate on the dirent.

\begin{Desc}
\item[Parameters:]
\begin{description}
\item[{\em pentry\_\-parent}][IN] directory entry to be looked. \item[{\em name}][IN] name for the searched entry. \item[{\em newname}][IN] newname if function is used to rename a dirent \item[{\em dirent\_\-op}][IN] operation (ADD, LOOKUP or REMOVE) to do on the dirent if found.  [OUT] returned status.\end{description}
\end{Desc}
\begin{Desc}
\item[Returns:]the found entry if its exists and NULL if it is not in the dirent arrays. REMOVE always returns NULL. \end{Desc}


Definition at line 123 of file cache\_\-inode\_\-readdir.c.

References cache\_\-inode\_\-error\_\-convert().

Referenced by cache\_\-inode\_\-remove\_\-cached\_\-dirent(), and cache\_\-inode\_\-rename\_\-cached\_\-dirent().\index{cache_inode_readdir.c@{cache\_\-inode\_\-readdir.c}!cache_inode_readdir@{cache\_\-inode\_\-readdir}}
\index{cache_inode_readdir@{cache\_\-inode\_\-readdir}!cache_inode_readdir.c@{cache\_\-inode\_\-readdir.c}}
\subsubsection{\setlength{\rightskip}{0pt plus 5cm}cache\_\-inode\_\-status\_\-t cache\_\-inode\_\-readdir (cache\_\-entry\_\-t $\ast$ {\em dir\_\-pentry}, unsigned int {\em cookie}, unsigned int {\em nbwanted}, unsigned int $\ast$ {\em pnbfound}, unsigned int $\ast$ {\em pend\_\-cookie}, cache\_\-inode\_\-endofdir\_\-t $\ast$ {\em peod\_\-met}, cache\_\-inode\_\-dir\_\-entry\_\-t $\ast$ {\em dirent\_\-array}, unsigned int $\ast$ {\em cookie\_\-array}, hash\_\-table\_\-t $\ast$ {\em ht}, cache\_\-inode\_\-client\_\-t $\ast$ {\em pclient}, fsal\_\-op\_\-context\_\-t $\ast$ {\em pcontext}, cache\_\-inode\_\-status\_\-t $\ast$ {\em pstatus})}\label{cache__inode__readdir_8c_a5}


cache\_\-inode\_\-readdir: Reads partially a directory.

Looks up for a name in a directory indicated by a cached entry. The directory should have been cached before. This is the only function in the {\bf cache\_\-inode\_\-readdir.c}{\rm (p.\,\pageref{cache__inode__readdir_8c})} file that manages MT safety on a dir chain.

\begin{Desc}
\item[Parameters:]
\begin{description}
\item[{\em pentry}][IN] entry for the parent directory to be read. \item[{\em cookie}][IN] cookie for the readdir operation (basically the offset). \item[{\em nbwanted}][IN] Maximum number of directory entries wanted. \item[{\em peod\_\-met}][OUT] A flag to know if end of directory was met during this call. \item[{\em dirent\_\-array}][OUT] the resulting array of found directory entries. \item[{\em ht}][IN] hash table used for the cache, unused in this call. \item[{\em pclient}][INOUT] ressource allocated by the client for the nfs management. \item[{\em pcontext}][IN] FSAL credentials \item[{\em pstatus}][OUT] returned status.\end{description}
\end{Desc}
\begin{Desc}
\item[Returns:]CACHE\_\-INODE\_\-SUCCESS if operation is a success \par
 

CACHE\_\-INODE\_\-BAD\_\-TYPE if entry is not related to a directory\par
 

CACHE\_\-INODE\_\-LRU\_\-ERROR if allocation error occured when validating the entry \end{Desc}


Definition at line 976 of file cache\_\-inode\_\-readdir.c.

References cache\_\-inode\_\-access\_\-no\_\-mutex(), cache\_\-inode\_\-readdir\_\-populate(), cache\_\-inode\_\-renew\_\-entry(), and cache\_\-inode\_\-valid().

Referenced by main().\index{cache_inode_readdir.c@{cache\_\-inode\_\-readdir.c}!cache_inode_readdir_populate@{cache\_\-inode\_\-readdir\_\-populate}}
\index{cache_inode_readdir_populate@{cache\_\-inode\_\-readdir\_\-populate}!cache_inode_readdir.c@{cache\_\-inode\_\-readdir.c}}
\subsubsection{\setlength{\rightskip}{0pt plus 5cm}cache\_\-inode\_\-status\_\-t cache\_\-inode\_\-readdir\_\-populate (cache\_\-entry\_\-t $\ast$ {\em pentry\_\-dir}, hash\_\-table\_\-t $\ast$ {\em ht}, cache\_\-inode\_\-client\_\-t $\ast$ {\em pclient}, fsal\_\-op\_\-context\_\-t $\ast$ {\em pcontext}, cache\_\-inode\_\-status\_\-t $\ast$ {\em pstatus})}\label{cache__inode__readdir_8c_a4}


cache\_\-inode\_\-readdir\_\-populate: fully reads a directory in FSAL and caches the related entries.

fully reads a directory in FSAL and caches the related entries. No MT safety managed here !!

\begin{Desc}
\item[Parameters:]
\begin{description}
\item[{\em pentry}][IN] entry for the parent directory to be read. This must be a DIR\_\-BEGINNING \item[{\em ht}][IN] hash table used for the cache, unused in this call. \item[{\em pclient}][INOUT] ressource allocated by the client for the nfs management. \item[{\em pcontext}][IN] FSAL credentials \item[{\em pstatus}][OUT] returned status. \end{description}
\end{Desc}


Definition at line 752 of file cache\_\-inode\_\-readdir.c.

References cache\_\-inode\_\-add\_\-cached\_\-dirent(), cache\_\-inode\_\-error\_\-convert(), cache\_\-inode\_\-fsal\_\-type\_\-convert(), cache\_\-inode\_\-invalidate\_\-all\_\-cached\_\-dirent(), cache\_\-inode\_\-kill\_\-entry(), and cache\_\-inode\_\-new\_\-entry().

Referenced by cache\_\-inode\_\-create(), and cache\_\-inode\_\-readdir().\index{cache_inode_readdir.c@{cache\_\-inode\_\-readdir.c}!cache_inode_remove_cached_dirent@{cache\_\-inode\_\-remove\_\-cached\_\-dirent}}
\index{cache_inode_remove_cached_dirent@{cache\_\-inode\_\-remove\_\-cached\_\-dirent}!cache_inode_readdir.c@{cache\_\-inode\_\-readdir.c}}
\subsubsection{\setlength{\rightskip}{0pt plus 5cm}cache\_\-inode\_\-status\_\-t cache\_\-inode\_\-remove\_\-cached\_\-dirent (cache\_\-entry\_\-t $\ast$ {\em pentry\_\-parent}, fsal\_\-name\_\-t $\ast$ {\em pname}, hash\_\-table\_\-t $\ast$ {\em ht}, cache\_\-inode\_\-client\_\-t $\ast$ {\em pclient}, cache\_\-inode\_\-status\_\-t $\ast$ {\em pstatus})}\label{cache__inode__readdir_8c_a3}


cache\_\-inode\_\-remove\_\-cached\_\-dirent: Removes a directory entry to a cached directory.

Removes a directory entry to a cached directory. No MT safety managed here !!

\begin{Desc}
\item[Parameters:]
\begin{description}
\item[{\em pentry\_\-parent}][INOUT] cache entry representing the directory to be managed. \item[{\em name}][IN] name of the entry to remove. \item[{\em ht}][IN] hash table used for the cache, unused in this call. \item[{\em pclient}][INOUT] ressource allocated by the client for the nfs management. \item[{\em pstatus}][OUT] returned status.\end{description}
\end{Desc}
\begin{Desc}
\item[Returns:]the same as $\ast$pstatus \end{Desc}


Definition at line 669 of file cache\_\-inode\_\-readdir.c.

References cache\_\-inode\_\-operate\_\-cached\_\-dirent().

Referenced by cache\_\-inode\_\-remove\_\-sw(), and cache\_\-inode\_\-rename().