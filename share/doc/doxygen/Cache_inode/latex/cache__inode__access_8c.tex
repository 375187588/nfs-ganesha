\section{cache\_\-inode\_\-access.c File Reference}
\label{cache__inode__access_8c}\index{cache_inode_access.c@{cache\_\-inode\_\-access.c}}
Check for object accessibility. 

{\tt \#include \char`\"{}LRU\_\-List.h\char`\"{}}\par
{\tt \#include \char`\"{}log\_\-functions.h\char`\"{}}\par
{\tt \#include \char`\"{}Hash\-Data.h\char`\"{}}\par
{\tt \#include \char`\"{}Hash\-Table.h\char`\"{}}\par
{\tt \#include \char`\"{}fsal.h\char`\"{}}\par
{\tt \#include \char`\"{}cache\_\-inode.h\char`\"{}}\par
{\tt \#include \char`\"{}stuff\_\-alloc.h\char`\"{}}\par
{\tt \#include $<$unistd.h$>$}\par
{\tt \#include $<$sys/types.h$>$}\par
{\tt \#include $<$sys/param.h$>$}\par
{\tt \#include $<$time.h$>$}\par
{\tt \#include $<$pthread.h$>$}\par
\subsection*{Functions}
\begin{CompactItemize}
\item 
cache\_\-inode\_\-status\_\-t {\bf cache\_\-inode\_\-access\_\-sw} (cache\_\-entry\_\-t $\ast$pentry, fsal\_\-accessflags\_\-t access\_\-type, hash\_\-table\_\-t $\ast$ht, cache\_\-inode\_\-client\_\-t $\ast$pclient, fsal\_\-op\_\-context\_\-t $\ast$pcontext, cache\_\-inode\_\-status\_\-t $\ast$pstatus, int use\_\-mutex)
\item 
cache\_\-inode\_\-status\_\-t {\bf cache\_\-inode\_\-access\_\-no\_\-mutex} (cache\_\-entry\_\-t $\ast$pentry, fsal\_\-accessflags\_\-t access\_\-type, hash\_\-table\_\-t $\ast$ht, cache\_\-inode\_\-client\_\-t $\ast$pclient, fsal\_\-op\_\-context\_\-t $\ast$pcontext, cache\_\-inode\_\-status\_\-t $\ast$pstatus)
\item 
cache\_\-inode\_\-status\_\-t {\bf cache\_\-inode\_\-access} (cache\_\-entry\_\-t $\ast$pentry, fsal\_\-accessflags\_\-t access\_\-type, hash\_\-table\_\-t $\ast$ht, cache\_\-inode\_\-client\_\-t $\ast$pclient, fsal\_\-op\_\-context\_\-t $\ast$pcontext, cache\_\-inode\_\-status\_\-t $\ast$pstatus)
\end{CompactItemize}


\subsection{Detailed Description}
Check for object accessibility. 

\begin{Desc}
\item[Author:]\begin{Desc}
\item[Author]deniel \end{Desc}
\end{Desc}
\begin{Desc}
\item[Date:]\begin{Desc}
\item[Date]2005/11/28 17:02:26 \end{Desc}
\end{Desc}
\begin{Desc}
\item[Version:]\begin{Desc}
\item[Revision]1.19 \end{Desc}
\end{Desc}
{\bf cache\_\-inode\_\-access.c}{\rm (p.\,\pageref{cache__inode__access_8c})} : Check for object accessibility.

Definition in file {\bf cache\_\-inode\_\-access.c}.

\subsection{Function Documentation}
\index{cache_inode_access.c@{cache\_\-inode\_\-access.c}!cache_inode_access@{cache\_\-inode\_\-access}}
\index{cache_inode_access@{cache\_\-inode\_\-access}!cache_inode_access.c@{cache\_\-inode\_\-access.c}}
\subsubsection{\setlength{\rightskip}{0pt plus 5cm}cache\_\-inode\_\-status\_\-t cache\_\-inode\_\-access (cache\_\-entry\_\-t $\ast$ {\em pentry}, fsal\_\-accessflags\_\-t {\em access\_\-type}, hash\_\-table\_\-t $\ast$ {\em ht}, cache\_\-inode\_\-client\_\-t $\ast$ {\em pclient}, fsal\_\-op\_\-context\_\-t $\ast$ {\em pcontext}, cache\_\-inode\_\-status\_\-t $\ast$ {\em pstatus})}\label{cache__inode__access_8c_a2}


cache\_\-inode\_\-access: checks for an entry accessibility.

Checks for an entry accessibility.

\begin{Desc}
\item[Parameters:]
\begin{description}
\item[{\em pentry}][IN] entry pointer for the fs object to be checked. \item[{\em access\_\-type}][IN] flags used to describe the kind of access to be checked. \item[{\em ht}][INOUT] hash table used for the cache. \item[{\em pclient}][INOUT] ressource allocated by the client for the nfs management. \item[{\em pcontext}][IN] FSAL credentials \item[{\em pstatus}][OUT] returned status.\end{description}
\end{Desc}
\begin{Desc}
\item[Returns:]CACHE\_\-INODE\_\-SUCCESS if operation is a success \par
 

CACHE\_\-INODE\_\-LRU\_\-ERROR if allocation error occured when validating the entry \end{Desc}


Definition at line 277 of file cache\_\-inode\_\-access.c.

References cache\_\-inode\_\-access\_\-sw().\index{cache_inode_access.c@{cache\_\-inode\_\-access.c}!cache_inode_access_no_mutex@{cache\_\-inode\_\-access\_\-no\_\-mutex}}
\index{cache_inode_access_no_mutex@{cache\_\-inode\_\-access\_\-no\_\-mutex}!cache_inode_access.c@{cache\_\-inode\_\-access.c}}
\subsubsection{\setlength{\rightskip}{0pt plus 5cm}cache\_\-inode\_\-status\_\-t cache\_\-inode\_\-access\_\-no\_\-mutex (cache\_\-entry\_\-t $\ast$ {\em pentry}, fsal\_\-accessflags\_\-t {\em access\_\-type}, hash\_\-table\_\-t $\ast$ {\em ht}, cache\_\-inode\_\-client\_\-t $\ast$ {\em pclient}, fsal\_\-op\_\-context\_\-t $\ast$ {\em pcontext}, cache\_\-inode\_\-status\_\-t $\ast$ {\em pstatus})}\label{cache__inode__access_8c_a1}


cache\_\-inode\_\-access\_\-no\_\-mutex: checks for an entry accessibility. No mutex management

Checks for an entry accessibility.

\begin{Desc}
\item[Parameters:]
\begin{description}
\item[{\em pentry}][IN] entry pointer for the fs object to be checked. \item[{\em access\_\-type}][IN] flags used to describe the kind of access to be checked. \item[{\em ht}][INOUT] hash table used for the cache. \item[{\em pclient}][INOUT] ressource allocated by the client for the nfs management. \item[{\em pcontext}][IN] FSAL credentials \item[{\em pstatus}][OUT] returned status.\end{description}
\end{Desc}
\begin{Desc}
\item[Returns:]CACHE\_\-INODE\_\-SUCCESS if operation is a success \par
 

CACHE\_\-INODE\_\-LRU\_\-ERROR if allocation error occured when validating the entry \end{Desc}


Definition at line 244 of file cache\_\-inode\_\-access.c.

References cache\_\-inode\_\-access\_\-sw().

Referenced by cache\_\-inode\_\-lookup\_\-sw(), and cache\_\-inode\_\-readdir().\index{cache_inode_access.c@{cache\_\-inode\_\-access.c}!cache_inode_access_sw@{cache\_\-inode\_\-access\_\-sw}}
\index{cache_inode_access_sw@{cache\_\-inode\_\-access\_\-sw}!cache_inode_access.c@{cache\_\-inode\_\-access.c}}
\subsubsection{\setlength{\rightskip}{0pt plus 5cm}cache\_\-inode\_\-status\_\-t cache\_\-inode\_\-access\_\-sw (cache\_\-entry\_\-t $\ast$ {\em pentry}, fsal\_\-accessflags\_\-t {\em access\_\-type}, hash\_\-table\_\-t $\ast$ {\em ht}, cache\_\-inode\_\-client\_\-t $\ast$ {\em pclient}, fsal\_\-op\_\-context\_\-t $\ast$ {\em pcontext}, cache\_\-inode\_\-status\_\-t $\ast$ {\em pstatus}, int {\em use\_\-mutex})}\label{cache__inode__access_8c_a0}


cache\_\-inode\_\-access: checks for an entry accessibility.

Checks for an entry accessibility.

\begin{Desc}
\item[Parameters:]
\begin{description}
\item[{\em pentry}][IN] entry pointer for the fs object to be checked. \item[{\em access\_\-type}][IN] flags used to describe the kind of access to be checked. \item[{\em ht}][INOUT] hash table used for the cache. \item[{\em pclient}][INOUT] ressource allocated by the client for the nfs management. \item[{\em pcontext}][IN] FSAL context \item[{\em pstatus}][OUT] returned status. \item[{\em use\_\-mutex}][IN] a flag to tell if mutex are to be used or not.\end{description}
\end{Desc}
\begin{Desc}
\item[Returns:]CACHE\_\-INODE\_\-SUCCESS if operation is a success \par
 

CACHE\_\-INODE\_\-LRU\_\-ERROR if allocation error occured when validating the entry \par
 

any other values show an unauthorized access. \end{Desc}


Definition at line 123 of file cache\_\-inode\_\-access.c.

References cache\_\-inode\_\-error\_\-convert(), cache\_\-inode\_\-get\_\-attributes(), cache\_\-inode\_\-get\_\-fsal\_\-handle(), cache\_\-inode\_\-kill\_\-entry(), and cache\_\-inode\_\-valid().

Referenced by cache\_\-inode\_\-access(), and cache\_\-inode\_\-access\_\-no\_\-mutex().