\section{cache\_\-content\_\-flush.c File Reference}
\label{cache__content__flush_8c}\index{cache\_\-content\_\-flush.c@{cache\_\-content\_\-flush.c}}
{\tt \#include \char`\"{}LRU\_\-List.h\char`\"{}}\par
{\tt \#include \char`\"{}log\_\-functions.h\char`\"{}}\par
{\tt \#include \char`\"{}HashData.h\char`\"{}}\par
{\tt \#include \char`\"{}HashTable.h\char`\"{}}\par
{\tt \#include \char`\"{}fsal.h\char`\"{}}\par
{\tt \#include \char`\"{}cache\_\-inode.h\char`\"{}}\par
{\tt \#include \char`\"{}cache\_\-content.h\char`\"{}}\par
{\tt \#include $<$unistd.h$>$}\par
{\tt \#include $<$sys/types.h$>$}\par
{\tt \#include $<$sys/param.h$>$}\par
{\tt \#include $<$time.h$>$}\par
{\tt \#include $<$pthread.h$>$}\par
{\tt \#include $<$errno.h$>$}\par
{\tt \#include $<$string.h$>$}\par
\subsection*{Functions}
\begin{CompactItemize}
\item 
cache\_\-content\_\-status\_\-t {\bf cache\_\-content\_\-flush} (cache\_\-content\_\-entry\_\-t $\ast$pentry, cache\_\-content\_\-flush\_\-behaviour\_\-t flushhow, cache\_\-content\_\-client\_\-t $\ast$pclient, fsal\_\-op\_\-context\_\-t $\ast$pcontext, cache\_\-content\_\-status\_\-t $\ast$pstatus)
\item 
cache\_\-content\_\-status\_\-t {\bf cache\_\-content\_\-refresh} (cache\_\-content\_\-entry\_\-t $\ast$pentry, cache\_\-content\_\-client\_\-t $\ast$pclient, fsal\_\-op\_\-context\_\-t $\ast$pcontext, cache\_\-content\_\-refresh\_\-how\_\-t how, cache\_\-content\_\-status\_\-t $\ast$pstatus)
\item 
cache\_\-content\_\-status\_\-t {\bf cache\_\-content\_\-sync\_\-all} (cache\_\-content\_\-client\_\-t $\ast$pclient, fsal\_\-op\_\-context\_\-t $\ast$pcontext, cache\_\-content\_\-status\_\-t $\ast$pstatus)
\end{CompactItemize}


\subsection{Function Documentation}
\index{cache\_\-content\_\-flush.c@{cache\_\-content\_\-flush.c}!cache\_\-content\_\-flush@{cache\_\-content\_\-flush}}
\index{cache\_\-content\_\-flush@{cache\_\-content\_\-flush}!cache_content_flush.c@{cache\_\-content\_\-flush.c}}
\subsubsection[{cache\_\-content\_\-flush}]{\setlength{\rightskip}{0pt plus 5cm}cache\_\-content\_\-status\_\-t cache\_\-content\_\-flush (cache\_\-content\_\-entry\_\-t $\ast$ {\em pentry}, \/  cache\_\-content\_\-flush\_\-behaviour\_\-t {\em flushhow}, \/  cache\_\-content\_\-client\_\-t $\ast$ {\em pclient}, \/  fsal\_\-op\_\-context\_\-t $\ast$ {\em pcontext}, \/  cache\_\-content\_\-status\_\-t $\ast$ {\em pstatus})}\label{cache__content__flush_8c_c66f7369f13dd6fec79abfccabde8a18}


cache\_\-content\_\-flush: Flushes the content of a file in the local cache to the FSAL data.

Flushes the content of a file in the local cache to the FSAL data. This routine should be called only from the cache\_\-inode layer.

No lock management is done in this layer: the related pentry in the cache inode layer is locked and will prevent from concurent accesses.

\begin{Desc}
\item[Parameters:]
\begin{description}
\item[{\em pentry}][IN] entry in file content layer whose content is to be flushed. \item[{\em flushhow}][IN] should we delete the cached entry in local or not ? \item[{\em pclient}][IN] ressource allocated by the client for the nfs management.  [OUT] returned status.\end{description}
\end{Desc}
\begin{Desc}
\item[Returns:]CACHE\_\-CONTENT\_\-SUCCESS is successful . \end{Desc}


Definition at line 128 of file cache\_\-content\_\-flush.c.\index{cache\_\-content\_\-flush.c@{cache\_\-content\_\-flush.c}!cache\_\-content\_\-refresh@{cache\_\-content\_\-refresh}}
\index{cache\_\-content\_\-refresh@{cache\_\-content\_\-refresh}!cache_content_flush.c@{cache\_\-content\_\-flush.c}}
\subsubsection[{cache\_\-content\_\-refresh}]{\setlength{\rightskip}{0pt plus 5cm}cache\_\-content\_\-status\_\-t cache\_\-content\_\-refresh (cache\_\-content\_\-entry\_\-t $\ast$ {\em pentry}, \/  cache\_\-content\_\-client\_\-t $\ast$ {\em pclient}, \/  fsal\_\-op\_\-context\_\-t $\ast$ {\em pcontext}, \/  cache\_\-content\_\-refresh\_\-how\_\-t {\em how}, \/  cache\_\-content\_\-status\_\-t $\ast$ {\em pstatus})}\label{cache__content__flush_8c_8caa78929e28b45a53868eb38c0d54a2}


cache\_\-content\_\-refresh: Refreshes the whole content of a file in the local cache to the FSAL data.

Refreshes the whole content of a file in the local cache to the FSAL data. This routine should be called only from the cache\_\-inode layer.

No lock management is done in this layer: the related pentry in the cache inode layer is locked and will prevent from concurent accesses.

\begin{Desc}
\item[Parameters:]
\begin{description}
\item[{\em pentry}][IN] entry in file content layer whose content is to be flushed. \item[{\em pclient}][IN] ressource allocated by the client for the nfs management.  [OUT] returned status.\end{description}
\end{Desc}
\begin{Desc}
\item[Returns:]CACHE\_\-CONTENT\_\-SUCCESS is successful .\end{Desc}
\begin{Desc}
\item[{\bf Todo}]: BUGAZOMEU: gestion de coherence de date a mettre en place \end{Desc}


Definition at line 275 of file cache\_\-content\_\-flush.c.\index{cache\_\-content\_\-flush.c@{cache\_\-content\_\-flush.c}!cache\_\-content\_\-sync\_\-all@{cache\_\-content\_\-sync\_\-all}}
\index{cache\_\-content\_\-sync\_\-all@{cache\_\-content\_\-sync\_\-all}!cache_content_flush.c@{cache\_\-content\_\-flush.c}}
\subsubsection[{cache\_\-content\_\-sync\_\-all}]{\setlength{\rightskip}{0pt plus 5cm}cache\_\-content\_\-status\_\-t cache\_\-content\_\-sync\_\-all (cache\_\-content\_\-client\_\-t $\ast$ {\em pclient}, \/  fsal\_\-op\_\-context\_\-t $\ast$ {\em pcontext}, \/  cache\_\-content\_\-status\_\-t $\ast$ {\em pstatus})}\label{cache__content__flush_8c_479b822aceaf5a1bc414d603b250cc84}




Definition at line 405 of file cache\_\-content\_\-flush.c.