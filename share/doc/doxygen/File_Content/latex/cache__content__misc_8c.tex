\section{cache\_\-content\_\-misc.c File Reference}
\label{cache__content__misc_8c}\index{cache_content_misc.c@{cache\_\-content\_\-misc.c}}
Management of the file content cache: miscellaneous functions. 

{\tt \#include \char`\"{}fsal.h\char`\"{}}\par
{\tt \#include \char`\"{}LRU\_\-List.h\char`\"{}}\par
{\tt \#include \char`\"{}log\_\-functions.h\char`\"{}}\par
{\tt \#include \char`\"{}Hash\-Data.h\char`\"{}}\par
{\tt \#include \char`\"{}Hash\-Table.h\char`\"{}}\par
{\tt \#include \char`\"{}cache\_\-inode.h\char`\"{}}\par
{\tt \#include \char`\"{}cache\_\-content.h\char`\"{}}\par
{\tt \#include \char`\"{}stuff\_\-alloc.h\char`\"{}}\par
{\tt \#include \char`\"{}nfs\_\-exports.h\char`\"{}}\par
{\tt \#include $<$unistd.h$>$}\par
{\tt \#include $<$sys/types.h$>$}\par
{\tt \#include $<$sys/param.h$>$}\par
{\tt \#include $<$time.h$>$}\par
{\tt \#include $<$pthread.h$>$}\par
{\tt \#include $<$string.h$>$}\par
{\tt \#include $<$sys/vfs.h$>$}\par
{\tt \#include $<$libgen.h$>$}\par
\subsection*{Defines}
\begin{CompactItemize}
\item 
\#define {\bf ALPHABET\_\-LEN}\ 16
\item 
\#define {\bf PRIME\_\-16BITS}\ 65521
\end{CompactItemize}
\subsection*{Functions}
\begin{CompactItemize}
\item 
short {\bf Hash\-File\-ID4} (u\_\-int64\_\-t fileid4)
\item 
cache\_\-content\_\-status\_\-t {\bf cache\_\-content\_\-create\_\-name} (char $\ast$path, cache\_\-content\_\-nametype\_\-t type, fsal\_\-op\_\-context\_\-t $\ast$pcontext, cache\_\-entry\_\-t $\ast$pentry\_\-inode, cache\_\-content\_\-client\_\-t $\ast$pclient)
\item 
int {\bf cache\_\-content\_\-get\_\-export\_\-id} (char $\ast$dirname)
\item 
u\_\-int64\_\-t {\bf cache\_\-content\_\-get\_\-inum} (char $\ast$filename)
\item 
int {\bf cache\_\-content\_\-get\_\-datapath} (char $\ast$basepath, u\_\-int64\_\-t inum, char $\ast$datapath)
\item 
off\_\-t {\bf cache\_\-content\_\-recover\_\-size} (char $\ast$basepath, u\_\-int64\_\-t inum)
\item 
off\_\-t {\bf cache\_\-content\_\-get\_\-cached\_\-size} (cache\_\-content\_\-entry\_\-t $\ast$pentry)
\item 
cache\_\-inode\_\-status\_\-t {\bf cache\_\-content\_\-error\_\-convert} (cache\_\-content\_\-status\_\-t status)
\item 
off\_\-t {\bf cache\_\-content\_\-fsal\_\-seek\_\-convert} (fsal\_\-seek\_\-t seek, cache\_\-content\_\-status\_\-t $\ast$pstatus)
\item 
size\_\-t {\bf cache\_\-content\_\-fsal\_\-size\_\-convert} (fsal\_\-size\_\-t size, cache\_\-content\_\-status\_\-t $\ast$pstatus)
\item 
cache\_\-content\_\-status\_\-t {\bf cache\_\-content\_\-prepare\_\-directories} (exportlist\_\-t $\ast$pexportlist, char $\ast$cache\_\-dir, cache\_\-content\_\-status\_\-t $\ast$pstatus)
\item 
cache\_\-content\_\-status\_\-t {\bf cache\_\-content\_\-valid} (cache\_\-content\_\-entry\_\-t $\ast$pentry, cache\_\-inode\_\-op\_\-t op, cache\_\-content\_\-client\_\-t $\ast$pclient)
\item 
cache\_\-content\_\-status\_\-t {\bf cache\_\-content\_\-check\_\-threshold} (char $\ast$datacache\_\-path, unsigned int threshold\_\-min, unsigned int threshold\_\-max, int $\ast$p\_\-bool\_\-overflow, unsigned long $\ast$p\_\-blocks\_\-to\_\-lwm)
\item 
int {\bf cache\_\-content\_\-local\_\-cache\_\-opendir} (char $\ast$cache\_\-dir, cache\_\-content\_\-dirinfo\_\-t $\ast$pdirectory)
\item 
cache\_\-content\_\-status\_\-t {\bf cache\_\-content\_\-test\_\-cached} (cache\_\-entry\_\-t $\ast$pentry\_\-inode, cache\_\-content\_\-client\_\-t $\ast$pclient, fsal\_\-op\_\-context\_\-t $\ast$pcontext, cache\_\-content\_\-status\_\-t $\ast$pstatus)
\item 
int {\bf cache\_\-content\_\-local\_\-cache\_\-dir\_\-iter} (cache\_\-content\_\-dirinfo\_\-t $\ast$directory, struct dirent $\ast$pdir\_\-entry, unsigned int index, unsigned int mod)
\item 
void {\bf cache\_\-content\_\-local\_\-cache\_\-closedir} (cache\_\-content\_\-dirinfo\_\-t $\ast$directory)
\end{CompactItemize}
\subsection*{Variables}
\begin{CompactItemize}
\item 
unsigned int {\bf cache\_\-content\_\-dir\_\-errno}
\end{CompactItemize}


\subsection{Detailed Description}
Management of the file content cache: miscellaneous functions. 

\begin{Desc}
\item[Author:]\begin{Desc}
\item[Author]leibovic \end{Desc}
\end{Desc}
\begin{Desc}
\item[Date:]\begin{Desc}
\item[Date]2006/01/18 07:29:11 \end{Desc}
\end{Desc}
\begin{Desc}
\item[Version:]\begin{Desc}
\item[Revision]1.14 \end{Desc}
\end{Desc}
{\bf cache\_\-content\_\-misc.c}{\rm (p.\,\pageref{cache__content__misc_8c})} : Management of the file content cache, miscellaneous functions.

Definition in file {\bf cache\_\-content\_\-misc.c}.

\subsection{Define Documentation}
\index{cache_content_misc.c@{cache\_\-content\_\-misc.c}!ALPHABET_LEN@{ALPHABET\_\-LEN}}
\index{ALPHABET_LEN@{ALPHABET\_\-LEN}!cache_content_misc.c@{cache\_\-content\_\-misc.c}}
\subsubsection{\setlength{\rightskip}{0pt plus 5cm}\#define ALPHABET\_\-LEN\ 16}\label{cache__content__misc_8c_a0}




Referenced by Hash\-File\-ID4().\index{cache_content_misc.c@{cache\_\-content\_\-misc.c}!PRIME_16BITS@{PRIME\_\-16BITS}}
\index{PRIME_16BITS@{PRIME\_\-16BITS}!cache_content_misc.c@{cache\_\-content\_\-misc.c}}
\subsubsection{\setlength{\rightskip}{0pt plus 5cm}\#define PRIME\_\-16BITS\ 65521}\label{cache__content__misc_8c_a1}




\subsection{Function Documentation}
\index{cache_content_misc.c@{cache\_\-content\_\-misc.c}!cache_content_check_threshold@{cache\_\-content\_\-check\_\-threshold}}
\index{cache_content_check_threshold@{cache\_\-content\_\-check\_\-threshold}!cache_content_misc.c@{cache\_\-content\_\-misc.c}}
\subsubsection{\setlength{\rightskip}{0pt plus 5cm}cache\_\-content\_\-status\_\-t cache\_\-content\_\-check\_\-threshold (char $\ast$ {\em datacache\_\-path}, unsigned int {\em threshold\_\-min}, unsigned int {\em threshold\_\-max}, int $\ast$ {\em p\_\-bool\_\-overflow}, unsigned long $\ast$ {\em p\_\-blocks\_\-to\_\-lwm})}\label{cache__content__misc_8c_a15}


cache\_\-content\_\-check\_\-threshold: check datacache filesystem's threshold.

\begin{Desc}
\item[Parameters:]
\begin{description}
\item[{\em datacache\_\-path}][IN] the datacache filesystem's path. \item[{\em threshold\_\-min}][IN] the low watermark for the filesystem (in percent). \item[{\em threshold\_\-max}][IN] the high watermark for the filesystem (in percent). \item[{\em p\_\-bool\_\-overflow}][OUT] boolean that indicates whether the FS overcomes the high watermark. \item[{\em p\_\-blocks\_\-to\_\-lwm}][OUT] if bool\_\-overflow is set to true, this indicates the number of blocks to be purged in order to reach the low watermark.\end{description}
\end{Desc}
\begin{Desc}
\item[Returns:]CACHE\_\-CONTENT\_\-SUCCESS if successful \par
 

CACHE\_\-CONTENT\_\-INVALID\_\-ARGUMENT if some argument has an unexpected value\par
 

CACHE\_\-CONTENT\_\-LOCAL\_\-CACHE\_\-ERROR if an error occured while getting informations from datacache filesystem. \end{Desc}


Definition at line 650 of file cache\_\-content\_\-misc.c.

Referenced by main().\index{cache_content_misc.c@{cache\_\-content\_\-misc.c}!cache_content_create_name@{cache\_\-content\_\-create\_\-name}}
\index{cache_content_create_name@{cache\_\-content\_\-create\_\-name}!cache_content_misc.c@{cache\_\-content\_\-misc.c}}
\subsubsection{\setlength{\rightskip}{0pt plus 5cm}cache\_\-content\_\-status\_\-t cache\_\-content\_\-create\_\-name (char $\ast$ {\em path}, cache\_\-content\_\-nametype\_\-t {\em type}, fsal\_\-op\_\-context\_\-t $\ast$ {\em pcontext}, cache\_\-entry\_\-t $\ast$ {\em pentry\_\-inode}, cache\_\-content\_\-client\_\-t $\ast$ {\em pclient})}\label{cache__content__misc_8c_a4}


cache\_\-content\_\-create\_\-name: Creates a name in the local fs for a cached entry.

Creates a name in the local fs for a cached entry and creates the directories that whil contain this file.

\begin{Desc}
\item[Parameters:]
\begin{description}
\item[{\em path}][OUT] buffer to be used for storing the name. \item[{\em type}][IN] type of pathname to be created. \item[{\em pentry\_\-inode}][OUT] Entry in Cache inode layer related to the file content entry. \item[{\em pclient}][IN] resources allocated for the file content client.\end{description}
\end{Desc}
\begin{Desc}
\item[Returns:]CACHE\_\-CONTENT\_\-SUCCESS if operation is a success, other values show an error. \end{Desc}


Definition at line 150 of file cache\_\-content\_\-misc.c.

References Hash\-File\-ID4().

Referenced by cache\_\-content\_\-new\_\-entry(), and cache\_\-content\_\-test\_\-cached().\index{cache_content_misc.c@{cache\_\-content\_\-misc.c}!cache_content_error_convert@{cache\_\-content\_\-error\_\-convert}}
\index{cache_content_error_convert@{cache\_\-content\_\-error\_\-convert}!cache_content_misc.c@{cache\_\-content\_\-misc.c}}
\subsubsection{\setlength{\rightskip}{0pt plus 5cm}cache\_\-inode\_\-status\_\-t cache\_\-content\_\-error\_\-convert (cache\_\-content\_\-status\_\-t {\em status})}\label{cache__content__misc_8c_a10}


cache\_\-content\_\-error\_\-convert: Converts a cache\_\-content\_\-status to a cache\_\-inode\_\-status.

Converts a cache\_\-content\_\-status to a cache\_\-inode\_\-status.

\begin{Desc}
\item[Parameters:]
\begin{description}
\item[{\em status}][IN] File content status to be converted.\end{description}
\end{Desc}
\begin{Desc}
\item[Returns:]a cache\_\-inode\_\-status\_\-t resulting from the conversion. \end{Desc}


Definition at line 400 of file cache\_\-content\_\-misc.c.\index{cache_content_misc.c@{cache\_\-content\_\-misc.c}!cache_content_fsal_seek_convert@{cache\_\-content\_\-fsal\_\-seek\_\-convert}}
\index{cache_content_fsal_seek_convert@{cache\_\-content\_\-fsal\_\-seek\_\-convert}!cache_content_misc.c@{cache\_\-content\_\-misc.c}}
\subsubsection{\setlength{\rightskip}{0pt plus 5cm}off\_\-t cache\_\-content\_\-fsal\_\-seek\_\-convert (fsal\_\-seek\_\-t {\em seek}, cache\_\-content\_\-status\_\-t $\ast$ {\em pstatus})}\label{cache__content__misc_8c_a11}


cache\_\-content\_\-fsal\_\-seek\_\-convert: converts a fsal\_\-seek\_\-t to unix offet.

Converts a fsal\_\-seek\_\-t to unix offet. Non absolulte fsal\_\-seek\_\-t will produce an error.

\begin{Desc}
\item[Parameters:]
\begin{description}
\item[{\em seek}][IN] FSAL Seek descriptor. \item[{\em pstatus}][OUT] pointer to the status.\end{description}
\end{Desc}
\begin{Desc}
\item[Returns:]the converted value. \end{Desc}


Definition at line 459 of file cache\_\-content\_\-misc.c.

Referenced by cache\_\-content\_\-rdwr().\index{cache_content_misc.c@{cache\_\-content\_\-misc.c}!cache_content_fsal_size_convert@{cache\_\-content\_\-fsal\_\-size\_\-convert}}
\index{cache_content_fsal_size_convert@{cache\_\-content\_\-fsal\_\-size\_\-convert}!cache_content_misc.c@{cache\_\-content\_\-misc.c}}
\subsubsection{\setlength{\rightskip}{0pt plus 5cm}size\_\-t cache\_\-content\_\-fsal\_\-size\_\-convert (fsal\_\-size\_\-t {\em size}, cache\_\-content\_\-status\_\-t $\ast$ {\em pstatus})}\label{cache__content__misc_8c_a12}


cache\_\-content\_\-fsal\_\-size\_\-convert: converts a fsal\_\-size\_\-t to unix size.

Converts a fsal\_\-seek\_\-t to unix size.

\begin{Desc}
\item[Parameters:]
\begin{description}
\item[{\em seek}][IN] FSAL Seek descriptor. \item[{\em pstatus}][OUT] pointer to the status.\end{description}
\end{Desc}
\begin{Desc}
\item[Returns:]the converted value. \end{Desc}


Definition at line 486 of file cache\_\-content\_\-misc.c.

Referenced by cache\_\-content\_\-rdwr().\index{cache_content_misc.c@{cache\_\-content\_\-misc.c}!cache_content_get_cached_size@{cache\_\-content\_\-get\_\-cached\_\-size}}
\index{cache_content_get_cached_size@{cache\_\-content\_\-get\_\-cached\_\-size}!cache_content_misc.c@{cache\_\-content\_\-misc.c}}
\subsubsection{\setlength{\rightskip}{0pt plus 5cm}off\_\-t cache\_\-content\_\-get\_\-cached\_\-size (cache\_\-content\_\-entry\_\-t $\ast$ {\em pentry})}\label{cache__content__misc_8c_a9}


cache\_\-content\_\-get\_\-cached\_\-size: recovers the size of a data cached file.

Recovers the size of a data cached file.

\begin{Desc}
\item[Parameters:]
\begin{description}
\item[{\em pentry}][IN] related pentry\end{description}
\end{Desc}
\begin{Desc}
\item[Returns:]the recovered size (as a off\_\-t) or -1 is failed. \end{Desc}


Definition at line 372 of file cache\_\-content\_\-misc.c.\index{cache_content_misc.c@{cache\_\-content\_\-misc.c}!cache_content_get_datapath@{cache\_\-content\_\-get\_\-datapath}}
\index{cache_content_get_datapath@{cache\_\-content\_\-get\_\-datapath}!cache_content_misc.c@{cache\_\-content\_\-misc.c}}
\subsubsection{\setlength{\rightskip}{0pt plus 5cm}int cache\_\-content\_\-get\_\-datapath (char $\ast$ {\em basepath}, u\_\-int64\_\-t {\em inum}, char $\ast$ {\em datapath})}\label{cache__content__misc_8c_a7}


cache\_\-content\_\-get\_\-datapath : recovers the path for a file of a specified inum.

\begin{Desc}
\item[Parameters:]
\begin{description}
\item[{\em basepath}][IN] path to the root of the directory in the cache for the related export entry \item[{\em inum}][IN] inode number for the file whose size is to be recovered. \item[{\em path}][OUT] the absolute path of the file (must be at least a MAXPATHLEN length string).\end{description}
\end{Desc}
\begin{Desc}
\item[Returns:]0 if OK, or -1 is failed. \end{Desc}


Definition at line 304 of file cache\_\-content\_\-misc.c.

References Hash\-File\-ID4().

Referenced by cache\_\-content\_\-emergency\_\-flush(), and cache\_\-content\_\-recover\_\-size().\index{cache_content_misc.c@{cache\_\-content\_\-misc.c}!cache_content_get_export_id@{cache\_\-content\_\-get\_\-export\_\-id}}
\index{cache_content_get_export_id@{cache\_\-content\_\-get\_\-export\_\-id}!cache_content_misc.c@{cache\_\-content\_\-misc.c}}
\subsubsection{\setlength{\rightskip}{0pt plus 5cm}int cache\_\-content\_\-get\_\-export\_\-id (char $\ast$ {\em dirname})}\label{cache__content__misc_8c_a5}


cache\_\-content\_\-get\_\-export\_\-id: gets an export id from an export dirname.

Gets an export id from an export dirname.

\begin{Desc}
\item[Parameters:]
\begin{description}
\item[{\em dirname}][IN] The dirname for the export\_\-id dirname.\end{description}
\end{Desc}
\begin{Desc}
\item[Returns:]-1 if failed, the export\_\-id if successfull. \end{Desc}


Definition at line 242 of file cache\_\-content\_\-misc.c.

Referenced by cache\_\-content\_\-crash\_\-recover().\index{cache_content_misc.c@{cache\_\-content\_\-misc.c}!cache_content_get_inum@{cache\_\-content\_\-get\_\-inum}}
\index{cache_content_get_inum@{cache\_\-content\_\-get\_\-inum}!cache_content_misc.c@{cache\_\-content\_\-misc.c}}
\subsubsection{\setlength{\rightskip}{0pt plus 5cm}u\_\-int64\_\-t cache\_\-content\_\-get\_\-inum (char $\ast$ {\em filename})}\label{cache__content__misc_8c_a6}


cache\_\-content\_\-get\_\-inum: gets an inode number fronm a cache filename.

Gets an inode number fronm a cache filename.

\begin{Desc}
\item[Parameters:]
\begin{description}
\item[{\em filename}][IN] The filename to be parsed.\end{description}
\end{Desc}
\begin{Desc}
\item[Returns:]0 if failed, the inum if successfull. \end{Desc}


Definition at line 267 of file cache\_\-content\_\-misc.c.

Referenced by cache\_\-content\_\-crash\_\-recover(), and cache\_\-content\_\-emergency\_\-flush().\index{cache_content_misc.c@{cache\_\-content\_\-misc.c}!cache_content_local_cache_closedir@{cache\_\-content\_\-local\_\-cache\_\-closedir}}
\index{cache_content_local_cache_closedir@{cache\_\-content\_\-local\_\-cache\_\-closedir}!cache_content_misc.c@{cache\_\-content\_\-misc.c}}
\subsubsection{\setlength{\rightskip}{0pt plus 5cm}void cache\_\-content\_\-local\_\-cache\_\-closedir (cache\_\-content\_\-dirinfo\_\-t $\ast$ {\em directory})}\label{cache__content__misc_8c_a19}


cache\_\-content\_\-local\_\-cache\_\-closedir: Close a local cache directory associated to an export entry.

\begin{Desc}
\item[Parameters:]
\begin{description}
\item[{\em directory\mbox{[}IN\mbox{]}}]the handle to the directory to be closed\end{description}
\end{Desc}
\begin{Desc}
\item[Returns:]nothing (void function) \end{Desc}


Definition at line 1034 of file cache\_\-content\_\-misc.c.

Referenced by cache\_\-content\_\-crash\_\-recover(), and cache\_\-content\_\-emergency\_\-flush().\index{cache_content_misc.c@{cache\_\-content\_\-misc.c}!cache_content_local_cache_dir_iter@{cache\_\-content\_\-local\_\-cache\_\-dir\_\-iter}}
\index{cache_content_local_cache_dir_iter@{cache\_\-content\_\-local\_\-cache\_\-dir\_\-iter}!cache_content_misc.c@{cache\_\-content\_\-misc.c}}
\subsubsection{\setlength{\rightskip}{0pt plus 5cm}int cache\_\-content\_\-local\_\-cache\_\-dir\_\-iter (cache\_\-content\_\-dirinfo\_\-t $\ast$ {\em directory}, struct dirent $\ast$ {\em pdir\_\-entry}, unsigned int {\em index}, unsigned int {\em mod})}\label{cache__content__misc_8c_a18}


cache\_\-content\_\-local\_\-cache\_\-dir\_\-iter: iterate on a local cache directory to get the entry one by one

\begin{Desc}
\item[Parameters:]
\begin{description}
\item[{\em directory}][IN] the directory to be read \item[{\em index}][IN] thread index for multithreaded flushes (first has index 0) \item[{\em mod}][IN] modulus for multithreaded flushes (number of threads) \item[{\em pdir\_\-entry}][OUT] found dir\_\-entry\end{description}
\end{Desc}
\begin{Desc}
\item[Returns:]TRUE if OK, FALSE if NOK. \end{Desc}


Definition at line 827 of file cache\_\-content\_\-misc.c.

References cache\_\-content\_\-dir\_\-errno.

Referenced by cache\_\-content\_\-crash\_\-recover(), and cache\_\-content\_\-emergency\_\-flush().\index{cache_content_misc.c@{cache\_\-content\_\-misc.c}!cache_content_local_cache_opendir@{cache\_\-content\_\-local\_\-cache\_\-opendir}}
\index{cache_content_local_cache_opendir@{cache\_\-content\_\-local\_\-cache\_\-opendir}!cache_content_misc.c@{cache\_\-content\_\-misc.c}}
\subsubsection{\setlength{\rightskip}{0pt plus 5cm}int cache\_\-content\_\-local\_\-cache\_\-opendir (char $\ast$ {\em cache\_\-dir}, cache\_\-content\_\-dirinfo\_\-t $\ast$ {\em pdirectory})}\label{cache__content__misc_8c_a16}


cache\_\-content\_\-local\_\-cache\_\-opendir: Open a local cache directory associated to an export entry.

\begin{Desc}
\item[Parameters:]
\begin{description}
\item[{\em cache\_\-dir}][IN] the path to the directory associated with the export entry \item[{\em pdirectory}][OUT] pointer to trhe openend directory\end{description}
\end{Desc}
\begin{Desc}
\item[Returns:]the handle to the directory or NULL is failed \end{Desc}


Definition at line 729 of file cache\_\-content\_\-misc.c.

References cache\_\-content\_\-dir\_\-errno.

Referenced by cache\_\-content\_\-crash\_\-recover(), and cache\_\-content\_\-emergency\_\-flush().\index{cache_content_misc.c@{cache\_\-content\_\-misc.c}!cache_content_prepare_directories@{cache\_\-content\_\-prepare\_\-directories}}
\index{cache_content_prepare_directories@{cache\_\-content\_\-prepare\_\-directories}!cache_content_misc.c@{cache\_\-content\_\-misc.c}}
\subsubsection{\setlength{\rightskip}{0pt plus 5cm}cache\_\-content\_\-status\_\-t cache\_\-content\_\-prepare\_\-directories (exportlist\_\-t $\ast$ {\em pexportlist}, char $\ast$ {\em cache\_\-dir}, cache\_\-content\_\-status\_\-t $\ast$ {\em pstatus})}\label{cache__content__misc_8c_a13}


cache\_\-content\_\-prepare\_\-directories: do the mkdir to set the data cache directories

do the mkdir to set the data cache directories.

\begin{Desc}
\item[Parameters:]
\begin{description}
\item[{\em pexportlist}][IN] export list \item[{\em pstatus}][OUT] pointer to the status.\end{description}
\end{Desc}
\begin{Desc}
\item[Returns:]the status for the operation \end{Desc}


Definition at line 508 of file cache\_\-content\_\-misc.c.\index{cache_content_misc.c@{cache\_\-content\_\-misc.c}!cache_content_recover_size@{cache\_\-content\_\-recover\_\-size}}
\index{cache_content_recover_size@{cache\_\-content\_\-recover\_\-size}!cache_content_misc.c@{cache\_\-content\_\-misc.c}}
\subsubsection{\setlength{\rightskip}{0pt plus 5cm}off\_\-t cache\_\-content\_\-recover\_\-size (char $\ast$ {\em basepath}, u\_\-int64\_\-t {\em inum})}\label{cache__content__misc_8c_a8}


cache\_\-content\_\-recover\_\-size: recovers the size of a data cached file.

Recovers the size of a data cached file.

\begin{Desc}
\item[Parameters:]
\begin{description}
\item[{\em basepath}][IN] path to the root of the directory in the cache for the related export entry \item[{\em inum}][IN] inode number for the file whose size is to be recovered.\end{description}
\end{Desc}
\begin{Desc}
\item[Returns:]the recovered size (as a off\_\-t) or -1 is failed. \end{Desc}


Definition at line 337 of file cache\_\-content\_\-misc.c.

References cache\_\-content\_\-get\_\-datapath().

Referenced by cache\_\-content\_\-crash\_\-recover().\index{cache_content_misc.c@{cache\_\-content\_\-misc.c}!cache_content_test_cached@{cache\_\-content\_\-test\_\-cached}}
\index{cache_content_test_cached@{cache\_\-content\_\-test\_\-cached}!cache_content_misc.c@{cache\_\-content\_\-misc.c}}
\subsubsection{\setlength{\rightskip}{0pt plus 5cm}cache\_\-content\_\-status\_\-t cache\_\-content\_\-test\_\-cached (cache\_\-entry\_\-t $\ast$ {\em pentry\_\-inode}, cache\_\-content\_\-client\_\-t $\ast$ {\em pclient}, fsal\_\-op\_\-context\_\-t $\ast$ {\em pcontext}, cache\_\-content\_\-status\_\-t $\ast$ {\em pstatus})}\label{cache__content__misc_8c_a17}


cache\_\-content\_\-test\_\-cached: Tests if a given pentry\_\-inode has already an associated data cache

Tests if a given pentry\_\-inode has already an associated data cache. This is useful to recover data from a data cache built by a former server instance.

\begin{Desc}
\item[Parameters:]
\begin{description}
\item[{\em pentry\_\-inode}][IN] entry in cache\_\-inode layer for this file. \item[{\em pclient}][IN] ressource allocated by the client for the nfs management. \item[{\em pcontext}][IN] the related FSAL Context  [OUT] returned status.\end{description}
\end{Desc}
\begin{Desc}
\item[Returns:]CACHE\_\-CONTENT\_\-SUCCESS if entry is found, CACHE\_\-CONTENT\_\-NOT\_\-FOUND if not found \end{Desc}


Definition at line 773 of file cache\_\-content\_\-misc.c.

References cache\_\-content\_\-create\_\-name().\index{cache_content_misc.c@{cache\_\-content\_\-misc.c}!cache_content_valid@{cache\_\-content\_\-valid}}
\index{cache_content_valid@{cache\_\-content\_\-valid}!cache_content_misc.c@{cache\_\-content\_\-misc.c}}
\subsubsection{\setlength{\rightskip}{0pt plus 5cm}cache\_\-content\_\-status\_\-t cache\_\-content\_\-valid (cache\_\-content\_\-entry\_\-t $\ast$ {\em pentry}, cache\_\-inode\_\-op\_\-t {\em op}, cache\_\-content\_\-client\_\-t $\ast$ {\em pclient})}\label{cache__content__misc_8c_a14}


cache\_\-content\_\-valid: validates an entry to update its garbagge status.

Validates an error to update its garbagge status. Entry is supposed to be locked when this function is called !!

\begin{Desc}
\item[Parameters:]
\begin{description}
\item[{\em pentry}][INOUT] entry to be validated. \item[{\em op}][IN] can be set to CACHE\_\-INODE\_\-OP\_\-GET or CACHE\_\-INODE\_\-OP\_\-SET to show the type of operation done. \item[{\em pclient}][INOUT] ressource allocated by the client for the nfs management.\end{description}
\end{Desc}
\begin{Desc}
\item[Returns:]CACHE\_\-INODE\_\-SUCCESS if successful \par
 

CACHE\_\-INODE\_\-LRU\_\-ERROR if an errorr occured in LRU management. \end{Desc}


Definition at line 560 of file cache\_\-content\_\-misc.c.

Referenced by cache\_\-content\_\-crash\_\-recover(), and cache\_\-content\_\-rdwr().\index{cache_content_misc.c@{cache\_\-content\_\-misc.c}!HashFileID4@{HashFileID4}}
\index{HashFileID4@{HashFileID4}!cache_content_misc.c@{cache\_\-content\_\-misc.c}}
\subsubsection{\setlength{\rightskip}{0pt plus 5cm}short Hash\-File\-ID4 (u\_\-int64\_\-t {\em fileid4})}\label{cache__content__misc_8c_a3}




Definition at line 118 of file cache\_\-content\_\-misc.c.

References ALPHABET\_\-LEN.

Referenced by cache\_\-content\_\-create\_\-name(), and cache\_\-content\_\-get\_\-datapath().

\subsection{Variable Documentation}
\index{cache_content_misc.c@{cache\_\-content\_\-misc.c}!cache_content_dir_errno@{cache\_\-content\_\-dir\_\-errno}}
\index{cache_content_dir_errno@{cache\_\-content\_\-dir\_\-errno}!cache_content_misc.c@{cache\_\-content\_\-misc.c}}
\subsubsection{\setlength{\rightskip}{0pt plus 5cm}unsigned int {\bf cache\_\-content\_\-dir\_\-errno}}\label{cache__content__misc_8c_a2}




Definition at line 111 of file cache\_\-content\_\-misc.c.

Referenced by cache\_\-content\_\-emergency\_\-flush(), cache\_\-content\_\-local\_\-cache\_\-dir\_\-iter(), and cache\_\-content\_\-local\_\-cache\_\-opendir().