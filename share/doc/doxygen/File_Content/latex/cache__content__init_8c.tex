\section{cache\_\-content\_\-init.c File Reference}
\label{cache__content__init_8c}\index{cache\_\-content\_\-init.c@{cache\_\-content\_\-init.c}}
Management of the file content cache: initialisation.  


{\tt \#include \char`\"{}stuff\_\-alloc.h\char`\"{}}\par
{\tt \#include \char`\"{}LRU\_\-List.h\char`\"{}}\par
{\tt \#include \char`\"{}log\_\-functions.h\char`\"{}}\par
{\tt \#include \char`\"{}HashData.h\char`\"{}}\par
{\tt \#include \char`\"{}HashTable.h\char`\"{}}\par
{\tt \#include \char`\"{}fsal.h\char`\"{}}\par
{\tt \#include \char`\"{}cache\_\-inode.h\char`\"{}}\par
{\tt \#include \char`\"{}cache\_\-content.h\char`\"{}}\par
{\tt \#include $<$unistd.h$>$}\par
{\tt \#include $<$sys/types.h$>$}\par
{\tt \#include $<$sys/param.h$>$}\par
{\tt \#include $<$time.h$>$}\par
{\tt \#include $<$pthread.h$>$}\par
{\tt \#include $<$errno.h$>$}\par
{\tt \#include $<$string.h$>$}\par
\subsection*{Functions}
\begin{CompactItemize}
\item 
int {\bf cache\_\-content\_\-init} (cache\_\-content\_\-client\_\-parameter\_\-t param, cache\_\-content\_\-status\_\-t $\ast$pstatus)
\item 
int {\bf cache\_\-content\_\-init\_\-dir} (cache\_\-content\_\-client\_\-parameter\_\-t param, unsigned short export\_\-id)
\item 
int {\bf cache\_\-content\_\-client\_\-init} (cache\_\-content\_\-client\_\-t $\ast$pclient, cache\_\-content\_\-client\_\-parameter\_\-t param)
\end{CompactItemize}


\subsection{Detailed Description}
Management of the file content cache: initialisation. 

\begin{Desc}
\item[Author:]\end{Desc}
\begin{Desc}
\item[Author]deniel \end{Desc}
\begin{Desc}
\item[Date:]\end{Desc}
\begin{Desc}
\item[Date]2005/11/28 17:02:33 \end{Desc}
\begin{Desc}
\item[Version:]\end{Desc}
\begin{Desc}
\item[Revision]1.8 \end{Desc}
cache\_\-content.c : Management of the file content cache: initialisation.

\begin{Desc}
\item[Author:]\end{Desc}
\begin{Desc}
\item[Author]leibovic \end{Desc}
\begin{Desc}
\item[Date:]\end{Desc}
\begin{Desc}
\item[Date]2006/01/24 13:46:35 \end{Desc}
\begin{Desc}
\item[Version:]\end{Desc}
\begin{Desc}
\item[Revision]1.8 \end{Desc}
cache\_\-content.c : Management of the file content cache: initialisation. 

Definition in file {\bf cache\_\-content\_\-init.c}.

\subsection{Function Documentation}
\index{cache\_\-content\_\-init.c@{cache\_\-content\_\-init.c}!cache\_\-content\_\-client\_\-init@{cache\_\-content\_\-client\_\-init}}
\index{cache\_\-content\_\-client\_\-init@{cache\_\-content\_\-client\_\-init}!cache_content_init.c@{cache\_\-content\_\-init.c}}
\subsubsection[{cache\_\-content\_\-client\_\-init}]{\setlength{\rightskip}{0pt plus 5cm}int cache\_\-content\_\-client\_\-init (cache\_\-content\_\-client\_\-t $\ast$ {\em pclient}, \/  cache\_\-content\_\-client\_\-parameter\_\-t {\em param})}\label{cache__content__init_8c_fa9637b135c1e70734fcaaf042583824}


cache\_\-content\_\-client\_\-init: Init the ressource necessary for the cache content client.

Init the ressource necessary for the cache content client.

\begin{Desc}
\item[Parameters:]
\begin{description}
\item[{\em param}][IN] the parameter for this client \item[{\em pstatus}][OUT] pointer to buffer used to store the status for the operation.\end{description}
\end{Desc}
\begin{Desc}
\item[Returns:]0 if operation failed, -1 if failed. \end{Desc}


Definition at line 178 of file cache\_\-content\_\-init.c.\index{cache\_\-content\_\-init.c@{cache\_\-content\_\-init.c}!cache\_\-content\_\-init@{cache\_\-content\_\-init}}
\index{cache\_\-content\_\-init@{cache\_\-content\_\-init}!cache_content_init.c@{cache\_\-content\_\-init.c}}
\subsubsection[{cache\_\-content\_\-init}]{\setlength{\rightskip}{0pt plus 5cm}int cache\_\-content\_\-init (cache\_\-content\_\-client\_\-parameter\_\-t {\em param}, \/  cache\_\-content\_\-status\_\-t $\ast$ {\em pstatus})}\label{cache__content__init_8c_24a42a2805937468b2828d2835f52050}


cache\_\-inode\_\-init: Init the ressource necessary for the cache inode management.

Init the ressource necessary for the cache inode management.

\begin{Desc}
\item[Parameters:]
\begin{description}
\item[{\em param}][IN] the parameter for this cache. \item[{\em pstatus}][OUT] pointer to buffer used to store the status for the operation.\end{description}
\end{Desc}
\begin{Desc}
\item[Returns:]0 if operation failed, -1 if failed. \end{Desc}


Definition at line 124 of file cache\_\-content\_\-init.c.\index{cache\_\-content\_\-init.c@{cache\_\-content\_\-init.c}!cache\_\-content\_\-init\_\-dir@{cache\_\-content\_\-init\_\-dir}}
\index{cache\_\-content\_\-init\_\-dir@{cache\_\-content\_\-init\_\-dir}!cache_content_init.c@{cache\_\-content\_\-init.c}}
\subsubsection[{cache\_\-content\_\-init\_\-dir}]{\setlength{\rightskip}{0pt plus 5cm}int cache\_\-content\_\-init\_\-dir (cache\_\-content\_\-client\_\-parameter\_\-t {\em param}, \/  unsigned short {\em export\_\-id})}\label{cache__content__init_8c_d776d12ad1a4a6efc8ff8246c26d0572}


cache\_\-content\_\-init\_\-dir: Init the directory for caching entries from a given export id.

\begin{Desc}
\item[Parameters:]
\begin{description}
\item[{\em param}][IN] the parameter for this cache. \item[{\em export\_\-id}][IN] export id for the entries to be cached.\end{description}
\end{Desc}
\begin{Desc}
\item[Returns:]0 if ok, -1 otherwise. Errno will be set with the error's value. \end{Desc}


Definition at line 152 of file cache\_\-content\_\-init.c.