\section{Name handling functions.}
\label{group__FSALNameFunctions}\index{Name handling functions.@{Name handling functions.}}
\subsection*{Functions}
\begin{CompactItemize}
\item 
fsal\_\-status\_\-t {\bf FSAL\_\-str2name} (const char $\ast$string, fsal\_\-mdsize\_\-t in\_\-str\_\-maxlen, fsal\_\-name\_\-t $\ast$name)
\item 
fsal\_\-status\_\-t {\bf FSAL\_\-str2path} (char $\ast$string, fsal\_\-mdsize\_\-t in\_\-str\_\-maxlen, fsal\_\-path\_\-t $\ast$p\_\-path)
\item 
fsal\_\-status\_\-t {\bf FSAL\_\-name2str} (fsal\_\-name\_\-t $\ast$p\_\-name, char $\ast$string, fsal\_\-mdsize\_\-t out\_\-str\_\-maxlen)
\item 
fsal\_\-status\_\-t {\bf FSAL\_\-path2str} (fsal\_\-path\_\-t $\ast$p\_\-path, char $\ast$string, fsal\_\-mdsize\_\-t out\_\-str\_\-maxlen)
\item 
int {\bf FSAL\_\-namecmp} (fsal\_\-name\_\-t $\ast$p\_\-name1, fsal\_\-name\_\-t $\ast$p\_\-name2)
\item 
int {\bf FSAL\_\-pathcmp} (fsal\_\-path\_\-t $\ast$p\_\-path1, fsal\_\-path\_\-t $\ast$p\_\-path2)
\item 
fsal\_\-status\_\-t {\bf FSAL\_\-namecpy} (fsal\_\-name\_\-t $\ast$p\_\-tgt\_\-name, fsal\_\-name\_\-t $\ast$p\_\-src\_\-name)
\item 
fsal\_\-status\_\-t {\bf FSAL\_\-pathcpy} (fsal\_\-path\_\-t $\ast$p\_\-tgt\_\-path, fsal\_\-path\_\-t $\ast$p\_\-src\_\-path)
\item 
fsal\_\-status\_\-t {\bf FSAL\_\-buffdesc2name} (fsal\_\-buffdesc\_\-t $\ast$in\_\-buf, fsal\_\-name\_\-t $\ast$out\_\-name)
\item 
fsal\_\-status\_\-t {\bf FSAL\_\-buffdesc2path} (fsal\_\-buffdesc\_\-t $\ast$in\_\-buf, fsal\_\-path\_\-t $\ast$out\_\-path)
\item 
fsal\_\-status\_\-t {\bf FSAL\_\-path2buffdesc} (fsal\_\-path\_\-t $\ast$in\_\-path, fsal\_\-buffdesc\_\-t $\ast$out\_\-buff)
\item 
fsal\_\-status\_\-t {\bf FSAL\_\-name2buffdesc} (fsal\_\-name\_\-t $\ast$in\_\-name, fsal\_\-buffdesc\_\-t $\ast$out\_\-buff)
\end{CompactItemize}


\subsection{Detailed Description}
Those functions handle FS object names. 

\subsection{Function Documentation}
\index{FSALNameFunctions@{FSALName\-Functions}!FSAL_buffdesc2name@{FSAL\_\-buffdesc2name}}
\index{FSAL_buffdesc2name@{FSAL\_\-buffdesc2name}!FSALNameFunctions@{FSALName\-Functions}}
\subsubsection{\setlength{\rightskip}{0pt plus 5cm}fsal\_\-status\_\-t FSAL\_\-buffdesc2name (fsal\_\-buffdesc\_\-t $\ast$ {\em in\_\-buf}, fsal\_\-name\_\-t $\ast$ {\em out\_\-name})}\label{group__FSALNameFunctions_ga8}


FSAL\_\-buffdesc2name: Convert a buffer descriptor to an fsal name. 

Definition at line 305 of file fsal\_\-strings.c.

References FSAL\_\-str2name().\index{FSALNameFunctions@{FSALName\-Functions}!FSAL_buffdesc2path@{FSAL\_\-buffdesc2path}}
\index{FSAL_buffdesc2path@{FSAL\_\-buffdesc2path}!FSALNameFunctions@{FSALName\-Functions}}
\subsubsection{\setlength{\rightskip}{0pt plus 5cm}fsal\_\-status\_\-t FSAL\_\-buffdesc2path (fsal\_\-buffdesc\_\-t $\ast$ {\em in\_\-buf}, fsal\_\-path\_\-t $\ast$ {\em out\_\-path})}\label{group__FSALNameFunctions_ga9}


FSAL\_\-buffdesc2path: Convert a buffer descriptor to an fsal path. 

Definition at line 320 of file fsal\_\-strings.c.

References FSAL\_\-str2path().\index{FSALNameFunctions@{FSALName\-Functions}!FSAL_name2buffdesc@{FSAL\_\-name2buffdesc}}
\index{FSAL_name2buffdesc@{FSAL\_\-name2buffdesc}!FSALNameFunctions@{FSALName\-Functions}}
\subsubsection{\setlength{\rightskip}{0pt plus 5cm}fsal\_\-status\_\-t FSAL\_\-name2buffdesc (fsal\_\-name\_\-t $\ast$ {\em in\_\-name}, fsal\_\-buffdesc\_\-t $\ast$ {\em out\_\-buff})}\label{group__FSALNameFunctions_ga11}


FSAL\_\-name2buffdesc: Convert an fsal name to a buffer descriptor (utf8 like).

\begin{Desc}
\item[Parameters:]
\begin{description}
\item[{\em in\_\-name}](input): The fsal name to be converted. \item[{\em out\_\-buff}](output): Pointer to the buffer descriptor to be filled.\end{description}
\end{Desc}
\begin{Desc}
\item[Warning:]The buffer descriptor only contains pointers to the in\_\-name structure. Thus, if the in\_\-name structure is modified or destroyed, the out\_\-buff will be affected. \end{Desc}


Definition at line 373 of file fsal\_\-strings.c.\index{FSALNameFunctions@{FSALName\-Functions}!FSAL_name2str@{FSAL\_\-name2str}}
\index{FSAL_name2str@{FSAL\_\-name2str}!FSALNameFunctions@{FSALName\-Functions}}
\subsubsection{\setlength{\rightskip}{0pt plus 5cm}fsal\_\-status\_\-t FSAL\_\-name2str (fsal\_\-name\_\-t $\ast$ {\em p\_\-name}, char $\ast$ {\em string}, fsal\_\-mdsize\_\-t {\em out\_\-str\_\-maxlen})}\label{group__FSALNameFunctions_ga2}


FSAL\_\-name2str : converts an fsal\_\-name\_\-t to a char $\ast$ .

\begin{Desc}
\item[Parameters:]
\begin{description}
\item[{\em p\_\-name}](in, fsal\_\-name\_\-t $\ast$ ) Pointer to the structure to be converted. \item[{\em string}](out, char $\ast$) Address of the string to be filled. \item[{\em out\_\-str\_\-maxlen}](in, fsal\_\-mdsize\_\-t) Maximum size for the string to be filled.\end{description}
\end{Desc}
\begin{Desc}
\item[Returns:]major codes :\begin{itemize}
\item ERR\_\-FSAL\_\-FAULT\item ERR\_\-FSAL\_\-TOOSMALL \end{itemize}
\end{Desc}


Definition at line 147 of file fsal\_\-strings.c.\index{FSALNameFunctions@{FSALName\-Functions}!FSAL_namecmp@{FSAL\_\-namecmp}}
\index{FSAL_namecmp@{FSAL\_\-namecmp}!FSALNameFunctions@{FSALName\-Functions}}
\subsubsection{\setlength{\rightskip}{0pt plus 5cm}int FSAL\_\-namecmp (fsal\_\-name\_\-t $\ast$ {\em p\_\-name1}, fsal\_\-name\_\-t $\ast$ {\em p\_\-name2})}\label{group__FSALNameFunctions_ga4}


FSAL\_\-namecmp : compares two FSAL\_\-name\_\-t.

\begin{Desc}
\item[Returns:]The same value as strcmp. \end{Desc}


Definition at line 227 of file fsal\_\-strings.c.

Referenced by fsal\_\-internal\_\-get\-Info\-From\-Children\-List(), and FSAL\_\-lookup().\index{FSALNameFunctions@{FSALName\-Functions}!FSAL_namecpy@{FSAL\_\-namecpy}}
\index{FSAL_namecpy@{FSAL\_\-namecpy}!FSALNameFunctions@{FSALName\-Functions}}
\subsubsection{\setlength{\rightskip}{0pt plus 5cm}fsal\_\-status\_\-t FSAL\_\-namecpy (fsal\_\-name\_\-t $\ast$ {\em p\_\-tgt\_\-name}, fsal\_\-name\_\-t $\ast$ {\em p\_\-src\_\-name})}\label{group__FSALNameFunctions_ga6}


FSAL\_\-namecpy. copies a name to another.

\begin{Desc}
\item[Parameters:]
\begin{description}
\item[{\em p\_\-tgt\_\-name}]pointer to target name. \item[{\em p\_\-src\_\-name}]pointer to source name. \end{description}
\end{Desc}
\begin{Desc}
\item[Returns:]major code ERR\_\-FSAL\_\-FAULT, if tgt\_\-name is NULL. \end{Desc}


Definition at line 255 of file fsal\_\-strings.c.\index{FSALNameFunctions@{FSALName\-Functions}!FSAL_path2buffdesc@{FSAL\_\-path2buffdesc}}
\index{FSAL_path2buffdesc@{FSAL\_\-path2buffdesc}!FSALNameFunctions@{FSALName\-Functions}}
\subsubsection{\setlength{\rightskip}{0pt plus 5cm}fsal\_\-status\_\-t FSAL\_\-path2buffdesc (fsal\_\-path\_\-t $\ast$ {\em in\_\-path}, fsal\_\-buffdesc\_\-t $\ast$ {\em out\_\-buff})}\label{group__FSALNameFunctions_ga10}


FSAL\_\-path2buffdesc: Convert an fsal path to a buffer descriptor (utf8 like).

\begin{Desc}
\item[Parameters:]
\begin{description}
\item[{\em in\_\-path}](input): The fsal path to be converted. \item[{\em out\_\-buff}](output): Pointer to the buffer descriptor to be filled.\end{description}
\end{Desc}
\begin{Desc}
\item[Warning:]The buffer descriptor only contains pointers to the in\_\-path structure. Thus, if the in\_\-path structure is modified or destroyed, the out\_\-buff will be affected. \end{Desc}


Definition at line 345 of file fsal\_\-strings.c.\index{FSALNameFunctions@{FSALName\-Functions}!FSAL_path2str@{FSAL\_\-path2str}}
\index{FSAL_path2str@{FSAL\_\-path2str}!FSALNameFunctions@{FSALName\-Functions}}
\subsubsection{\setlength{\rightskip}{0pt plus 5cm}fsal\_\-status\_\-t FSAL\_\-path2str (fsal\_\-path\_\-t $\ast$ {\em p\_\-path}, char $\ast$ {\em string}, fsal\_\-mdsize\_\-t {\em out\_\-str\_\-maxlen})}\label{group__FSALNameFunctions_ga3}


FSAL\_\-path2str : converts an fsal\_\-path\_\-t to a char $\ast$ .

\begin{Desc}
\item[Parameters:]
\begin{description}
\item[{\em p\_\-path}](in, fsal\_\-path\_\-t $\ast$ ) The structure to be converted. \item[{\em string}](out, char $\ast$) Address of the string to be filled. \item[{\em out\_\-str\_\-maxlen}](in, fsal\_\-mdsize\_\-t) Maximum size for the string to be filled.\end{description}
\end{Desc}
\begin{Desc}
\item[Returns:]major codes :\begin{itemize}
\item ERR\_\-FSAL\_\-FAULT\item ERR\_\-FSAL\_\-TOOSMALL \end{itemize}
\end{Desc}


Definition at line 192 of file fsal\_\-strings.c.\index{FSALNameFunctions@{FSALName\-Functions}!FSAL_pathcmp@{FSAL\_\-pathcmp}}
\index{FSAL_pathcmp@{FSAL\_\-pathcmp}!FSALNameFunctions@{FSALName\-Functions}}
\subsubsection{\setlength{\rightskip}{0pt plus 5cm}int FSAL\_\-pathcmp (fsal\_\-path\_\-t $\ast$ {\em p\_\-path1}, fsal\_\-path\_\-t $\ast$ {\em p\_\-path2})}\label{group__FSALNameFunctions_ga5}


FSAL\_\-pathcmp : compares two FSAL\_\-path\_\-t.

\begin{Desc}
\item[Returns:]The same value as strcmp. \end{Desc}


Definition at line 240 of file fsal\_\-strings.c.\index{FSALNameFunctions@{FSALName\-Functions}!FSAL_pathcpy@{FSAL\_\-pathcpy}}
\index{FSAL_pathcpy@{FSAL\_\-pathcpy}!FSALNameFunctions@{FSALName\-Functions}}
\subsubsection{\setlength{\rightskip}{0pt plus 5cm}fsal\_\-status\_\-t FSAL\_\-pathcpy (fsal\_\-path\_\-t $\ast$ {\em p\_\-tgt\_\-path}, fsal\_\-path\_\-t $\ast$ {\em p\_\-src\_\-path})}\label{group__FSALNameFunctions_ga7}


FSAL\_\-pathcpy. copies a path to another.

\begin{Desc}
\item[Parameters:]
\begin{description}
\item[{\em p\_\-tgt\_\-name}]pointer to the target name. \item[{\em p\_\-src\_\-name}]pointer to the source name. \end{description}
\end{Desc}
\begin{Desc}
\item[Returns:]major code ERR\_\-FSAL\_\-FAULT, if tgt\_\-name is NULL. \end{Desc}


Definition at line 282 of file fsal\_\-strings.c.

Referenced by fsal\_\-internal\_\-get\-Path\-From\-Handle(), and FSAL\_\-rename().\index{FSALNameFunctions@{FSALName\-Functions}!FSAL_str2name@{FSAL\_\-str2name}}
\index{FSAL_str2name@{FSAL\_\-str2name}!FSALNameFunctions@{FSALName\-Functions}}
\subsubsection{\setlength{\rightskip}{0pt plus 5cm}fsal\_\-status\_\-t FSAL\_\-str2name (const char $\ast$ {\em string}, fsal\_\-mdsize\_\-t {\em in\_\-str\_\-maxlen}, fsal\_\-name\_\-t $\ast$ {\em name})}\label{group__FSALNameFunctions_ga0}


FSAL\_\-str2name : converts a char $\ast$ to an fsal\_\-name\_\-t.

\begin{Desc}
\item[Parameters:]
\begin{description}
\item[{\em string}](in, char $\ast$) Address of the string to be converted. \item[{\em in\_\-str\_\-maxlen}](in, fsal\_\-mdsize\_\-t) Maximum size for the string to be converted. \item[{\em name}](out, fsal\_\-name\_\-t $\ast$) The structure to be filled with the name.\end{description}
\end{Desc}
\begin{Desc}
\item[Returns:]major codes :\begin{itemize}
\item ERR\_\-FSAL\_\-FAULT\item ERR\_\-FSAL\_\-NAMETOOLONG \end{itemize}
\end{Desc}


Definition at line 47 of file fsal\_\-strings.c.

Referenced by FSAL\_\-buffdesc2name(), fsal\_\-internal\_\-get\-Path\-From\-Handle(), FSAL\_\-List\-XAttrs(), and FSAL\_\-readdir().\index{FSALNameFunctions@{FSALName\-Functions}!FSAL_str2path@{FSAL\_\-str2path}}
\index{FSAL_str2path@{FSAL\_\-str2path}!FSALNameFunctions@{FSALName\-Functions}}
\subsubsection{\setlength{\rightskip}{0pt plus 5cm}fsal\_\-status\_\-t FSAL\_\-str2path (char $\ast$ {\em string}, fsal\_\-mdsize\_\-t {\em in\_\-str\_\-maxlen}, fsal\_\-path\_\-t $\ast$ {\em p\_\-path})}\label{group__FSALNameFunctions_ga1}


FSAL\_\-str2path : converts a char $\ast$ to an fsal\_\-path\_\-t.

\begin{Desc}
\item[Parameters:]
\begin{description}
\item[{\em string}](in, char $\ast$) Address of the string to be converted. \item[{\em in\_\-str\_\-maxlen}](in, fsal\_\-mdsize\_\-t) Maximum size for the string to be converted. \item[{\em p\_\-path}](out, fsal\_\-path\_\-t $\ast$) The structure to be filled with the name.\end{description}
\end{Desc}
\begin{Desc}
\item[Returns:]major codes :\begin{itemize}
\item ERR\_\-FSAL\_\-FAULT\item ERR\_\-FSAL\_\-NAMETOOLONG \end{itemize}
\end{Desc}


Definition at line 97 of file fsal\_\-strings.c.

Referenced by FSAL\_\-buffdesc2path(), fsal\_\-internal\_\-get\-Path\-From\-Handle(), and FSAL\_\-readlink().