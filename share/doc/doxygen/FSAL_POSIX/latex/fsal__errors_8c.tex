\section{fsal\_\-errors.c File Reference}
\label{fsal__errors_8c}\index{fsal\_\-errors.c@{fsal\_\-errors.c}}
Routines for handling errors.  


{\tt \#include \char`\"{}fsal.h\char`\"{}}\par
\subsection*{Functions}
\begin{CompactItemize}
\item 
fsal\_\-boolean\_\-t {\bf fsal\_\-is\_\-retryable} (fsal\_\-status\_\-t status)
\end{CompactItemize}


\subsection{Detailed Description}
Routines for handling errors. 

\begin{Desc}
\item[Author:]\end{Desc}
\begin{Desc}
\item[Author]leibovic \end{Desc}
\begin{Desc}
\item[Date:]\end{Desc}
\begin{Desc}
\item[Date]2005/07/27 13:30:26 \end{Desc}
\begin{Desc}
\item[Version:]\end{Desc}
\begin{Desc}
\item[Revision]1.3 \end{Desc}


Definition in file {\bf fsal\_\-errors.c}.

\subsection{Function Documentation}
\index{fsal\_\-errors.c@{fsal\_\-errors.c}!fsal\_\-is\_\-retryable@{fsal\_\-is\_\-retryable}}
\index{fsal\_\-is\_\-retryable@{fsal\_\-is\_\-retryable}!fsal_errors.c@{fsal\_\-errors.c}}
\subsubsection[{fsal\_\-is\_\-retryable}]{\setlength{\rightskip}{0pt plus 5cm}fsal\_\-boolean\_\-t fsal\_\-is\_\-retryable (fsal\_\-status\_\-t {\em status})}\label{fsal__errors_8c_dd52bcb75d7fb46f9c0355f264a6689f}


fsal\_\-is\_\-retryable: Indicates if an FSAL error is retryable, i.e. if the caller has a chance of succeeding if it tries to call again the function that returned such an error code.

\begin{Desc}
\item[Parameters:]
\begin{description}
\item[{\em status(input),:}]The fsal status whom retryability is to be tested.\end{description}
\end{Desc}
\begin{Desc}
\item[Returns:]- TRUE if the error is retryable.\begin{itemize}
\item FALSE if the error is NOT retryable. \end{itemize}
\end{Desc}


\begin{Desc}
\item[{\bf Todo}]: ERR\_\-FSAL\_\-DELAY : The only retryable error ? \end{Desc}


Definition at line 34 of file fsal\_\-errors.c.

Referenced by fsal\_\-increment\_\-nbcall().