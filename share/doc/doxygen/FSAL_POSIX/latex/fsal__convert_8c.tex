\section{fsal\_\-convert.c File Reference}
\label{fsal__convert_8c}\index{fsal\_\-convert.c@{fsal\_\-convert.c}}
HPSS-FSAL type translation functions.  


{\tt \#include \char`\"{}fsal\_\-convert.h\char`\"{}}\par
{\tt \#include \char`\"{}fsal\_\-internal.h\char`\"{}}\par
{\tt \#include $<$sys/types.h$>$}\par
{\tt \#include $<$sys/stat.h$>$}\par
{\tt \#include $<$errno.h$>$}\par
{\tt \#include $<$string.h$>$}\par
\subsection*{Defines}
\begin{CompactItemize}
\item 
\#define \textbf{MAX\_\-2}(x, y)~( (x) $>$ (y) ? (x) : (y) )\label{fsal__convert_8c_29bc465820d0719bc24cb339e75806d0}

\item 
\#define \textbf{MAX\_\-3}(x, y, z)~( (x) $>$ (y) ? MAX\_\-2((x),(z)) : MAX\_\-2((y),(z)) )\label{fsal__convert_8c_e28623bae2af7d233caa8038601aa853}

\end{CompactItemize}
\subsection*{Functions}
\begin{CompactItemize}
\item 
int {\bf posix2fsal\_\-error} (int posix\_\-errorcode)
\item 
int {\bf fsal2posix\_\-testperm} (fsal\_\-accessflags\_\-t testperm)
\item 
int {\bf fsal2posix\_\-openflags} (fsal\_\-openflags\_\-t fsal\_\-flags, char $\ast$p\_\-posix\_\-flags)
\item 
mode\_\-t {\bf fsal2unix\_\-mode} (fsal\_\-accessmode\_\-t fsal\_\-mode)
\item 
fsal\_\-accessmode\_\-t {\bf unix2fsal\_\-mode} (mode\_\-t unix\_\-mode)
\item 
fsal\_\-nodetype\_\-t {\bf posix2fsal\_\-type} (mode\_\-t posix\_\-type\_\-in)
\item 
fsal\_\-time\_\-t \textbf{posix2fsal\_\-time} (time\_\-t tsec)\label{fsal__convert_8c_605bbbb2ba51449ed78266ad86d2d772}

\item 
fsal\_\-fsid\_\-t \textbf{posix2fsal\_\-fsid} (dev\_\-t posix\_\-devid)\label{fsal__convert_8c_33a5c004f7b33df3dd462d9325305277}

\item 
fsal\_\-dev\_\-t \textbf{posix2fsal\_\-devt} (dev\_\-t posix\_\-devid)\label{fsal__convert_8c_aca862341c53d983d23b597cc082d69d}

\item 
fsal\_\-status\_\-t \textbf{posix2fsal\_\-attributes} (struct stat $\ast$p\_\-buffstat, fsal\_\-attrib\_\-list\_\-t $\ast$p\_\-fsalattr\_\-out)\label{fsal__convert_8c_add7b590c15483d42c8b635e4810f506}

\item 
fsal\_\-status\_\-t {\bf posixdb2fsal\_\-error} (fsal\_\-posixdb\_\-status\_\-t statusdb)
\end{CompactItemize}


\subsection{Detailed Description}
HPSS-FSAL type translation functions. 

\begin{Desc}
\item[Author:]\end{Desc}
\begin{Desc}
\item[Author]leibovic \end{Desc}
\begin{Desc}
\item[Date:]\end{Desc}
\begin{Desc}
\item[Date]2006/01/17 15:53:39 \end{Desc}
\begin{Desc}
\item[Version:]\end{Desc}
\begin{Desc}
\item[Revision]1.31 \end{Desc}


Definition in file {\bf fsal\_\-convert.c}.

\subsection{Function Documentation}
\index{fsal\_\-convert.c@{fsal\_\-convert.c}!fsal2posix\_\-openflags@{fsal2posix\_\-openflags}}
\index{fsal2posix\_\-openflags@{fsal2posix\_\-openflags}!fsal_convert.c@{fsal\_\-convert.c}}
\subsubsection[{fsal2posix\_\-openflags}]{\setlength{\rightskip}{0pt plus 5cm}int fsal2posix\_\-openflags (fsal\_\-openflags\_\-t {\em fsal\_\-flags}, \/  char $\ast$ {\em p\_\-posix\_\-flags})}\label{fsal__convert_8c_d6abc88341bcfc03ff48c1ea7e78166d}


pen\_\-by\_\-n fsal2posix\_\-openflags: Convert FSAL open flags to Posix open flags.

\begin{Desc}
\item[Parameters:]
\begin{description}
\item[{\em fsal\_\-flags}](input): The FSAL open flags to be translated. \item[{\em p\_\-hpss\_\-flags}](output): Pointer to the POSIX open flags.\end{description}
\end{Desc}
\begin{Desc}
\item[Returns:]- ERR\_\-FSAL\_\-NO\_\-ERROR (no error).\begin{itemize}
\item ERR\_\-FSAL\_\-FAULT (p\_\-hpss\_\-flags is a NULL pointer).\item ERR\_\-FSAL\_\-INVAL (invalid or incompatible input flags). \end{itemize}
\end{Desc}


Definition at line 205 of file fsal\_\-convert.c.

Referenced by FSAL\_\-open().\index{fsal\_\-convert.c@{fsal\_\-convert.c}!fsal2posix\_\-testperm@{fsal2posix\_\-testperm}}
\index{fsal2posix\_\-testperm@{fsal2posix\_\-testperm}!fsal_convert.c@{fsal\_\-convert.c}}
\subsubsection[{fsal2posix\_\-testperm}]{\setlength{\rightskip}{0pt plus 5cm}int fsal2posix\_\-testperm (fsal\_\-accessflags\_\-t {\em testperm})}\label{fsal__convert_8c_1c6839e19758c2f3a6fa4a75121ce552}


fsal2posix\_\-testperm: Convert FSAL permission flags to Posix permission flags.

\begin{Desc}
\item[Parameters:]
\begin{description}
\item[{\em testperm}](input): The FSAL permission flags to be tested.\end{description}
\end{Desc}
\begin{Desc}
\item[Returns:]The POSIX permission flags to be tested. \end{Desc}


Definition at line 177 of file fsal\_\-convert.c.\index{fsal\_\-convert.c@{fsal\_\-convert.c}!fsal2unix\_\-mode@{fsal2unix\_\-mode}}
\index{fsal2unix\_\-mode@{fsal2unix\_\-mode}!fsal_convert.c@{fsal\_\-convert.c}}
\subsubsection[{fsal2unix\_\-mode}]{\setlength{\rightskip}{0pt plus 5cm}mode\_\-t fsal2unix\_\-mode (fsal\_\-accessmode\_\-t {\em fsal\_\-mode})}\label{fsal__convert_8c_d39aceba2aa5bbef96f35a802e463222}


fsal2unix\_\-mode: Convert FSAL mode to posix mode.

\begin{Desc}
\item[Parameters:]
\begin{description}
\item[{\em fsal\_\-mode}](input): The FSAL mode to be translated.\end{description}
\end{Desc}
\begin{Desc}
\item[Returns:]The posix mode associated to fsal\_\-mode. \end{Desc}


Definition at line 263 of file fsal\_\-convert.c.

Referenced by FSAL\_\-create(), FSAL\_\-mkdir(), FSAL\_\-mknode(), and FSAL\_\-setattrs().\index{fsal\_\-convert.c@{fsal\_\-convert.c}!posix2fsal\_\-error@{posix2fsal\_\-error}}
\index{posix2fsal\_\-error@{posix2fsal\_\-error}!fsal_convert.c@{fsal\_\-convert.c}}
\subsubsection[{posix2fsal\_\-error}]{\setlength{\rightskip}{0pt plus 5cm}int posix2fsal\_\-error (int {\em posix\_\-errorcode})}\label{fsal__convert_8c_d420f004b68f6720ab45980298177bd9}


posix2fsal\_\-error : Convert POSIX error codes to FSAL error codes.

\begin{Desc}
\item[Parameters:]
\begin{description}
\item[{\em posix\_\-errorcode}](input): The error code returned from POSIX.\end{description}
\end{Desc}
\begin{Desc}
\item[Returns:]The FSAL error code associated to posix\_\-errorcode. \end{Desc}


\begin{Desc}
\item[{\bf Todo}]: The EBADF error also happens when file is opened for reading, and we try writting in it. In this case, we return ERR\_\-FSAL\_\-NOT\_\-OPENED, but it doesn't seems to be a correct error translation. \end{Desc}


\begin{Desc}
\item[Warning:]AIX returns EEXIST where BSD uses ENOTEMPTY; We want ENOTEMPTY to be interpreted anyway on AIX plateforms. Thus, we explicitely write its value (87). \end{Desc}


Definition at line 41 of file fsal\_\-convert.c.

Referenced by FSAL\_\-close(), FSAL\_\-closedir(), FSAL\_\-create(), FSAL\_\-dynamic\_\-fsinfo(), FSAL\_\-link(), FSAL\_\-lookup(), FSAL\_\-mkdir(), FSAL\_\-mknode(), FSAL\_\-open(), FSAL\_\-opendir(), FSAL\_\-rcp(), FSAL\_\-read(), FSAL\_\-readdir(), FSAL\_\-readlink(), FSAL\_\-rename(), FSAL\_\-setattrs(), FSAL\_\-symlink(), FSAL\_\-truncate(), FSAL\_\-unlink(), and FSAL\_\-write().\index{fsal\_\-convert.c@{fsal\_\-convert.c}!posix2fsal\_\-type@{posix2fsal\_\-type}}
\index{posix2fsal\_\-type@{posix2fsal\_\-type}!fsal_convert.c@{fsal\_\-convert.c}}
\subsubsection[{posix2fsal\_\-type}]{\setlength{\rightskip}{0pt plus 5cm}fsal\_\-nodetype\_\-t posix2fsal\_\-type (mode\_\-t {\em posix\_\-type\_\-in})}\label{fsal__convert_8c_ef52f65960ee7556010a7082de7b097e}


posix2fsal\_\-type: Convert posix object type to FSAL node type.

\begin{Desc}
\item[Parameters:]
\begin{description}
\item[{\em posix\_\-type\_\-in}](input): The POSIX object type.\end{description}
\end{Desc}
\begin{Desc}
\item[Returns:]- The FSAL node type associated to posix\_\-type\_\-in.\begin{itemize}
\item -1 if the input type is unknown. \end{itemize}
\end{Desc}


Definition at line 331 of file fsal\_\-convert.c.

Referenced by fsal\_\-internal\_\-posix2posixdb\_\-fileinfo(), and FSAL\_\-lookup().\index{fsal\_\-convert.c@{fsal\_\-convert.c}!posixdb2fsal\_\-error@{posixdb2fsal\_\-error}}
\index{posixdb2fsal\_\-error@{posixdb2fsal\_\-error}!fsal_convert.c@{fsal\_\-convert.c}}
\subsubsection[{posixdb2fsal\_\-error}]{\setlength{\rightskip}{0pt plus 5cm}fsal\_\-status\_\-t posixdb2fsal\_\-error (fsal\_\-posixdb\_\-status\_\-t {\em statusdb})}\label{fsal__convert_8c_9ea1023587982b3087afb63e2afe32b4}


Concert fsal\_\-posixdb\_\-status\_\-t to fsal\_\-status\_\-t 

Definition at line 518 of file fsal\_\-convert.c.

Referenced by fsal\_\-internal\_\-getInfoFromChildrenList(), fsal\_\-internal\_\-getInfoFromName(), fsal\_\-internal\_\-getPathFromHandle(), FSAL\_\-readdir(), FSAL\_\-rename(), and FSAL\_\-unlink().\index{fsal\_\-convert.c@{fsal\_\-convert.c}!unix2fsal\_\-mode@{unix2fsal\_\-mode}}
\index{unix2fsal\_\-mode@{unix2fsal\_\-mode}!fsal_convert.c@{fsal\_\-convert.c}}
\subsubsection[{unix2fsal\_\-mode}]{\setlength{\rightskip}{0pt plus 5cm}fsal\_\-accessmode\_\-t unix2fsal\_\-mode (mode\_\-t {\em unix\_\-mode})}\label{fsal__convert_8c_af1ff789aff0ed7114b8e525e01e268c}


unix2fsal\_\-mode: Convert posix mode to FSAL mode.

\begin{Desc}
\item[Parameters:]
\begin{description}
\item[{\em unix\_\-mode}](input): The posix mode to be translated.\end{description}
\end{Desc}
\begin{Desc}
\item[Returns:]The FSAL mode associated to unix\_\-mode. \end{Desc}


Definition at line 296 of file fsal\_\-convert.c.