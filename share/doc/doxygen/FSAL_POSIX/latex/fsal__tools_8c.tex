\section{fsal\_\-tools.c File Reference}
\label{fsal__tools_8c}\index{fsal_tools.c@{fsal\_\-tools.c}}
miscelaneous FSAL tools. 

{\tt \#include \char`\"{}fsal.h\char`\"{}}\par
{\tt \#include \char`\"{}fsal\_\-internal.h\char`\"{}}\par
{\tt \#include \char`\"{}fsal\_\-convert.h\char`\"{}}\par
{\tt \#include \char`\"{}config\_\-parsing.h\char`\"{}}\par
{\tt \#include $<$string.h$>$}\par
\subsection*{Defines}
\begin{CompactItemize}
\item 
\#define {\bf STRCMP}\ strcasecmp\label{fsal__tools_8c_a0}

\item 
\#define {\bf MAGIC}\ 0x\-ABCD1234\label{fsal__tools_8c_a1}

\end{CompactItemize}
\subsection*{Functions}
\begin{CompactItemize}
\item 
char $\ast$ {\bf FSAL\_\-Get\-FSName} ()\label{fsal__tools_8c_a2}

\item 
int {\bf FSAL\_\-handlecmp} (fsal\_\-handle\_\-t $\ast$handle1, fsal\_\-handle\_\-t $\ast$handle2, fsal\_\-status\_\-t $\ast$status)
\item 
unsigned int {\bf FSAL\_\-Handle\_\-to\_\-Hash\-Index} (fsal\_\-handle\_\-t $\ast$p\_\-handle, unsigned int cookie, unsigned int alphabet\_\-len, unsigned int index\_\-size)
\item 
unsigned int {\bf FSAL\_\-Handle\_\-to\_\-RBTIndex} (fsal\_\-handle\_\-t $\ast$p\_\-handle, unsigned int cookie)\label{fsal__tools_8c_a5}

\item 
fsal\_\-status\_\-t {\bf FSAL\_\-Digest\-Handle} (fsal\_\-export\_\-context\_\-t $\ast$p\_\-expcontext, fsal\_\-digesttype\_\-t output\_\-type, fsal\_\-handle\_\-t $\ast$p\_\-in\_\-fsal\_\-handle, caddr\_\-t out\_\-buff)
\item 
fsal\_\-status\_\-t {\bf FSAL\_\-Expand\-Handle} (fsal\_\-export\_\-context\_\-t $\ast$p\_\-expcontext, fsal\_\-digesttype\_\-t in\_\-type, caddr\_\-t in\_\-buff, fsal\_\-handle\_\-t $\ast$p\_\-out\_\-fsal\_\-handle)
\item 
fsal\_\-status\_\-t {\bf FSAL\_\-Set\-Default\_\-FSAL\_\-parameter} (fsal\_\-parameter\_\-t $\ast$out\_\-parameter)
\item 
fsal\_\-status\_\-t {\bf FSAL\_\-Set\-Default\_\-FS\_\-common\_\-parameter} (fsal\_\-parameter\_\-t $\ast$out\_\-parameter)\label{fsal__tools_8c_a9}

\item 
fsal\_\-status\_\-t {\bf FSAL\_\-Set\-Default\_\-FS\_\-specific\_\-parameter} (fsal\_\-parameter\_\-t $\ast$out\_\-parameter)\label{fsal__tools_8c_a10}

\item 
fsal\_\-status\_\-t {\bf FSAL\_\-load\_\-FSAL\_\-parameter\_\-from\_\-conf} (config\_\-file\_\-t in\_\-config, fsal\_\-parameter\_\-t $\ast$out\_\-parameter)
\item 
fsal\_\-status\_\-t {\bf FSAL\_\-load\_\-FS\_\-common\_\-parameter\_\-from\_\-conf} (config\_\-file\_\-t in\_\-config, fsal\_\-parameter\_\-t $\ast$out\_\-parameter)\label{fsal__tools_8c_a12}

\item 
fsal\_\-status\_\-t {\bf FSAL\_\-load\_\-FS\_\-specific\_\-parameter\_\-from\_\-conf} (config\_\-file\_\-t in\_\-config, fsal\_\-parameter\_\-t $\ast$out\_\-parameter)\label{fsal__tools_8c_a13}

\end{CompactItemize}


\subsection{Detailed Description}
miscelaneous FSAL tools. 

\begin{Desc}
\item[Author:]\begin{Desc}
\item[Author]leibovic \end{Desc}
\end{Desc}
\begin{Desc}
\item[Date:]\begin{Desc}
\item[Date]2006/01/17 14:20:07 \end{Desc}
\end{Desc}
\begin{Desc}
\item[Version:]\begin{Desc}
\item[Revision]1.26 \end{Desc}
\end{Desc}


Definition in file {\bf fsal\_\-tools.c}.

\subsection{Function Documentation}
\index{fsal_tools.c@{fsal\_\-tools.c}!FSAL_DigestHandle@{FSAL\_\-DigestHandle}}
\index{FSAL_DigestHandle@{FSAL\_\-DigestHandle}!fsal_tools.c@{fsal\_\-tools.c}}
\subsubsection{\setlength{\rightskip}{0pt plus 5cm}fsal\_\-status\_\-t FSAL\_\-Digest\-Handle (fsal\_\-export\_\-context\_\-t $\ast$ {\em p\_\-expcontext}, fsal\_\-digesttype\_\-t {\em output\_\-type}, fsal\_\-handle\_\-t $\ast$ {\em p\_\-in\_\-fsal\_\-handle}, caddr\_\-t {\em out\_\-buff})}\label{fsal__tools_8c_a6}


FSAL\_\-Digest\-Handle : Convert an fsal\_\-handle\_\-t to a buffer to be included into NFS handles, or another digest.

\begin{Desc}
\item[Parameters:]
\begin{description}
\item[{\em output\_\-type}](input): Indicates the type of digest to do. \item[{\em in\_\-fsal\_\-handle}](input): The handle to be converted to digest. \item[{\em out\_\-buff}](output): The buffer where the digest is to be stored.\end{description}
\end{Desc}
\begin{Desc}
\item[Returns:]The major code is ERR\_\-FSAL\_\-NO\_\-ERROR is no error occured. Else, it is a non null value. \end{Desc}


Definition at line 129 of file fsal\_\-tools.c.\index{fsal_tools.c@{fsal\_\-tools.c}!FSAL_ExpandHandle@{FSAL\_\-ExpandHandle}}
\index{FSAL_ExpandHandle@{FSAL\_\-ExpandHandle}!fsal_tools.c@{fsal\_\-tools.c}}
\subsubsection{\setlength{\rightskip}{0pt plus 5cm}fsal\_\-status\_\-t FSAL\_\-Expand\-Handle (fsal\_\-export\_\-context\_\-t $\ast$ {\em p\_\-expcontext}, fsal\_\-digesttype\_\-t {\em in\_\-type}, caddr\_\-t {\em in\_\-buff}, fsal\_\-handle\_\-t $\ast$ {\em p\_\-out\_\-fsal\_\-handle})}\label{fsal__tools_8c_a7}


FSAL\_\-Expand\-Handle : Convert a buffer extracted from NFS handles to an FSAL handle.

\begin{Desc}
\item[Parameters:]
\begin{description}
\item[{\em in\_\-type}](input): Indicates the type of digest to be expanded. \item[{\em in\_\-buff}](input): Pointer to the digest to be expanded. \item[{\em out\_\-fsal\_\-handle}](output): The handle built from digest.\end{description}
\end{Desc}
\begin{Desc}
\item[Returns:]The major code is ERR\_\-FSAL\_\-NO\_\-ERROR is no error occured. Else, it is a non null value. \end{Desc}


Definition at line 260 of file fsal\_\-tools.c.\index{fsal_tools.c@{fsal\_\-tools.c}!FSAL_Handle_to_HashIndex@{FSAL\_\-Handle\_\-to\_\-HashIndex}}
\index{FSAL_Handle_to_HashIndex@{FSAL\_\-Handle\_\-to\_\-HashIndex}!fsal_tools.c@{fsal\_\-tools.c}}
\subsubsection{\setlength{\rightskip}{0pt plus 5cm}unsigned int FSAL\_\-Handle\_\-to\_\-Hash\-Index (fsal\_\-handle\_\-t $\ast$ {\em p\_\-handle}, unsigned int {\em cookie}, unsigned int {\em alphabet\_\-len}, unsigned int {\em index\_\-size})}\label{fsal__tools_8c_a4}


FSAL\_\-Handle\_\-to\_\-Hash\-Index This function is used for hashing a FSAL handle in order to dispatch entries into the hash table array.

\begin{Desc}
\item[Parameters:]
\begin{description}
\item[{\em p\_\-handle}]The handle to be hashed \item[{\em cookie}]Makes it possible to have different hash value for the same handle, when cookie changes. \item[{\em alphabet\_\-len}]Parameter for polynomial hashing algorithm \item[{\em index\_\-size}]The range of hash value will be [0..index\_\-size-1]\end{description}
\end{Desc}
\begin{Desc}
\item[Returns:]The hash value \end{Desc}


Definition at line 80 of file fsal\_\-tools.c.\index{fsal_tools.c@{fsal\_\-tools.c}!FSAL_handlecmp@{FSAL\_\-handlecmp}}
\index{FSAL_handlecmp@{FSAL\_\-handlecmp}!fsal_tools.c@{fsal\_\-tools.c}}
\subsubsection{\setlength{\rightskip}{0pt plus 5cm}int FSAL\_\-handlecmp (fsal\_\-handle\_\-t $\ast$ {\em handle1}, fsal\_\-handle\_\-t $\ast$ {\em handle2}, fsal\_\-status\_\-t $\ast$ {\em status})}\label{fsal__tools_8c_a3}


FSAL\_\-handlecmp: Compare 2 handles.

\begin{Desc}
\item[Parameters:]
\begin{description}
\item[{\em handle1}](input): The first handle to be compared. \item[{\em handle2}](input): The second handle to be compared. \item[{\em status}](output): The status of the compare operation.\end{description}
\end{Desc}
\begin{Desc}
\item[Returns:]- 0 if handles are the same.\begin{itemize}
\item A non null value else.\item Segfault if status is a NULL pointer. \end{itemize}
\end{Desc}


Definition at line 48 of file fsal\_\-tools.c.

Referenced by FSAL\_\-rename().\index{fsal_tools.c@{fsal\_\-tools.c}!FSAL_load_FSAL_parameter_from_conf@{FSAL\_\-load\_\-FSAL\_\-parameter\_\-from\_\-conf}}
\index{FSAL_load_FSAL_parameter_from_conf@{FSAL\_\-load\_\-FSAL\_\-parameter\_\-from\_\-conf}!fsal_tools.c@{fsal\_\-tools.c}}
\subsubsection{\setlength{\rightskip}{0pt plus 5cm}fsal\_\-status\_\-t FSAL\_\-load\_\-FSAL\_\-parameter\_\-from\_\-conf (config\_\-file\_\-t {\em in\_\-config}, fsal\_\-parameter\_\-t $\ast$ {\em out\_\-parameter})}\label{fsal__tools_8c_a11}


FSAL\_\-load\_\-FSAL\_\-parameter\_\-from\_\-conf, FSAL\_\-load\_\-FS\_\-common\_\-parameter\_\-from\_\-conf, FSAL\_\-load\_\-FS\_\-specific\_\-parameter\_\-from\_\-conf:

Those functions initialize the FSAL init parameter structure from a configuration structure.

\begin{Desc}
\item[Parameters:]
\begin{description}
\item[{\em in\_\-config}](input): Structure that represents the parsed configuration file. \item[{\em out\_\-parameter}](ouput) FSAL initialization structure filled according to the configuration file given as parameter.\end{description}
\end{Desc}
\begin{Desc}
\item[Returns:]ERR\_\-FSAL\_\-NO\_\-ERROR (no error) , ERR\_\-FSAL\_\-NOENT (missing a mandatory stanza in config file), ERR\_\-FSAL\_\-INVAL (invalid parameter), ERR\_\-FSAL\_\-SERVERFAULT (unexpected error) ERR\_\-FSAL\_\-FAULT (null pointer given as parameter), \end{Desc}


Definition at line 420 of file fsal\_\-tools.c.\index{fsal_tools.c@{fsal\_\-tools.c}!FSAL_SetDefault_FSAL_parameter@{FSAL\_\-SetDefault\_\-FSAL\_\-parameter}}
\index{FSAL_SetDefault_FSAL_parameter@{FSAL\_\-SetDefault\_\-FSAL\_\-parameter}!fsal_tools.c@{fsal\_\-tools.c}}
\subsubsection{\setlength{\rightskip}{0pt plus 5cm}fsal\_\-status\_\-t FSAL\_\-Set\-Default\_\-FSAL\_\-parameter (fsal\_\-parameter\_\-t $\ast$ {\em out\_\-parameter})}\label{fsal__tools_8c_a8}


Those routines set the default parameters for FSAL init structure. \begin{Desc}
\item[Returns:]ERR\_\-FSAL\_\-NO\_\-ERROR (no error) , ERR\_\-FSAL\_\-FAULT (null pointer given as parameter), ERR\_\-FSAL\_\-SERVERFAULT (unexpected error) \end{Desc}


Definition at line 309 of file fsal\_\-tools.c.