\section{shell.c File Reference}
\label{shell_8c}\index{shell.c@{shell.c}}
Internal routines for the shell. 

{\tt \#include \char`\"{}shell.h\char`\"{}}\par
{\tt \#include \char`\"{}shell\_\-utils.h\char`\"{}}\par
{\tt \#include \char`\"{}shell\_\-vars.h\char`\"{}}\par
{\tt \#include \char`\"{}log\_\-functions.h\char`\"{}}\par
{\tt \#include \char`\"{}commands.h\char`\"{}}\par
{\tt \#include \char`\"{}cmd\_\-tools.h\char`\"{}}\par
{\tt \#include \char`\"{}Buddy\-Malloc.h\char`\"{}}\par
{\tt \#include \char`\"{}stuff\_\-alloc.h\char`\"{}}\par
{\tt \#include $<$unistd.h$>$}\par
{\tt \#include $<$string.h$>$}\par
{\tt \#include $<$stdio.h$>$}\par
{\tt \#include $<$readline/readline.h$>$}\par
{\tt \#include $<$readline/history.h$>$}\par
{\tt \#include $<$sys/time.h$>$}\par
\subsection*{Defines}
\begin{CompactItemize}
\item 
\#define {\bf MAX\_\-OUTPUT\_\-LEN}\ (1024$\ast$1024)
\item 
\#define {\bf TRACEBUFFSIZE}\ 1024
\item 
\#define {\bf PROMPTSIZE}\ 64
\item 
\#define {\bf P}(\_\-mutex\_\-)\ pthread\_\-mutex\_\-lock( \&\_\-mutex\_\- )
\item 
\#define {\bf V}(\_\-mutex\_\-)\ pthread\_\-mutex\_\-unlock( \&\_\-mutex\_\- )
\end{CompactItemize}
\subsection*{Functions}
\begin{CompactItemize}
\item 
int {\bf shell\_\-Barrier\-Init} (int nb\_\-threads)
\item 
int {\bf shell\_\-Init} (int {\bf verbose}, char $\ast$input\_\-file, char $\ast$prompt, int shell\_\-index)
\item 
int {\bf shell\_\-Launch} ()
\item 
int {\bf shell\_\-Parse\-Line} (char $\ast$in\_\-out\_\-line, char $\ast$$\ast$out\_\-arglist, int $\ast$p\_\-argcount)
\item 
int {\bf shell\_\-Solve\-Args} (int argc, char $\ast$$\ast$in\_\-out\_\-argv, int $\ast$out\_\-allocated)
\item 
void {\bf shell\_\-Clean\-Args} (int argc, char $\ast$$\ast$in\_\-out\_\-argv, int $\ast$in\_\-allocated)
\item 
int {\bf shell\_\-Execute} (int argc, char $\ast$$\ast$argv, FILE $\ast$output)
\item 
void {\bf shell\_\-Print\-Error} ({\bf shell\_\-state\_\-t} $\ast$context, char $\ast$error\_\-msg)
\item 
void {\bf shell\_\-Print\-Trace} ({\bf shell\_\-state\_\-t} $\ast$context, char $\ast$msg)
\item 
int {\bf shell\_\-Set\-Layer} ({\bf shell\_\-state\_\-t} $\ast$context, char $\ast$layer\_\-name)
\item 
{\bf layer\_\-def\_\-t} $\ast$ {\bf shell\_\-Get\-Layer} ({\bf shell\_\-state\_\-t} $\ast$context)
\item 
int {\bf shell\_\-Set\-Status} ({\bf shell\_\-state\_\-t} $\ast$context, int returned\_\-status)
\item 
int {\bf shell\_\-Get\-Status} ({\bf shell\_\-state\_\-t} $\ast$context)
\item 
int {\bf shell\_\-Set\-Verbose} ({\bf shell\_\-state\_\-t} $\ast$context, char $\ast$str\_\-verbose)
\item 
int {\bf shell\_\-Get\-Verbose} ({\bf shell\_\-state\_\-t} $\ast$context)
\item 
int {\bf shell\_\-Set\-Dbg\-Lvl} ({\bf shell\_\-state\_\-t} $\ast$context, char $\ast$str\_\-debug\_\-level)
\item 
int {\bf shell\_\-Get\-Dbg\-Lvl} ({\bf shell\_\-state\_\-t} $\ast$context)
\item 
int {\bf shell\_\-Set\-Input} ({\bf shell\_\-state\_\-t} $\ast$context, char $\ast$file\_\-name)
\item 
FILE $\ast$ {\bf shell\_\-Get\-Input\-Stream} ({\bf shell\_\-state\_\-t} $\ast$context)
\item 
int {\bf shell\_\-Set\-Prompt} ({\bf shell\_\-state\_\-t} $\ast$context, char $\ast$str\_\-prompt)
\item 
char $\ast$ {\bf shell\_\-Get\-Prompt} ({\bf shell\_\-state\_\-t} $\ast$context)
\item 
int {\bf shell\_\-Set\-Shell\-Id} ({\bf shell\_\-state\_\-t} $\ast$context, int shell\_\-index)
\item 
int {\bf shell\_\-Set\-Line} ({\bf shell\_\-state\_\-t} $\ast$context, int lineno)
\item 
int {\bf shell\_\-Get\-Line} ({\bf shell\_\-state\_\-t} $\ast$context)
\item 
int {\bf shellcmd\_\-help} (int argc, char $\ast$$\ast$argv, FILE $\ast$output)
\item 
int {\bf shellcmd\_\-if} (int argc, char $\ast$$\ast$argv, FILE $\ast$output)
\item 
int {\bf shellcmd\_\-interactive} (int argc, char $\ast$$\ast$argv, FILE $\ast$output)
\item 
int {\bf shellcmd\_\-set} (int argc, char $\ast$$\ast$argv, FILE $\ast$output)
\item 
int {\bf shellcmd\_\-unset} (int argc, char $\ast$$\ast$argv, FILE $\ast$output)
\item 
int {\bf shellcmd\_\-print} (int argc, char $\ast$$\ast$argv, FILE $\ast$output)
\item 
int {\bf shellcmd\_\-varlist} (int argc, char $\ast$$\ast$argv, FILE $\ast$output)
\item 
int {\bf shellcmd\_\-time} (int argc, char $\ast$$\ast$argv, FILE $\ast$output)
\item 
int {\bf shellcmd\_\-quit} (int argc, char $\ast$$\ast$argv, FILE $\ast$output)
\item 
int {\bf shellcmd\_\-barrier} (int argc, char $\ast$$\ast$argv, FILE $\ast$output)
\end{CompactItemize}


\subsection{Detailed Description}
Internal routines for the shell. 

\begin{Desc}
\item[Author:]\begin{Desc}
\item[Author]leibovic \end{Desc}
\end{Desc}
\begin{Desc}
\item[Date:]\begin{Desc}
\item[Date]2006/02/24 08:33:44 \end{Desc}
\end{Desc}
\begin{Desc}
\item[Version:]\begin{Desc}
\item[Revision]1.20 \end{Desc}
\end{Desc}
\begin{Desc}
\item[Log]{\bf shell.c}{\rm (p.\,\pageref{shell_8c})},v \end{Desc}
Revision 1.20 2006/02/24 08:33:44 leibovic shellid is read only.

Revision 1.19 2006/02/23 07:42:53 leibovic Adding -n option to shell.

Revision 1.18 2006/02/08 12:50:00 leibovic changing NIV\_\-EVNMT to NIV\_\-EVENT.

Revision 1.17 2006/01/17 14:56:22 leibovic Adaptation de HPSS 6.2.

Revision 1.16 2005/11/28 17:03:00 deniel Added Ce\-CILL headers

Revision 1.15 2005/09/27 09:30:16 leibovic Removing non-thread safe trace buffer.

Revision 1.14 2005/09/27 08:15:13 leibovic Adding traces and changhing readexport prototype.

Revision 1.13 2005/08/12 12:15:33 leibovic Erreur d'init.

Revision 1.12 2005/08/12 11:21:27 leibovic Now, set cat concatenate strings.

Revision 1.11 2005/08/05 07:59:21 leibovic Better help printing.

Revision 1.10 2005/07/29 13:34:28 leibovic Changing \_\-FULL\_\-DEBUG to \_\-DEBUG\_\-SHELL

Revision 1.9 2005/07/26 12:54:47 leibovic Multi-thread shell with synchronisation routines.

Revision 1.8 2005/07/25 12:50:45 leibovic Adding thr\_\-create and thr\_\-join commands.

Revision 1.7 2005/05/11 15:53:37 leibovic Adding time function.

Revision 1.6 2005/05/11 07:25:58 leibovic Escaped char support.

Revision 1.5 2005/05/10 14:02:27 leibovic Fixed bug in log management.

Revision 1.4 2005/05/10 11:38:07 leibovic Adding log initialization.

Revision 1.3 2005/05/10 11:07:21 leibovic Adapting to ganeshell v2.

Revision 1.2 2005/05/09 14:54:59 leibovic Adding if.

Revision 1.1 2005/05/09 12:23:55 leibovic Version 2 of ganeshell.

Definition in file {\bf shell.c}.

\subsection{Define Documentation}
\index{shell.c@{shell.c}!MAX_OUTPUT_LEN@{MAX\_\-OUTPUT\_\-LEN}}
\index{MAX_OUTPUT_LEN@{MAX\_\-OUTPUT\_\-LEN}!shell.c@{shell.c}}
\subsubsection{\setlength{\rightskip}{0pt plus 5cm}\#define MAX\_\-OUTPUT\_\-LEN\ (1024$\ast$1024)}\label{shell_8c_a0}




Definition at line 173 of file shell.c.

Referenced by shell\_\-Solve\-Args(), and shellcmd\_\-set().\index{shell.c@{shell.c}!P@{P}}
\index{P@{P}!shell.c@{shell.c}}
\subsubsection{\setlength{\rightskip}{0pt plus 5cm}\#define P(\_\-mutex\_\-)\ pthread\_\-mutex\_\-lock( \&\_\-mutex\_\- )}\label{shell_8c_a3}




Definition at line 187 of file shell.c.

Referenced by shell\_\-Barrier\-Init().\index{shell.c@{shell.c}!PROMPTSIZE@{PROMPTSIZE}}
\index{PROMPTSIZE@{PROMPTSIZE}!shell.c@{shell.c}}
\subsubsection{\setlength{\rightskip}{0pt plus 5cm}\#define PROMPTSIZE\ 64}\label{shell_8c_a2}




Definition at line 177 of file shell.c.\index{shell.c@{shell.c}!TRACEBUFFSIZE@{TRACEBUFFSIZE}}
\index{TRACEBUFFSIZE@{TRACEBUFFSIZE}!shell.c@{shell.c}}
\subsubsection{\setlength{\rightskip}{0pt plus 5cm}\#define TRACEBUFFSIZE\ 1024}\label{shell_8c_a1}




Definition at line 175 of file shell.c.

Referenced by shell\_\-Execute(), shell\_\-Set\-Dbg\-Lvl(), shell\_\-Set\-Input(), shell\_\-Set\-Layer(), shell\_\-Set\-Line(), shell\_\-Set\-Prompt(), shell\_\-Set\-Shell\-Id(), shell\_\-Set\-Status(), shell\_\-Set\-Verbose(), shell\_\-Solve\-Args(), shellcmd\_\-barrier(), shellcmd\_\-help(), shellcmd\_\-interactive(), shellcmd\_\-quit(), shellcmd\_\-set(), shellcmd\_\-unset(), and shellcmd\_\-varlist().\index{shell.c@{shell.c}!V@{V}}
\index{V@{V}!shell.c@{shell.c}}
\subsubsection{\setlength{\rightskip}{0pt plus 5cm}\#define V(\_\-mutex\_\-)\ pthread\_\-mutex\_\-unlock( \&\_\-mutex\_\- )}\label{shell_8c_a4}




Definition at line 188 of file shell.c.

Referenced by shell\_\-Barrier\-Init().

\subsection{Function Documentation}
\index{shell.c@{shell.c}!shell_BarrierInit@{shell\_\-BarrierInit}}
\index{shell_BarrierInit@{shell\_\-BarrierInit}!shell.c@{shell.c}}
\subsubsection{\setlength{\rightskip}{0pt plus 5cm}int shell\_\-Barrier\-Init (int {\em nb\_\-threads})}\label{shell_8c_a12}


Initialize the barrier for shell synchronization routines. The number of threads to wait for is given as parameter. 

Definition at line 209 of file shell.c.

References P, and V.

Referenced by main().\index{shell.c@{shell.c}!shell_CleanArgs@{shell\_\-CleanArgs}}
\index{shell_CleanArgs@{shell\_\-CleanArgs}!shell.c@{shell.c}}
\subsubsection{\setlength{\rightskip}{0pt plus 5cm}void shell\_\-Clean\-Args (int {\em argc}, char $\ast$$\ast$ {\em in\_\-out\_\-argv}, int $\ast$ {\em in\_\-allocated})}\label{shell_8c_a25}


shell\_\-Clean\-Args: Free allocated arguments.

\begin{Desc}
\item[Parameters:]
\begin{description}
\item[{\em argc}]The number of command line tokens. \item[{\em in\_\-out\_\-argv}]The list of command line tokens (modified). \item[{\em in\_\-allocated}]Indicates which tokens must be freed. \end{description}
\end{Desc}


Definition at line 1090 of file shell.c.

Referenced by shell\_\-Launch(), and shell\_\-Solve\-Args().\index{shell.c@{shell.c}!shell_Execute@{shell\_\-Execute}}
\index{shell_Execute@{shell\_\-Execute}!shell.c@{shell.c}}
\subsubsection{\setlength{\rightskip}{0pt plus 5cm}int shell\_\-Execute (int {\em argc}, char $\ast$$\ast$ {\em argv}, FILE $\ast$ {\em output})}\label{shell_8c_a26}


shell\_\-Execute: Commands dispatcher.

\begin{Desc}
\item[Parameters:]
\begin{description}
\item[{\em argc}]The number of arguments of this command. \item[{\em argv}]The arguments for this command. \item[{\em output}]The output stream of this command.\end{description}
\end{Desc}
\begin{Desc}
\item[Returns:]The returned status of this command. \end{Desc}


Definition at line 1123 of file shell.c.

References command\_\-def\_\-\_\-::command\_\-func, layer\_\-def\_\-\_\-::command\_\-list, command\_\-def\_\-\_\-::command\_\-name, layer\_\-def\_\-t, layer\_\-def\_\-\_\-::setlog\_\-func, shell\_\-Get\-Dbg\-Lvl(), shell\_\-Get\-Layer(), shell\_\-Get\-Verbose(), shell\_\-Print\-Error(), shell\_\-Print\-Trace(), shell\_\-state\_\-t, and TRACEBUFFSIZE.

Referenced by shell\_\-Launch(), shell\_\-Solve\-Args(), shellcmd\_\-if(), and shellcmd\_\-time().\index{shell.c@{shell.c}!shell_GetDbgLvl@{shell\_\-GetDbgLvl}}
\index{shell_GetDbgLvl@{shell\_\-GetDbgLvl}!shell.c@{shell.c}}
\subsubsection{\setlength{\rightskip}{0pt plus 5cm}int shell\_\-Get\-Dbg\-Lvl ({\bf shell\_\-state\_\-t} $\ast$ {\em context})}\label{shell_8c_a36}


shell\_\-Get\-Dbg\-Lvl Get the special variable \$DEBUG\_\-LEVEL and \$DBG\_\-LVL (internal use). 

Definition at line 1558 of file shell.c.

References shell\_\-state\_\-\_\-::debug\_\-level, and shell\_\-state\_\-t.\index{shell.c@{shell.c}!shell_GetInputStream@{shell\_\-GetInputStream}}
\index{shell_GetInputStream@{shell\_\-GetInputStream}!shell.c@{shell.c}}
\subsubsection{\setlength{\rightskip}{0pt plus 5cm}FILE$\ast$ shell\_\-Get\-Input\-Stream ({\bf shell\_\-state\_\-t} $\ast$ {\em context})}\label{shell_8c_a38}


shell\_\-Get\-Input\-Stream Get the input stream for reading commands (internal use). 

Definition at line 1688 of file shell.c.

References shell\_\-state\_\-\_\-::input\_\-stream, and shell\_\-state\_\-t.\index{shell.c@{shell.c}!shell_GetLayer@{shell\_\-GetLayer}}
\index{shell_GetLayer@{shell\_\-GetLayer}!shell.c@{shell.c}}
\subsubsection{\setlength{\rightskip}{0pt plus 5cm}{\bf layer\_\-def\_\-t}$\ast$ shell\_\-Get\-Layer ({\bf shell\_\-state\_\-t} $\ast$ {\em context})}\label{shell_8c_a30}


shell\_\-Get\-Layer: Retrieves the current active layer (internal use). 

Definition at line 1366 of file shell.c.

References shell\_\-state\_\-\_\-::layer, layer\_\-def\_\-t, and shell\_\-state\_\-t.

Referenced by shell\_\-Execute(), and shellcmd\_\-help().\index{shell.c@{shell.c}!shell_GetLine@{shell\_\-GetLine}}
\index{shell_GetLine@{shell\_\-GetLine}!shell.c@{shell.c}}
\subsubsection{\setlength{\rightskip}{0pt plus 5cm}int shell\_\-Get\-Line ({\bf shell\_\-state\_\-t} $\ast$ {\em context})}\label{shell_8c_a43}


shell\_\-Get\-Line Get the special variable \$LINE 

Definition at line 1790 of file shell.c.

References shell\_\-state\_\-\_\-::line, and shell\_\-state\_\-t.

Referenced by shell\_\-Launch(), shell\_\-Print\-Error(), and shell\_\-Print\-Trace().\index{shell.c@{shell.c}!shell_GetPrompt@{shell\_\-GetPrompt}}
\index{shell_GetPrompt@{shell\_\-GetPrompt}!shell.c@{shell.c}}
\subsubsection{\setlength{\rightskip}{0pt plus 5cm}char$\ast$ shell\_\-Get\-Prompt ({\bf shell\_\-state\_\-t} $\ast$ {\em context})}\label{shell_8c_a40}


shell\_\-Get\-Prompt Get the special variable \$PROMPT 

Definition at line 1724 of file shell.c.

References get\_\-var\_\-value(), and shell\_\-state\_\-t.\index{shell.c@{shell.c}!shell_GetStatus@{shell\_\-GetStatus}}
\index{shell_GetStatus@{shell\_\-GetStatus}!shell.c@{shell.c}}
\subsubsection{\setlength{\rightskip}{0pt plus 5cm}int shell\_\-Get\-Status ({\bf shell\_\-state\_\-t} $\ast$ {\em context})}\label{shell_8c_a32}


shell\_\-Get\-Status Get the special variables \$? or \$STATUS (internal use). 

Definition at line 1416 of file shell.c.

References shell\_\-state\_\-t, and shell\_\-state\_\-\_\-::status.\index{shell.c@{shell.c}!shell_GetVerbose@{shell\_\-GetVerbose}}
\index{shell_GetVerbose@{shell\_\-GetVerbose}!shell.c@{shell.c}}
\subsubsection{\setlength{\rightskip}{0pt plus 5cm}int shell\_\-Get\-Verbose ({\bf shell\_\-state\_\-t} $\ast$ {\em context})}\label{shell_8c_a34}


shell\_\-Get\-Verbose Get the special variable \$VERBOSE (internal use). 

Definition at line 1490 of file shell.c.

References shell\_\-state\_\-t, and shell\_\-state\_\-\_\-::verbose.

Referenced by shell\_\-Execute(), shell\_\-Print\-Trace(), and shellcmd\_\-varlist().\index{shell.c@{shell.c}!shell_Init@{shell\_\-Init}}
\index{shell_Init@{shell\_\-Init}!shell.c@{shell.c}}
\subsubsection{\setlength{\rightskip}{0pt plus 5cm}int shell\_\-Init (int {\em verbose}, char $\ast$ {\em input\_\-file}, char $\ast$ {\em prompt}, int {\em shell\_\-index})}\label{shell_8c_a16}


Initialize the shell. The command line for the shell is given as parameter. \begin{Desc}
\item[Parameters:]
\begin{description}
\item[{\em input\_\-file}]the file to read from (NULL if stdin). \end{description}
\end{Desc}


Definition at line 358 of file shell.c.

References NULL, shell\_\-Set\-Dbg\-Lvl(), shell\_\-Set\-Input(), shell\_\-Set\-Prompt(), shell\_\-Set\-Shell\-Id(), shell\_\-Set\-Verbose(), shell\_\-state\_\-t, and verbose.

Referenced by Launch\-Shell(), and main().\index{shell.c@{shell.c}!shell_Launch@{shell\_\-Launch}}
\index{shell_Launch@{shell\_\-Launch}!shell.c@{shell.c}}
\subsubsection{\setlength{\rightskip}{0pt plus 5cm}int shell\_\-Launch ()}\label{shell_8c_a18}


Run the interpreter. 

Definition at line 473 of file shell.c.

References shell\_\-state\_\-\_\-::input\_\-stream, shell\_\-state\_\-\_\-::interactive, MAX\_\-LINE\_\-LEN, shell\_\-Clean\-Args(), shell\_\-Execute(), shell\_\-Get\-Line(), shell\_\-Parse\-Line(), shell\_\-Set\-Line(), shell\_\-Set\-Status(), shell\_\-Solve\-Args(), and shell\_\-state\_\-t.

Referenced by Launch\-Shell(), and main().\index{shell.c@{shell.c}!shell_ParseLine@{shell\_\-ParseLine}}
\index{shell_ParseLine@{shell\_\-ParseLine}!shell.c@{shell.c}}
\subsubsection{\setlength{\rightskip}{0pt plus 5cm}int shell\_\-Parse\-Line (char $\ast$ {\em in\_\-out\_\-line}, char $\ast$$\ast$ {\em out\_\-arglist}, int $\ast$ {\em p\_\-argcount})}\label{shell_8c_a21}


shell\_\-Parse\-Line: Extract an arglist from a command line.

\begin{Desc}
\item[Parameters:]
\begin{description}
\item[{\em in\_\-out\_\-line}]The command line (modified). \item[{\em out\_\-arglist}]The list of command line tokens. \item[{\em p\_\-argcount}]The number of command line tokens.\end{description}
\end{Desc}
\begin{Desc}
\item[Returns:]0 if no errors. \end{Desc}


Definition at line 695 of file shell.c.

Referenced by shell\_\-Launch(), and shell\_\-Solve\-Args().\index{shell.c@{shell.c}!shell_PrintError@{shell\_\-PrintError}}
\index{shell_PrintError@{shell\_\-PrintError}!shell.c@{shell.c}}
\subsubsection{\setlength{\rightskip}{0pt plus 5cm}void shell\_\-Print\-Error ({\bf shell\_\-state\_\-t} $\ast$ {\em context}, char $\ast$ {\em error\_\-msg})}\label{shell_8c_a27}


shell\_\-Print\-Error: Prints an error. 

Definition at line 1257 of file shell.c.

References get\_\-var\_\-value(), shell\_\-Get\-Line(), and shell\_\-state\_\-t.

Referenced by shell\_\-Execute(), shell\_\-Set\-Dbg\-Lvl(), shell\_\-Set\-Input(), shell\_\-Set\-Layer(), shell\_\-Set\-Line(), shell\_\-Set\-Prompt(), shell\_\-Set\-Shell\-Id(), shell\_\-Set\-Status(), shell\_\-Set\-Verbose(), shell\_\-Solve\-Args(), shellcmd\_\-barrier(), shellcmd\_\-help(), shellcmd\_\-interactive(), shellcmd\_\-quit(), shellcmd\_\-set(), shellcmd\_\-unset(), and shellcmd\_\-varlist().\index{shell.c@{shell.c}!shell_PrintTrace@{shell\_\-PrintTrace}}
\index{shell_PrintTrace@{shell\_\-PrintTrace}!shell.c@{shell.c}}
\subsubsection{\setlength{\rightskip}{0pt plus 5cm}void shell\_\-Print\-Trace ({\bf shell\_\-state\_\-t} $\ast$ {\em context}, char $\ast$ {\em msg})}\label{shell_8c_a28}


shell\_\-Print\-Trace: Prints a verbose trace. 

Definition at line 1275 of file shell.c.

References get\_\-var\_\-value(), shell\_\-Get\-Line(), shell\_\-Get\-Verbose(), and shell\_\-state\_\-t.

Referenced by shell\_\-Execute(), shell\_\-Set\-Input(), and shell\_\-Set\-Layer().\index{shell.c@{shell.c}!shell_SetDbgLvl@{shell\_\-SetDbgLvl}}
\index{shell_SetDbgLvl@{shell\_\-SetDbgLvl}!shell.c@{shell.c}}
\subsubsection{\setlength{\rightskip}{0pt plus 5cm}int shell\_\-Set\-Dbg\-Lvl ({\bf shell\_\-state\_\-t} $\ast$ {\em context}, char $\ast$ {\em str\_\-debug\_\-level})}\label{shell_8c_a35}


shell\_\-Set\-Dbg\-Lvl Set the special variables \$DEBUG\_\-LEVEL and \$DBG\_\-LVL 

Definition at line 1500 of file shell.c.

References shell\_\-state\_\-\_\-::debug\_\-level, set\_\-var\_\-value(), shell\_\-Print\-Error(), shell\_\-state\_\-t, and TRACEBUFFSIZE.

Referenced by shell\_\-Init(), and shellcmd\_\-set().\index{shell.c@{shell.c}!shell_SetInput@{shell\_\-SetInput}}
\index{shell_SetInput@{shell\_\-SetInput}!shell.c@{shell.c}}
\subsubsection{\setlength{\rightskip}{0pt plus 5cm}int shell\_\-Set\-Input ({\bf shell\_\-state\_\-t} $\ast$ {\em context}, char $\ast$ {\em file\_\-name})}\label{shell_8c_a37}


shell\_\-Set\-Input Set the input for reading commands and set the value of \$INPUT and \$INTERACTIVE.

\begin{Desc}
\item[Parameters:]
\begin{description}
\item[{\em file\_\-name:}]a script file or NULL for reading from stdin. \end{description}
\end{Desc}


Definition at line 1573 of file shell.c.

References shell\_\-state\_\-\_\-::input\_\-stream, shell\_\-state\_\-\_\-::interactive, set\_\-var\_\-value(), shell\_\-Print\-Error(), shell\_\-Print\-Trace(), shell\_\-Set\-Line(), shell\_\-state\_\-t, and TRACEBUFFSIZE.

Referenced by shell\_\-Init(), shellcmd\_\-interactive(), and shellcmd\_\-set().\index{shell.c@{shell.c}!shell_SetLayer@{shell\_\-SetLayer}}
\index{shell_SetLayer@{shell\_\-SetLayer}!shell.c@{shell.c}}
\subsubsection{\setlength{\rightskip}{0pt plus 5cm}int shell\_\-Set\-Layer ({\bf shell\_\-state\_\-t} $\ast$ {\em context}, char $\ast$ {\em layer\_\-name})}\label{shell_8c_a29}


shell\_\-Set\-Layer: Set the current active layer. \begin{Desc}
\item[Returns:]0 if OK. else, an error code. \end{Desc}


Definition at line 1306 of file shell.c.

References shell\_\-state\_\-\_\-::layer, layer\_\-def\_\-t, layer\_\-def\_\-\_\-::layer\_\-name, set\_\-var\_\-value(), shell\_\-Print\-Error(), shell\_\-Print\-Trace(), shell\_\-state\_\-t, and TRACEBUFFSIZE.

Referenced by shellcmd\_\-set().\index{shell.c@{shell.c}!shell_SetLine@{shell\_\-SetLine}}
\index{shell_SetLine@{shell\_\-SetLine}!shell.c@{shell.c}}
\subsubsection{\setlength{\rightskip}{0pt plus 5cm}int shell\_\-Set\-Line ({\bf shell\_\-state\_\-t} $\ast$ {\em context}, int {\em lineno})}\label{shell_8c_a42}


shell\_\-Set\-Line Set the special variable \$LINE 

Definition at line 1761 of file shell.c.

References shell\_\-state\_\-\_\-::line, set\_\-var\_\-value(), shell\_\-Print\-Error(), shell\_\-state\_\-t, and TRACEBUFFSIZE.

Referenced by shell\_\-Launch(), and shell\_\-Set\-Input().\index{shell.c@{shell.c}!shell_SetPrompt@{shell\_\-SetPrompt}}
\index{shell_SetPrompt@{shell\_\-SetPrompt}!shell.c@{shell.c}}
\subsubsection{\setlength{\rightskip}{0pt plus 5cm}int shell\_\-Set\-Prompt ({\bf shell\_\-state\_\-t} $\ast$ {\em context}, char $\ast$ {\em str\_\-prompt})}\label{shell_8c_a39}


shell\_\-Set\-Prompt Set the special variable \$PROMPT 

Definition at line 1703 of file shell.c.

References set\_\-var\_\-value(), shell\_\-Print\-Error(), shell\_\-state\_\-t, and TRACEBUFFSIZE.

Referenced by shell\_\-Init(), and shellcmd\_\-set().\index{shell.c@{shell.c}!shell_SetShellId@{shell\_\-SetShellId}}
\index{shell_SetShellId@{shell\_\-SetShellId}!shell.c@{shell.c}}
\subsubsection{\setlength{\rightskip}{0pt plus 5cm}int shell\_\-Set\-Shell\-Id ({\bf shell\_\-state\_\-t} $\ast$ {\em context}, int {\em shell\_\-index})}\label{shell_8c_a41}


shell\_\-Set\-Shell\-Id Set the special variable \$SHELLID 

Definition at line 1735 of file shell.c.

References set\_\-var\_\-value(), shell\_\-Print\-Error(), shell\_\-state\_\-t, and TRACEBUFFSIZE.

Referenced by shell\_\-Init().\index{shell.c@{shell.c}!shell_SetStatus@{shell\_\-SetStatus}}
\index{shell_SetStatus@{shell\_\-SetStatus}!shell.c@{shell.c}}
\subsubsection{\setlength{\rightskip}{0pt plus 5cm}int shell\_\-Set\-Status ({\bf shell\_\-state\_\-t} $\ast$ {\em context}, int {\em returned\_\-status})}\label{shell_8c_a31}


shell\_\-Set\-Status Set the special variables \$? and \$STATUS. 

Definition at line 1378 of file shell.c.

References set\_\-var\_\-value(), shell\_\-Print\-Error(), shell\_\-state\_\-t, shell\_\-state\_\-\_\-::status, and TRACEBUFFSIZE.

Referenced by shell\_\-Launch(), shell\_\-Solve\-Args(), and shellcmd\_\-set().\index{shell.c@{shell.c}!shell_SetVerbose@{shell\_\-SetVerbose}}
\index{shell_SetVerbose@{shell\_\-SetVerbose}!shell.c@{shell.c}}
\subsubsection{\setlength{\rightskip}{0pt plus 5cm}int shell\_\-Set\-Verbose ({\bf shell\_\-state\_\-t} $\ast$ {\em context}, char $\ast$ {\em str\_\-verbose})}\label{shell_8c_a33}


shell\_\-Set\-Verbose Set the special variable \$VERBOSE. 

Definition at line 1426 of file shell.c.

References set\_\-var\_\-value(), shell\_\-Print\-Error(), shell\_\-state\_\-t, TRACEBUFFSIZE, and shell\_\-state\_\-\_\-::verbose.

Referenced by shell\_\-Init(), and shellcmd\_\-set().\index{shell.c@{shell.c}!shell_SolveArgs@{shell\_\-SolveArgs}}
\index{shell_SolveArgs@{shell\_\-SolveArgs}!shell.c@{shell.c}}
\subsubsection{\setlength{\rightskip}{0pt plus 5cm}int shell\_\-Solve\-Args (int {\em argc}, char $\ast$$\ast$ {\em in\_\-out\_\-argv}, int $\ast$ {\em out\_\-allocated})}\label{shell_8c_a24}


shell\_\-Solve\-Args: Interprets arguments if they are vars or commands.

\begin{Desc}
\item[Parameters:]
\begin{description}
\item[{\em argc}]The number of command line tokens. \item[{\em in\_\-out\_\-argv}]The list of command line tokens (modified). \item[{\em out\_\-allocated}]Indicates which tokens must be freed.\end{description}
\end{Desc}
\begin{Desc}
\item[Returns:]0 if no errors. \end{Desc}


Definition at line 804 of file shell.c.

References get\_\-var\_\-value(), MAX\_\-OUTPUT\_\-LEN, shell\_\-Clean\-Args(), shell\_\-Execute(), shell\_\-Parse\-Line(), shell\_\-Print\-Error(), shell\_\-Set\-Status(), shell\_\-state\_\-t, and TRACEBUFFSIZE.

Referenced by shell\_\-Launch().\index{shell.c@{shell.c}!shellcmd_barrier@{shellcmd\_\-barrier}}
\index{shellcmd_barrier@{shellcmd\_\-barrier}!shell.c@{shell.c}}
\subsubsection{\setlength{\rightskip}{0pt plus 5cm}int shellcmd\_\-barrier (int {\em argc}, char $\ast$$\ast$ {\em argv}, FILE $\ast$ {\em output})}\label{shell_8c_a53}




Definition at line 2306 of file shell.c.

References shell\_\-Print\-Error(), and TRACEBUFFSIZE.\index{shell.c@{shell.c}!shellcmd_help@{shellcmd\_\-help}}
\index{shellcmd_help@{shellcmd\_\-help}!shell.c@{shell.c}}
\subsubsection{\setlength{\rightskip}{0pt plus 5cm}int shellcmd\_\-help (int {\em argc}, char $\ast$$\ast$ {\em argv}, FILE $\ast$ {\em output})}\label{shell_8c_a44}




Definition at line 1802 of file shell.c.

References command\_\-def\_\-\_\-::command\_\-help, layer\_\-def\_\-\_\-::command\_\-list, command\_\-def\_\-\_\-::command\_\-name, layer\_\-def\_\-t, layer\_\-def\_\-\_\-::layer\_\-name, shell\_\-Get\-Layer(), shell\_\-Print\-Error(), and TRACEBUFFSIZE.\index{shell.c@{shell.c}!shellcmd_if@{shellcmd\_\-if}}
\index{shellcmd_if@{shellcmd\_\-if}!shell.c@{shell.c}}
\subsubsection{\setlength{\rightskip}{0pt plus 5cm}int shellcmd\_\-if (int {\em argc}, char $\ast$$\ast$ {\em argv}, FILE $\ast$ {\em output})}\label{shell_8c_a45}




Definition at line 1879 of file shell.c.

References shell\_\-Execute().\index{shell.c@{shell.c}!shellcmd_interactive@{shellcmd\_\-interactive}}
\index{shellcmd_interactive@{shellcmd\_\-interactive}!shell.c@{shell.c}}
\subsubsection{\setlength{\rightskip}{0pt plus 5cm}int shellcmd\_\-interactive (int {\em argc}, char $\ast$$\ast$ {\em argv}, FILE $\ast$ {\em output})}\label{shell_8c_a46}




Definition at line 1973 of file shell.c.

References NULL, shell\_\-Print\-Error(), shell\_\-Set\-Input(), and TRACEBUFFSIZE.\index{shell.c@{shell.c}!shellcmd_print@{shellcmd\_\-print}}
\index{shellcmd_print@{shellcmd\_\-print}!shell.c@{shell.c}}
\subsubsection{\setlength{\rightskip}{0pt plus 5cm}int shellcmd\_\-print (int {\em argc}, char $\ast$$\ast$ {\em argv}, FILE $\ast$ {\em output})}\label{shell_8c_a49}




Definition at line 2179 of file shell.c.\index{shell.c@{shell.c}!shellcmd_quit@{shellcmd\_\-quit}}
\index{shellcmd_quit@{shellcmd\_\-quit}!shell.c@{shell.c}}
\subsubsection{\setlength{\rightskip}{0pt plus 5cm}int shellcmd\_\-quit (int {\em argc}, char $\ast$$\ast$ {\em argv}, FILE $\ast$ {\em output})}\label{shell_8c_a52}




Definition at line 2278 of file shell.c.

References shell\_\-Print\-Error(), and TRACEBUFFSIZE.\index{shell.c@{shell.c}!shellcmd_set@{shellcmd\_\-set}}
\index{shellcmd_set@{shellcmd\_\-set}!shell.c@{shell.c}}
\subsubsection{\setlength{\rightskip}{0pt plus 5cm}int shellcmd\_\-set (int {\em argc}, char $\ast$$\ast$ {\em argv}, FILE $\ast$ {\em output})}\label{shell_8c_a47}




Definition at line 2003 of file shell.c.

References concat(), is\_\-authorized\_\-varname(), MAX\_\-OUTPUT\_\-LEN, my\_\-atoi(), set\_\-var\_\-value(), shell\_\-Print\-Error(), shell\_\-Set\-Dbg\-Lvl(), shell\_\-Set\-Input(), shell\_\-Set\-Layer(), shell\_\-Set\-Prompt(), shell\_\-Set\-Status(), shell\_\-Set\-Verbose(), and TRACEBUFFSIZE.\index{shell.c@{shell.c}!shellcmd_time@{shellcmd\_\-time}}
\index{shellcmd_time@{shellcmd\_\-time}!shell.c@{shell.c}}
\subsubsection{\setlength{\rightskip}{0pt plus 5cm}int shellcmd\_\-time (int {\em argc}, char $\ast$$\ast$ {\em argv}, FILE $\ast$ {\em output})}\label{shell_8c_a51}




Definition at line 2231 of file shell.c.

References NULL, print\_\-timeval, shell\_\-Execute(), and time\_\-diff().\index{shell.c@{shell.c}!shellcmd_unset@{shellcmd\_\-unset}}
\index{shellcmd_unset@{shellcmd\_\-unset}!shell.c@{shell.c}}
\subsubsection{\setlength{\rightskip}{0pt plus 5cm}int shellcmd\_\-unset (int {\em argc}, char $\ast$$\ast$ {\em argv}, FILE $\ast$ {\em output})}\label{shell_8c_a48}




Definition at line 2115 of file shell.c.

References free\_\-var(), shell\_\-Print\-Error(), and TRACEBUFFSIZE.\index{shell.c@{shell.c}!shellcmd_varlist@{shellcmd\_\-varlist}}
\index{shellcmd_varlist@{shellcmd\_\-varlist}!shell.c@{shell.c}}
\subsubsection{\setlength{\rightskip}{0pt plus 5cm}int shellcmd\_\-varlist (int {\em argc}, char $\ast$$\ast$ {\em argv}, FILE $\ast$ {\em output})}\label{shell_8c_a50}




Definition at line 2201 of file shell.c.

References print\_\-varlist(), shell\_\-Get\-Verbose(), shell\_\-Print\-Error(), and TRACEBUFFSIZE.